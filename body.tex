\section{Introduction}

TODO

\subsection{Sections summary}

TODO

\subsection{A summary to experts}\label{subsection/a-summary-to-experts}

TODO

\section{Balanced Tensor Product}\label{section/balanced-tensor-product}

\begin{definition} (Balanced Module Category Functor)

  \noindent Let $C, D, E$ be tensor categories. Let $_{C}M_{D}$ be a $C-D$
  bimodule category, $_{D}N_{E}$ be a $D-E$ bimodule category, $_{C}L_{E}$ be a $C-E$ bimodule category.

  \noindent Then a $D$-balanced $C-E$ bimodule category functor (from $M \times N$ to $L$) is a pair
  \[(M \times N \xrightarrow{F} L, \alpha)\]
  where $F$ is a right-exact bilinear (?) $C-E$ bimodule category functor and $\alpha$ is a natural isomorphism between the following two functors
  \[
    \begin{tikzcd}
      M \times D \times N \arrow[r] \arrow[dr] &
      M \times N \arrow[d, Rightarrow, "\alpha"] \arrow[r, "F"] &
      L \\
      & M \times N \arrow[ur, "F"'] \\
    \end{tikzcd}
  \]
  where the object $(m,d,n)$ in $M \times D \times N$ is sent by the first
  upper arrow to $(m \lhd d, n)$, and by the first lower arrow to $(m, d \rhd n)$.
\end{definition}

\begin{remark}
  The triangle and pentagonal identities for $D$ is automatically satisfied
  (see Jin's note 20240906-110000 p.2).
\end{remark}

\begin{definition} (Balanced Natural Transformation)

  \noindent Let $C, D, E$ be tensor categories. Let $_{C}M_{D}$ be a $C-D$
  bimodule category, $_{D}N_{E}$ be a $D-E$ bimodule category, $_{C}L_{E}$ be
  a $C-E$ bimodule category. Let $F, F'$ be $D$-balanced $C-E$ bimodule
  category functors from $M \times N$ to $L$.

  \noindent A $D$-balanced transformation between from $F$ to $F'$ is a
  natural transformation $ F \xrightarrow{\eta} F'$ such that the following
  diagram commutes (in a naturally compatible way with the $C-E$ bimodule
  structures.)

  \[
    \begin{tikzcd}
      A \arrow[r, "\alpha_{m,c,n}"] \arrow[d, "\eta_{m \lhd c, n}"'] &
      B \arrow[d, "\eta_{m, c \rhd n}"] \\
      C \arrow[r, "\alpha_{m,c,n}'"'] & D
    \end{tikzcd}
  \]
\end{definition}

\begin{definition} (Category of Balanced Functors)

  \noindent Let $C, D, E$ be tensor categories. Let $_{C}M_{D}$ be a $C-D$
  bimodule category, $_{D}N_{E}$ be a $D-E$ bimodule category, $_{C}L_{E}$ be
  a $C-E$ bimodule category. Let $F, F'$ be $D$-balanced $C-E$ bimodule
  category functors from $M \times N$ to $L$.

  \noindent Then the category of balanced functors
  $Fun^{D-bal}_{C,E}(M \times N, L)$ is defined to be the category whose
  objects are $D$-balanced $C-E$ bimodule category functors $M \times N \to L$
  and whose morphisms are $D$-balanced natural transformations.
\end{definition}

\begin{definition} (Balanced Tensor Product)

  \noindent Let $C, D, E$ be tensor categories. Let $_{C}M_{D}$ be a $C-D$
  bimodule category, and $_{D}N_{E}$ be a $D-E$ bimodule category.

  \noindent Then the balanced tensor product $_{C}(M \boxtimes_{D} N)_{E}$ is
  the $C-E$ bimodule category defined as the initial $D$-balanced module
  functor from $M \times N$; i.e. for each map from $M \times N$ to some $C-E$
  bimodule category $L$, there exists a unique $C-E$ bimodule category functor
  $G$ such that the following diagram commutes,

  \[
    \begin{tikzcd}
      M \times N \arrow[r, "F"] \arrow[d, "\boxtimes_{D}"'] & L \\
      _{C}(M \boxtimes_{D} N)_{E} \arrow[ur, "G"']
    \end{tikzcd}
  \]

  \noindent and that it induces an equivalence of category
  \[Fun_{C,E}(M \boxtimes_{D} N, L) \simeq Fun_{C,E}^{D-bal}(M \times N, L).\]
\end{definition}

\begin{remark}
  This definition is a straight generatlization from axXiv:1406.4204. We want
  to understand $_{C}M \boxtimes_{D} N_{E}$ in terms of skein category. Note that that
  paper implements it using category of modules, while KirillovTham2020 and
  Hoek's master's thesis (and another paper by Etingof et al around 2010)
  implements it as a center (for the finite + ss case). 1202.0061 imeplements
  it as $Fun_{D,re}(M^{op},N)$ for finite case. Eventually it's ifeal for us
  to connect our construction back to these cases.
\end{remark}

\section{Skein Construction}\label{section/skein-construction}

\begin{definition} (Pre-Skein Category)

  \noindent Let $C$ be a tensor category, and $M_{C}, _{C}N$ be module
  categories. Define the pre-skein category $p.sk(M,C,N)$ to be the category
  $P$ with $Obj(P) = Obj(M \boxtimes N)$, and the morphism spaces
  $Mor_{P}((m \boxtimes n), (m' \boxtimes n'))$ are defined to be $V / \sim$,
  where
  \[
    V := \oplus_{c,\overline{c} \in Obj(C)} Hom_{M}(m, m' \rhd c) \otimes Hom_{C}(c,\overline{c}) \otimes Hom_{N} (\overline{c} \lhd n, n'),
  \]
  and $\sim$ are generated by the following types of relations
  \begin{align}
    \phi \otimes (\pi' \circ \pi) \otimes \psi - (\phi_{\pi'} \otimes \pi \otimes \psi)\\
    \phi \otimes (\pi' \circ \pi) \otimes \psi - (\phi \otimes \pi' \otimes _{\pi}\psi).
  \end{align}
  The composition
  \[
    \phi'' \otimes \pi'' \otimes \psi'' =
    (\phi' \otimes \pi' \otimes \psi' ) \circ (
    \phi \otimes \pi \otimes \psi ),
  \]
  is defined so that
  \[
    \pi'' = c' \otimes c \xrightarrow{\pi' \otimes \pi} \overline{c'} \otimes \overline{c},
  \]
  \noindent $\phi'' \in Hom_{M}(m, m'' \rhd (c' \otimes c))$ is
  \[
    m \xrightarrow{\phi} m' \rhd c \xrightarrow{\phi' \rhd 1_{c}} \xrightarrow[\sim]{\alpha} m'' \rhd (c' \otimes c).
  \]
  \noindent and $\psi'' \in Hom_{N}((\overline{c}' \otimes \overline{c}) \lhd n, n'')$ is
  \[
    (\overline{c}' \otimes \overline{c}) \lhd n \xrightarrow[\sim]{\alpha} \overline{c}' \lhd (\overline{c} \lhd n) \xrightarrow{1_{\overline{c}'} \lhd \psi} \overline{c}' \lhd n' \xrightarrow{\psi'} n''.
  \]
  % TODO Check that composition descends well to quotients.

\end{definition}

\noindent TODO: Provide a basis for the hom spaces in case of semisimple,
finite, $\otimes$-strict.

\begin{definition} (Skein Category)

  \noindent Let $C$ be a tensor category, and $M_{C}, _{C}N$ be module categories.
  Define the skein category $sk(M,C,N)$ to the Karoubi completion of its pre-skein category
  \[
    sk(M,C,N) := Kar(p.sk(M,C,N)).
  \]

\end{definition}
