\section{Introduction}

Classical theory of algebras admits a categorification to tensor categories
and module categories. They proved to be very useful in several fields:
representation theory \cite{egno/tensor-cats}, quantum field theory (conformal
in $2$ dimensions \cite{schweigert/rcft-tft} \cite{yadav/gv-voa}, topological
in $3$, $4$ dimensions), and topological matters
\cite{kong/topological-order}.

A well known construction that categorifies the tensor product of two algebras
is the Deligne tensor product for two linear categories
\cite{egno/tensor-cats}. The Deligne tensor product was later generalized by
the balanced tensor product \cite{etingof/fusion-cat-and-homotopy}
\cite{douglas/balanced-product}: given a tensor category $C$ and module
categories $M_{C} \,,\, {}_{C}N$, the balanced tensor product $M \boxtimes_{C} N$ is a
linear category that best approximates the Deligne tensor product in the
context of $C$-module categories. In the theory of topological phase, this
construction corresponds to fusing two domains (labeled by $M$ and $N$) over
the domain wall (labeled by $C$) (e.g. \cite{kong/topological-order}). In the
theory ($3$D, fully extended) topological quantum field theory, this provides
the natural composition for the $1$-morphisms in a usual target $3$-category
$TC$ \cite{douglas/dualizable-tensor-categories}. In the theory of vertex
operator algebra (VOA), this construction help study them by relating various
VOAs \cite{gannon/sln-II}, and constructing new VOAs. Given the work
\cite{maurer/computing-center}, it is also interesting to explore the
computational aspect of balanced tensor product, now enabled thanks to the
current work.

As a special case, the Drinfeld center $Z(C)$ for $C$ can be constructed as
$C \boxtimes_{C \boxtimes C^{op}} C$ \cite{kirillov/string-net-tv}
\cite{douglas/dualizable-tensor-categories}. Our main result in this paper
provides a new skein-theoretical proof.
% TODO Should we provide a proof for this in this paper? Maybe in the appendix.

As in the context of classical algebra, such tensor product is first defined
via a universal property. What makes the situation slightly different is that
in the classical theory, a construction for the tensor product is quite
straightforward. In our context, there were known a few constructions too: for
example (1) category of bimodules \cite{douglas/balanced-product}, (2) center
construction \cite{etingof/fusion-cat-and-homotopy}
\cite{kirillov/fact-homo-4d-tqft} \cite{hoek/master}.

However, these models obscure the topological and skein-theoretical nature of
tensor categories. This has two drawbacks: (1) the computation about the
balanced tensor products becomes more difficult once there are multiple
domains to be fused, e.g.
\[ {}_{A}{M}_{B} \boxtimes_{B} {}_{B}{N}_{C} \boxtimes_{{C}} {}_{C}{L}_{D} \ldots
\]
(2) it makes it difficult to see the relation between the fully extended $3$D
topological field theories (targeting $TC$) to the Turaev-Viro state sum.

This note provides a skein-theoretic model for $M \boxtimes_{C} N$, resolving these
drawbacks. In the following work \cite{guu/tv-as-3-functor}, the authors will
use this result to show a well- and long- expected relation between the
Turaev-Viro state sum and the fully extended $3$D TFT targeting $TC$. We
expect the same method will also work for revealing the relation between their
analogues in $4$D, i.e. the Crane-Yetter state sum model, and its
generalization to using Fusion 2-categories
\cite{douglas/fusion-2-cat-4d-tqft}. Such a proof implies the conjecture
(given therein) that, despite the strength and generality of the construction,
it is nevertheless an extended topological field theory, and therefore cannot
detect exotic structures \cite{reutter/no-go-exotic}. Also, this work can
serve as the basis for implementing state sum model with defect (cf
\cite{meusburger/defect-tv}) as a 3-functor.

\subsection*{A note to experts}\label{subsection/a-note-to-experts}

The idea of main theorem provided in this paper is well-expected, at least by
multiple (mathematical) physicists. However, to our best knowledge, it has not
been mathematically proven.

One might expect to see drawings of skeins in this paper, but we deliberately
avoid them due to the excessive time and energy they require. Instead, we
introduce the notation ${}^{m'}_{n'}I^{m}_{n}$, which mimics a skein flowing
from the right to the left, from the initial data $m \boxtimes n$ to the final data
$m' \boxtimes n'$, exchanging an extra ``particle'' during the process (the vertical
stroke in ``$I$''). With this notation, we conveniently have
${}^{m''}_{n''}I^{m'}_{n'} \circ {}^{m'}_{n'}I^{m}_{n} = {}^{m''}_{n''}I^{m}_{n}$.
See more details in remark \ref{remark/skein-nature-of-the-notation-I}.

This work assumes all of the best conditions: finiteness, semisimplicity,
pivotality. It is possible and useful to relax some of them. But for
simplicity, we postpone it to future work.

\section{Balanced Tensor Product}\label{section/balanced-tensor-product}

\begin{definition} (Balanced Module Category Functor)

  \noindent Let $C, D, E$ be tensor categories. Let $_{C}M_{D}$ be a $C-D$
  bimodule category, $_{D}N_{E}$ be a $D-E$ bimodule category, $_{C}L_{E}$ be a $C-E$ bimodule category.

  \noindent Then a $D$-balanced $C-E$ bimodule category functor (from $M \times N$ to $L$) is a pair
  \[(M \times N \xrightarrow{F} L, \alpha)\] where $F$ is a right-exact bilinear
  % TODO Why do we need right-exactness? Does this mean we have to check
  % right-exactness for our result as well?
  $C-E$ bimodule category functor and $\alpha$ is a natural isomorphism between the
  following two functors from $M \times D \times N$ to $L$
  \[
    \begin{tikzcd}
      M \times D \times N \arrow[r] \arrow[dr] &
      M \times N \arrow[d, Rightarrow, "\alpha"] \arrow[r, "F"] &
      L \\
      & M \times N \arrow[ur, "F"'] \\
    \end{tikzcd}
  \]
  where the object $(m,d,n)$ in $M \times D \times N$ is sent by the first
  upper arrow to $(m \lhd d, n)$, and by the first lower arrow to $(m, d \rhd n)$.
\end{definition}

% TODO
% \begin{remark}
%   The triangle and pentagonal identities for $D$ is automatically satisfied
%   [TODO: see Jin's note 20240906-110000 p.2].
% \end{remark}

\begin{definition} (Balanced Natural Transformation)

  \noindent Let $C, D, E$ be tensor categories. Let $_{C}M_{D}$ be a $C-D$
  bimodule category, $_{D}N_{E}$ be a $D-E$ bimodule category, $_{C}L_{E}$ be
  a $C-E$ bimodule category. Let $F, F'$ be $D$-balanced $C-E$ bimodule
  category functors from $M \times N$ to $L$. A $D$-balanced transformation
  from $F$ to $F'$ is a natural transformation $ F \xrightarrow{\eta} F'$ such
  that the following diagram commutes (in a naturally compatible way with the
  $C-E$ bimodule structures.)

  \[
    \begin{tikzcd}
      F(m \lhd c, n) \arrow[r, "\alpha_{m,c,n}"] \arrow[d, "\eta_{m \lhd c, n}"'] &
      F(m, c \rhd n) \arrow[d, "\eta_{m, c \rhd n}"] \\
      F'(m \lhd c, n) \arrow[r, "\alpha_{m,c,n}'"'] &
      F'(m, c \rhd n)
    \end{tikzcd}
  \]
\end{definition}

\begin{definition} (Category of Balanced Functors)

  \noindent Let $C, D, E$ be tensor categories. Let $_{C}M_{D}$ be a $C-D$
  bimodule category, $_{D}N_{E}$ be a $D-E$ bimodule category, $_{C}L_{E}$ be
  a $C-E$ bimodule category. Let $F, F'$ be $D$-balanced $C-E$ bimodule
  category functors from $M \times N$ to $L$.

  \noindent Then the category of balanced functors
  $Fun^{D-bal}_{C,E}(M \times N, L)$ is defined to be the category whose
  objects are $D$-balanced $C-E$ bimodule category functors $M \times N \to L$
  and whose morphisms are $D$-balanced natural transformations.
\end{definition}

% TODO: In this definition, I need M, N are semisimple already. But add
% a later remark saying that this can be circumvented.
\begin{definition} (Balanced Tensor Product)

  \noindent Let $C, D, E$ be tensor categories. Let $_{C}M_{D}$ be a $C-D$
  bimodule category, and $_{D}N_{E}$ be a $D-E$ bimodule category.

  \noindent Then the balanced tensor product $_{C}(M \boxtimes_{D} N)_{E}$ is
  the $C-E$ bimodule category defined as the initial $D$-balanced module
  functor from $M \times N$; i.e. for each map from $M \times N$ to some $C-E$
  bimodule category $L$, there exists a unique $C-E$ bimodule category functor
  $G$ such that the following diagram commutes,

  \[
    \begin{tikzcd}
      M \times N \arrow[r, "F"] \arrow[d, "\boxtimes_{D}"'] & L \\
      _{C}(M \boxtimes_{D} N)_{E} \arrow[ur, "G"']
    \end{tikzcd}
  \]

  \noindent and that it induces an equivalence of category
  \[Fun_{C,E}(M \boxtimes_{D} N, L) \simeq Fun_{C,E}^{D-bal}(M \times N, L).\]
\end{definition}

\begin{remark}
  This definition is a straight generalization from
  \cite{douglas/balanced-product}. We want to understand $_{C}M \boxtimes_{D} N_{E}$
  in terms of skein category. Note that that paper implements it using
  category of modules, while \cite{etingof/fusion-cat-and-homotopy},
  \cite{kirillov/fact-homo-4d-tqft} and \cite{hoek/master} implements it as a
  center (for the finite + ss case). \cite{davydov/picard} implements it as
  $Fun_{D,re}(M^{op},N)$ for finite case. Eventually it's ideal for us to
  connect our construction back to these cases.
\end{remark}

\section{Skein Construction}\label{section/skein-construction}

\begin{definition} (Pre-Skein Category)

  \noindent Let $C$ be a finite, semisimple, pivotal tensor category. Let
  $M_{C}$ and $ _{C}N$ be finite and semisimple module categories. Define the
  pre-skein category $p.sk(M,C,N)$ to be the category $P$ with
  $Obj(P) = Obj(M \boxtimes N)$, and the morphism spaces $Mor_{P}((m \boxtimes n), (m' \boxtimes n'))$
  are defined to be the vector space $(V \,/ \sim)$, where
  \[
    V := \oplus_{c,\overline{c} \in Obj(C)} Hom_{M}(m, m' \lhd c) \otimes Hom_{C}(c,\overline{c}) \otimes Hom_{N} (\overline{c} \rhd n, n'),
  \]
  and the relations $\sim$ are generated by the following types of relations
  [TODO: Add a picture.]
  \begin{align}
    \phi \otimes (\overline{\pi} \circ \pi) \otimes \psi &- (\phi_{\pi} \otimes \overline{\pi} \otimes \psi) \\
    \phi \otimes (\overline{\pi} \circ \pi) \otimes \psi &- (\phi \otimes \pi \otimes {}_{\overline{\pi}}\psi),
  \end{align}
  where
  \begin{align}
    \phi_{\pi}  &:= \left( m \xrightarrow{\phi} m' \lhd c \xrightarrow{1_{m'} \lhd \pi} m' \lhd c' \right)\\
    {}_{\overline{\pi}}\psi &:= \left( \overline{c} \rhd n \xrightarrow{\overline{\pi} \rhd 1_{n}} \overline{\overline{c}} \rhd n \xrightarrow{\psi} n' \right)
  \end{align}


  \noindent [TODO: Add a picture for composition.]

  \noindent We define the composition on the level of $V$. It is
  straightforward to check that it descends properly to $(V \, / \sim)$ using
  the $2$-categorically compositional nature of tensor categories:
  % TODO: If have time, add a complete proof for this claim, perhaps in the appendix.
  The composition
  $(\phi' \otimes \pi' \otimes \psi' ) \circ (\phi \otimes \pi \otimes \psi )$
  is defined to be $(\phi'' \otimes \pi'' \otimes \psi'')$ where \\
  $\pi'' \in Hom_{C}(c' \otimes c, \overline{c}' \otimes \overline{c})$ is equal to
  \[
    c' \otimes c \xrightarrow{\pi' \otimes \pi} \overline{c'} \otimes \overline{c},
  \]
  \noindent $\phi'' \in Hom_{M}(m, m'' \lhd (c' \otimes c))$ is equal to
  \[
    m \xrightarrow{\phi} m' \lhd c \xrightarrow{\phi' \lhd 1_{c}} (m'' \lhd c') \lhd c \xrightarrow[\sim]{\alpha} m'' \lhd (c' \otimes c),
  \]
  \noindent $\psi'' \in Hom_{N}((\overline{c}' \otimes \overline{c}) \rhd n, n'')$ is equal to
  \[
    (\overline{c}' \otimes \overline{c}) \rhd n \xrightarrow[\sim]{\alpha} \overline{c}' \rhd (\overline{c} \rhd n) \xrightarrow{1_{\overline{c}'} \rhd \psi} \overline{c}' \rhd n' \xrightarrow{\psi'} n''.
  \]

  \noindent Finally, extend the definition of the morphism spaces to those between
  general objects via direct sum.

\end{definition}

\noindent [TODO: At least add a picture to the notation $I$. It will make the
whole paper much easier to understand. Remembeer to change the notice to
expert.]
\begin{notation} (${}^{m'}_{n'}I^{m}_{n}$)

  \noindent Denote the equivalence class of the vector
  $(\phi \otimes \pi \otimes \psi) \in V$ by
  \[\III{m'}{n'}{m}{n}{\phi}{\psi}{c}{\pi}{c'}.\]
  When $m' = m$ and $n' = n$, we omit the primed symbols and write
  \[
    \III{}{}{m}{n}{\phi}{\psi}{c}{\pi}{c'} :=
    \III{m}{n}{m}{n}{\phi}{\psi}{c}{\pi}{c'}.
  \]
  When $\pi = 1_{c}$ (thus $c' = c$), we abbreviate it further to
  \[
    \II{m'}{n'}{m}{n}{\phi}{\psi}{c} :=
    \III{m'}{n'}{m}{n}{\phi}{\psi}{c}{1_{c}}{c},
  \]
  and
  \[
    \II{}{}{m}{n}{\phi}{\psi}{c} :=
    \III{m}{n}{m}{n}{\phi}{\psi}{c}{1_{c}}{c}.
  \]
\end{notation}

\begin{remark}\label{remark/skein-nature-of-the-notation-I} (Skein Nature of the Notation ${}^{m'}_{n'}I^{m}_{n}$)

  \noindent Informally yet instructively, it is helpful to view
  $\III{m'}{n'}{m}{n}{\phi}{\psi}{c}{\pi}{c'}$ as a skein flowing from the
  right to the left, starting from $m \boxtimes n$ to $m' \boxtimes n'$,
  passing through $\phi \boxtimes \psi$; during the process, the upper strain
  emits a particle $c$, which transforms to via $\pi$ to $\overline{c}$, and
  hits the lower strain. Hence under the defined composition rule (see the
  equation for $\phi'' \otimes \pi'' \otimes \psi''$ above), we have
  \[
    \III{m''}{n''}{m'}{n'}{\phi'}{\psi'}{c'}{\pi'}{\overline{c}'} \circ
    \III{m'}{n'}{m}{n}{\phi}{\psi}{c}{\pi}{\overline{c}} =
    \III{m''}{n''}{m}{n}{\phi''}{\psi'}{c' \otimes c}{\pi''}{\overline{c}' \otimes \overline{c}}.
  \]
\end{remark}

\begin{remark}\label{remark/hom-space-reduction} (Hom Space Reduction)

  \noindent By the relations in the definition, the transformation
  \[
    \pi = 1_{\overline{c}} \circ \pi = \pi \circ 1_{c}
  \]
  can be absorbed into each of the strain, so
  \[
    \II{m'}{n'}{m}{n}{\phi}{{}_{\pi}\psi}{c} =
    \III{m'}{n'}{m}{n}{\phi}{{}_{\pi}\psi}{c}{1_{c}}{c} =
    \III{m'}{n'}{m}{n}{\phi}{\psi}{c}{\pi}{c'} =
    \III{m'}{n'}{m}{n}{\phi_{\pi}}{\psi}{c'}{1_{c'}}{c'} =
    \II{m'}{n'}{m}{n}{\phi_{\pi}}{\psi}{c'}.
  \]

  \noindent Furthermore, if $c \simeq \oplus_{i=1}^{l} c_{i}$, then
  \[
    \II{m'}{n'}{m}{n}{\phi}{\psi}{c} = \II{m'}{n'}{m}{n}{\phi}{\psi}{\oplus_{i=1}^{l} c_{i}} = \sum_{j=1}^{l} \III{m'}{n'}{m}{n}{\phi_{j}}{\psi}{c_{j}}{\iota_{j}}{\oplus_{i=1}^{l} c_{i}} =
    \sum_{j=1}^{l}\II{m'}{n'}{m}{n}{\phi_{j}}{\psi_{j}}{c_{j}},
  \]
  where $\iota_{j}$ denotes the $j$-th embedding map from $c_{j}$ to
  $\oplus_{i=1}^{l}c_{i}$, and the $\phi_{j}, \psi_{j}$'s denote the $j$-th
  projection of $\phi, \psi$ respectively. In particular, when
  $c \simeq x^{\oplus l}$, the this gives a reduction from
  \[
    Hom_{M}(m, m' \rhd x^{\oplus l}) \otimes Hom_{C}(x^{\oplus l}, x^{\oplus l}) \otimes Hom_{N} (x^{\oplus l} \lhd n, n')
  \]
  to
  \[
    Hom_{M}(m, m' \rhd x) \otimes Hom_{C}(x, x) \otimes Hom_{N} (x \lhd n, n').
  \]
  It is helpful to regard the result as the ``inner product'' of $\phi$ and $\psi$.
\end{remark}

\begin{proposition} (Basis Theorem) \label{proposition/basis-theorem}

  \noindent Let $C$ be a finite, semisimple tensor category, and
  $M_{C}, _{C}N$ be finite, semisimple module categories over $C$. Thus the
  vector spaces $Hom_{M}(m, m' \lhd c), Hom_{N}(c \rhd n, n')$ have finite bases
  $\beta(m, m' \lhd c), \beta(c \rhd n, n')$ respectively. Then the hom space
  $Hom_{p.sk(M,C,N)}(m \boxtimes n, m' \boxtimes n')$ has a linear basis
  \[
    \bigsqcup_{c \in Irr(C)} \beta(m, m' \lhd c) \times \beta(c \rhd n, n').
  \]
\end{proposition}
\begin{proof}
  This follows immediately from the reductions given in \ref{remark/hom-space-reduction}.
\end{proof}

\begin{definition} (Karoubi Completion)

  \noindent Let $C$ be a category. The Karoubi completion $Kar(C)$ of $C$ is defined to be the category with
  \[
    Obj(Kar(C)) = \{(c, f) \,|\, c \in Obj(C), f \in End_{C}(c), f = f^{2}\}
  \] and
  \[
    Mor_{Kar(C)}((c,f), (c', f')) = \{\overline{f} \in Hom_{C}(c,c') \,|\, \overline{f}f = \overline{f} = f'\overline{f}\},
  \]
  with the obvious composition rule.
\end{definition}

\begin{definition}\label{definition/skein-category} (Skein Category)

  \noindent Let $C$ be a tensor category. Let $M_{C}$ and $_{C}N$ be module
  categories. Define the skein category $sk(M,C,N)$ to the Karoubi completion
  of its pre-skein category
  \[
    sk(M,C,N) := Kar(p.sk(M,C,N)).
  \]
  \noindent Thus, a typical object of the skein category $sk(M,C,N)$ is an
  idempotent skein $\II{m}{n}{m}{n}{\phi}{\psi}{c}$. The skein category is obviously a linear category.

  More generally, for any $n \in \mathbb{N}$, let
  $C_{0}, C_{1}, \ldots, C_{n}$ be finite, semisimple, pivotal tensor
  categories. Let
  ${}_{C_{0}}M^{1}_{C_{1}}, \, {}_{C_{1}}M^{2}_{C_{2}}, \ldots, {}_{C_{n-1}}M^{n}_{C_{n}}, \, $
  be finite, semisimple bimodule categories. One can in a similar way define the $C_{0}-C_{n}$ bimodule category
  \[
    sk(M^{1}, C_{1}, M^{2}, C_{2}, M^{3}, \ldots, C_{n-1}, M^{n}).
  \]
  \noindent Let $F^{i}: M^{i} \to M'^{i}$ be a $C_{i-1}-C_{i}$ bimodule category functor. Then it naturally induces a linear functor
  \[
    F^{i}:
    sk(M^{1}, C_{1}, M^{2}, C_{2}, M^{3}, \ldots M^{i} \ldots, C_{n-1}, M^{n})
    \to
    sk(M^{1}, C_{1}, M^{2}, C_{2}, M^{3}, \ldots M'^{i} \ldots, C_{n-1}, M^{n}).
  \]
\end{definition}

\noindent The skein category $sk(M,C,N)$ is equivalent to $M \boxtimes_{C} N$ (proven
in \ref{lemma/main-lemma}, \ref{theorem/main-theorem}). A necessary ingredient
is the canonical map $\boxtimes_{C}$ given in the defining universal property.

\begin{definition}\label{definition/canonical-map} (Canonical Map $\boxtimes_{C}$)

  \noindent Let $C$ be a tensor category, and $M_{C}, _{C}N$ be module categories. Define the functor
  \[
    M \times N \xrightarrow{\boxtimes_{C}} sk(M,C,N)
  \]
  to send the object $(m,n)$ to the object $\II{m}{n}{m}{n}{1_{m}}{1_{n}}{1_{1}}$, and the morphism
  \[
    (m,n) \xrightarrow{(\phi, \psi)} (m', n')
  \]
  to the morphism $\II{m'}{n'}{m}{n}{\phi}{\psi}{1_{1}}$.
\end{definition}

\noindent [TODO: add a picture for the following sketched proof, especially the
part that uses the (co)unit maps.]

\noindent To quickly convince the reader that the skein category $sk(M,C,N)$
is indeed $M \boxtimes_{C} N$; we consider $N$ to be the trivial left
$C$-module category $_{C}C$, and argue that
$M \xrightarrow{\boxtimes_{C}} sk(M,C,C)$ is an isomorphism if $C$ is a
finite, pivotal tensor category and if $M$ is semisimple. It is
straightforward to see that it is indeed fully faithful, so it suffices to
show that it is essentially surjective. A typical object in the codomain
$sk(M,C,C)$ is some idempotent skein $\II{}{}{m}{c}{\phi}{\psi}{\overline{c}}$
(without loss of generality, assume $m$ to be simple). We contend that this
object is isomorphic to $\II{}{}{m \lhd c}{1}{\mu_{\phi, \psi}}{1}{1}$, where
\[
  \mu_{\phi,\psi} := m \lhd c \xrightarrow{\phi \lhd c} (m \lhd \overline{c}) \lhd c \xrightarrow[\sim]{\alpha} \xrightarrow{m \lhd \psi} m \lhd c.
\]
Indeed, the isomorphism is provided by the following two morphisms
$\II{mc}{1}{m}{c}{n}{u}{c^{\star}}$ and $\II{m}{c}{mc}{1}{1}{1}{c}$, where $n$
denotes the counit and $u$ denotes the unit for $c$. [TODO: Check carefully
and provide full proof (but comment the proof out later)].

\hfill\break
\noindent We prove another lemma that will be useful later.

\begin{lemma}\label{lemma/I-provides-subobject}

  \noindent Let $C$ be a tensor category, and $M_{C}, _{C}N$ be module
  categories. Then any typical object $\II{}{}{m}{n}{\phi}{\psi}{c}$ (recall
  that it is an idempotent skein) in $sk(M,C,N)$ is a subobject of the
  canonical object $\boxtimes_{C}((m,n)) = \II{}{}{m}{n}{1_{m}}{1_{n}}{1}$.
\end{lemma}
\begin{proof}
  Indeed, both the inclusion arrow $\iota$ and the surjection arrow $\pi$ are
  given by the skein $\II{}{}{m}{n}{1_{m}}{1_{n}}{1}.$
\end{proof}

\section{Proof of Equivalence}\label{section/proof-of-equivalence}

We prove our main theorem (\ref{theorem/main-theorem}) in this section, justifying that the skein
construction $sk(M,C,N)$ is isomorphic to $M \boxtimes_{C} N$, and the obvious
generalization to the case of more bimodule categories.

\begin{lemma}\label{lemma/construction-of-theta} (Construction of $\Theta$)

  \noindent Given the assumptions in the main lemma (\ref{lemma/main-lemma}),
  there exists a linear functor
  \[
    \Theta: Fun(sk(M,C,N), L) \to Fun^{C-bal}(M \times N, L).
  \]
\end{lemma}

\noindent We construct $\Theta$ explicitly in the proof.

\begin{proof}
  \noindent (Object) Let $G$ be an object of the domain. Define $\Theta(G)$ to
  be $F := G \circ \boxtimes_{C} \in Fun(M \times N, L)$. We shall provide the
  balanced structure $\alpha$ for $F$, so that $(F, \alpha)$ is $C$-balanced. 
  We need to provide the $C$-balanced data for $F$
  \[
    \alpha_{m,c,n}: F(m \lhd c, n) = G(\II{}{}{mc}{n}{1}{1}{1}) \xrightarrow{\sim} G(\II{}{}{m}{cn}{1}{1}{1}) = F(m, c \rhd n),
  \]
  which is clearly satisfied by $G(\II{m}{cn}{mc}{n}{1}{1}{c}).$ So defined $\alpha$ is clearly natural.

  \noindent (Morphism) Let $G \xrightarrow{\eta} G'$ be a morphism in
  $Fun(sk(M,C,N), L)$. Its image under $\Theta$ is simply the horizontal
  composition $\eta \star (1_{\boxtimes_{C}})$. The remaining commutativity to be checked
  % TODO: cf p11 of Jin's note 20240906-110000
  is a direct consequence of $\eta$'s naturality.
\end{proof}

\begin{lemma}\label{lemma/theta-is-faithful} ($\Theta$ is faithful) The linear
  functor $\Theta$ (\ref{lemma/construction-of-theta}) is faithful.
\end{lemma}

\begin{proof}
  (We use the same notation found in this section.) This amounts to showing
  that the following map is an injective linear map:
  \[
    (G \xrightarrow{\eta} G') \mapsto (F \xrightarrow{\Theta(\eta) = \eta \star (1_{\boxtimes_{C}})} F').
  \]
  It is clearly linear. For injectivity, we notice that whenever we have
  linear functors
  \[
    X \xrightarrow{f} Y,\quad Y \xrightarrow{g, g'} Z,
  \]
  and a linear natural transformation $\eta: g \to g'$, then the map $(\eta \mapsto \eta \star 1_{f})$ is injective is equivalent to
  \[
    (\forall x \in Obj(X), \eta_{f(x)} = 0) \Rightarrow (\forall y \in Obj(Y), \eta_{y} = 0).
  \]
  This holds if $f$ is surjective on objects. However, in our case
  $f = \boxtimes_{C}$ is not as strong. Fortunately, clearly it also holds if
  $f$ is almost-surjective, in the sense that each $y \in Obj(Y)$ has an
  $x \in Obj(X)$ such that $y$ is a subobject of $f(X)$. Indeed,
  \[
    1_{g(y)} = g(1_{y}) = g(\pi_{y} \circ \iota_{y}),
  \]
  so
  \[
    (g(y) \xrightarrow{\eta_{y}} g'(y)) = \eta_{y} \circ 1_{g(y)} = \eta_{y} \circ g(\pi \circ \iota) = g(\pi) \circ \eta_{f(x)} \circ g(\iota) = g(\pi) \circ 0 \circ g(\iota) = 0.
  \]
  This applies to our case by putting $f = \boxtimes_{C}$ and $g = G$, because
  each object $\II{}{}{m}{n}{\phi}{\psi}{c}$ is clearly a subobject of
  $\boxtimes_{C}(m,n) = \II{}{}{m}{n}{1}{1}{1}$. Therefore, $\Theta$ is faithful.
\end{proof}

\begin{lemma}\label{lemma/theta-is-full} ($\Theta$ is full)
  The linear functor $\Theta$ (\ref{lemma/construction-of-theta}) is full.
\end{lemma}

\begin{proof}
  We need to show that for any $C$-balanced natural transformation
  \[
    \nu: G \circ \boxtimes_{C} = \Theta(G) \to \Theta(G') = G' \circ \boxtimes_{C}
  \]
  there is $\mu: G \to G'$ such that $\nu = \mu \star 1_{\boxtimes_{C}}$. The data $\nu$ are the maps
  \[
    \nu_{(m,n)}: G(\II{}{}{m}{n}{1}{1}{1}) \to G'(\II{}{}{m}{n}{1}{1}{1}).
  \]
  We only need to extend these data to all objects in $sk(M,C,N)$, i.e. define compatible maps
  \[
    \nu_{\II{}{}{m}{n}{\phi}{\psi}{c}}: G(\II{}{}{m}{n}{\phi}{\psi}{c}) \to G'(\II{}{}{m}{n}{\phi}{\psi}{c}).
  \]
  It is straightforward to check that the following works:
  \[
    \nu_{\II{}{}{m}{n}{\phi}{\psi}{c}}:= G(\II{}{}{m}{n}{\phi}{\psi}{c})
    \xrightarrow{G(\iota)}
    G(\II{}{}{m}{n}{1}{1}{1})
    \xrightarrow{\nu_{(m,n)}}
    G'(\II{}{}{m}{n}{1}{1}{1})
    \xrightarrow{G'(\pi)}
    G'(\II{}{}{m}{n}{\phi}{\psi}{c}),
  \]
  where $\iota$ and $\pi$ are the inclusion and projection (see
  Lemma \ref{lemma/I-provides-subobject}).
\end{proof}

\noindent To prove that $\Theta$ is essentially surjective, we need the following
lemma.

\begin{lemma} (Images of skeins) \label{lemma/image-of-skein}
  % TODO: Discussion: Can we relax conditions? We need to use the basis theorem,
  % so it seems that we must require semisimplicity here.

  \noindent
  Let $C$ be a finite, semisimple, pivotal tensor category, $L$ a linear
  category, $M_{C}$ and $_{C}N$ be finite and semisimple module categories,
  and $(F,\alpha) \in Fun^{C-bal}(M \times N, L)$. For each object
  $\II{m'}{n'}{m}{n}{\phi}{\psi}{c}$ in $sk(M,C,N)$, define
  $\tilde{F}(\II{m'}{n'}{m}{n}{\phi}{\psi}{c})$ to be
  \[
    F(m,n)
    \xrightarrow{F(\phi \times 1)}
    F(m' \lhd c, n)
    \xrightarrow[\sim]{\alpha}
    F(m', c \rhd n)
    \xrightarrow{F(1 \times \psi)}
    F(m',n').
  \]
  Suppose we have two identical skeins $\II{m'}{n'}{m}{n}{\phi}{\psi}{c}$ and $ \II{m'}{n'}{m}{n}{\phi'}{\psi'}{c'}$ in
  $sk(M,C,N)$, then
  \[
    \tilde{F}(\II{m'}{n'}{m}{n}{\phi}{\psi}{c}) = \tilde{F}(\II{m'}{n'}{m}{n}{\phi'}{\psi'}{c'}).
  \]
  Moreover, $\tilde{F}$ preserves compositions, i.e.
  \[
    \tilde{F}(\II{m''}{n''}{m'}{n'}{\overline{\phi}}{\overline{\psi}}{\overline{c}} \circ \II{m'}{n'}{m}{n}{\phi}{\psi}{c})
    = \tilde{F}(\II{m''}{n''}{m'}{n'}{\overline{\phi}}{\overline{\psi}}{\overline{c}})
    \circ
    \tilde{F}(\II{m'}{n'}{m}{n}{\phi}{\psi}{c}).
  \]
\end{lemma}

\begin{proof}
  To prove the first statement, use semisimplicity and the basis theorem
  (\ref{proposition/basis-theorem}). [TODO: Provide a detailed explanation.]
  To prove the second statement, note that the left-hand-side is
  $\tilde{F}(\II{m''}{n'}{m}{n}{\overline{\phi} \otimes \phi}{\overline{\psi} \otimes \psi}{\overline{c} \otimes c}),$ which is
  \begin{multline*}
    F(m,n)
    \xrightarrow{F(\phi \times 1)}
    F(m' \lhd c, n)
    \xrightarrow{F((\overline{\phi} \lhd 1) \times 1)} \\
    F((m'' \lhd \overline{c}) \lhd c, n)
    \xrightarrow[\sim]{}
    F((m'' \lhd (\overline{c} \otimes c)), n)
    \xrightarrow[\sim]{\alpha}
    F(m'', (\overline{c} \otimes c) \rhd n)
    \xrightarrow[\sim]{}
    F(m'', \overline{c} \rhd (c \rhd n)) \\
    \xrightarrow{F(1 \times (1 \rhd \psi))}
    F(m'', \overline{c} \rhd n')
    \xrightarrow{F(1 \times \overline{\psi})}
    F(m'',n'').
  \end{multline*}
  On the other hand, the right-hand-side is
  \begin{multline*}
    F(m,n)
    \xrightarrow{F(\phi \times 1)}
    F(m' \lhd c, n)
    \xrightarrow[\sim]{\alpha} \\
    F(m', c \rhd n)
    \xrightarrow{F(1 \times \psi)}
    F(m', n')
    \xrightarrow{F(\overline{\phi} \times 1)}
    F(m'' \rhd \overline{c}, n') \\
    \xrightarrow[\sim]{\alpha}
    F(m'', \overline{c} \lhd n')
    \xrightarrow{F(1 \times \overline{\psi})}
    F(m'',n'').
  \end{multline*}
  To prove that they are equal, we can omit their first and their last arrows. Note that the composed arrow in left-hand-side $F(m' \lhd c, n) \to F(m', \overline{c} \lhd (c \lhd n))$ is, by the naturality of $\alpha$, equal to
  \[
    F(m' \lhd c, n)
    \xrightarrow[\sim]{\alpha}
    F(m', c \rhd n)
    \xrightarrow{F(\overline{\phi} \times 1)}
    F(m'' \lhd \overline{c}, c \rhd n)
    \xrightarrow[\sim]{\alpha}
    F(m'', \overline{c} \rhd (c \rhd n)).
  \]
  Compose this with
  \[
    F(m'', \overline{c} \rhd (c \rhd n))
    \xrightarrow{F(1 \times (1 \rhd \psi))}
    F(m', \overline{c} \rhd n'),
  \]
  then we get
  \[
    F(m' \lhd c, n)
    \xrightarrow[\sim]{\alpha}
    F(m', c \rhd n)
    \xrightarrow{F(\overline{\phi} \times 1)}
    F(m'' \rhd \overline{c}, c \lhd n)
    \xrightarrow{F(1 \times \psi)}
    F(m'' \lhd \overline{c}, n'),
  \]
  which is equal to
  \[
    F(m' \lhd c, n)
    \xrightarrow[\sim]{\alpha}
    F(m', c \rhd n)
    \xrightarrow{F(\overline{\phi} \times \psi)}
    F(m'' \lhd \overline{c}, n').
  \]
  So both sides are equal.
\end{proof}

\begin{lemma}\label{lemma/theta-is-essentially-surjective} ($\Theta$ is essentially surjective)
  The linear functor $\Theta$ (\ref{lemma/construction-of-theta}) is essentially surjective.
\end{lemma}

\begin{proof}
  Let $(F, \alpha) \in Fun^{C-bal}(M \times N, L)$. It suffices to construct $G \in Fun(sk(M,C,N), L)$ such $\Theta(G) \simeq (F,\alpha)$.

  Recall that $L$ is an abelian category (so each $L$-morphism has an image),
  $F: M \times N \to L$ is a linear functor, and that
  \[
    F(m \lhd c, n) \xrightarrow[\sim]{\alpha_{m,c,n}} F(m, c \rhd n).
  \]

  \noindent ($G$ on objects) Recall the definition and properties of
  $\tilde{F}$ in \ref{lemma/image-of-skein}. Define
  $G(\II{}{}{m}{n}{\phi}{\psi}{c})$ to be the image (in $L$) of the
  $L$-morphism $\tilde{F}(\II{}{}{m}{n}{\phi}{\psi}{c})$. In particular, the
  image is a subobject and a quotient of $F(m,n)$.

  \noindent ($G$ on morphisms) We use $\tilde{F}$ again. Let
  $\II{m'}{n'}{m}{n}{\overline{\phi}}{\overline{\psi}}{\overline{c}}$ be a
  morphism from $\II{}{}{m}{n}{\phi}{\psi}{c}$ to
  $\II{}{}{m'}{n'}{\phi'}{\psi'}{c'}$. Define
  $G(\II{m'}{n'}{m}{n}{\overline{\phi}}{\overline{\psi}}{\overline{c}})$ to be
  the map induced by
  \[
    \tilde{F}(\II{m'}{n'}{m}{n}{\overline{\phi}}{\overline{\psi}}{\overline{c}}): F(m,n) \to F(m',n').
  \]
  To justify this, we must show that
  \[
    ker(
    \tilde{F}(\II{}{}{m'}{n'}{\phi'}{\psi'}{c'})
    \circ
    \tilde{F}(\II{m'}{n'}{m}{n}{\overline{\phi}}{\overline{\psi}}{\overline{c}})
    )
    \supseteq
    ker(\tilde{F}(\II{}{}{m}{n}{\phi}{\psi}{c})).
  \]
  By lemma \ref{lemma/image-of-skein}, $\tilde{F}$ respects compositions, so
  \[
    ker(
    \tilde{F}(\II{}{}{m'}{n'}{\phi'}{\psi'}{c'})
    \circ
    \tilde{F}(\II{m'}{n'}{m}{n}{\overline{\phi}}{\overline{\psi}}{\overline{c}})
    )
    =
    ker(
    \tilde{F}(\II{}{}{m'}{n'}{\phi'}{\psi'}{c'}
    \circ
    \II{m'}{n'}{m}{n}{\overline{\phi}}{\overline{\psi}}{\overline{c}})
    ).
  \]
  Then by the definition of $sk(M,N,C)$ and Karoubi completion,
  \[
    ker(
    \tilde{F}(\II{}{}{m'}{n'}{\phi'}{\psi'}{c'}
    \circ
    \II{m'}{n'}{m}{n}{\overline{\phi}}{\overline{\psi}}{\overline{c}})
    )
    =
    ker(
    \tilde{F}(
    \II{m'}{n'}{m}{n}{\overline{\phi}}{\overline{\psi}}{\overline{c}}
    \circ
    \II{}{}{m}{n}{\phi}{\psi}{c}
    )
    ).
  \]
  The final step is completed by using the composing property of $\tilde{F}$ again and the fact that
  $ker(a \circ b) \supseteq ker(b)$.
\end{proof}

\begin{lemma} (Main Lemma) \label{lemma/main-lemma}

  \noindent Let $C$ be a finite, semisimple, pivotal tensor category. Let
  $M_{C}, \, {}_{C}N$ be finite, semisimple $C$-module categories. Then the
  canonical map (\ref{definition/canonical-map}) $\boxtimes_{C}$:
  $M \times N \to sk(M,C,N)$ satisfies the universal property in the
  definition of $M \boxtimes_{C} N$. In particular, we have an equivalence of
  categories
  \[
    sk(M,C,N) \simeq M \boxtimes_{C} N.
  \]
\end{lemma}

\begin{proof}
  Let $L$ be a linear category. We only need to show that $\boxtimes_{C}$
  induces an equivalence of categories
  \[
    Fun(sk(M,C,N), L) \xrightarrow[\sim]{\Theta} Fun^{C-bal}(M \times N, L).
  \]
  We constructed $\Theta$ in (\ref{lemma/construction-of-theta}), proved that
  $\Theta$ is faithfulness in (\ref{lemma/theta-is-faithful}), is full in
  (\ref{lemma/theta-is-full}), and is essentially surjective in
  (\ref{lemma/theta-is-essentially-surjective}).
\end{proof}


\begin{theorem} (Main Theorem: Skein Construction of Balanced Tensor Product) \label{theorem/main-theorem}

  \noindent (1) Let $C, D, E$ be finite, semisimple, pivotal tensor
  categories. Let ${}_{C}M_{D}, \, {}_{D}N_{E}$ be finite, semisimple bimodule
  categories. Then the canonical map (\ref{definition/canonical-map})
  $\boxtimes_{D}$: $M \times N \to sk(M,D,N)$ satisfies the universal property
  in the definition of ${}_{C}M_{D} \boxtimes_{D} {}_{D}N_{E}$. In particular,
  we have an equivalence of $C-E$ bimodule categories.
  \[
    {}_{C}sk(M,D,N)_{E} \simeq {}_{C}M_{D} \boxtimes_{D} {}_{D}N_{E}.
  \]

  \noindent (2) More generally, for any $n \in \mathbb{N}$, let
  $C_{0}, C_{1}, \ldots, C_{n}$ be finite, semisimple, pivotal tensor categories. Let
  ${}_{C_{0}}M^{1}_{C_{1}}, \, {}_{C_{1}}M^{2}_{C_{2}}, \ldots,
  {}_{C_{n-1}}M^{n}_{C_{n}}, \,
  $ be finite, semisimple bimodule categories.
  Then we
  have an equivalence of $C_{0}-C_{n}$ bimodule categories.
  \[
    {}_{C_{0}}sk(M^{1},C_{1},M^{2},C_{2}, \ldots, C_{n-1}, M^{n})_{C_{n}}
    \simeq
    {}_{C_{0}}(M^{1}
    \boxtimes_{C_{1}}
    M^{2}
    \boxtimes_{C_{2}}
    M^{3}
    \ldots
    \boxtimes_{C_{n-1}}
    M^{n})_{C_{n}}.
  \]
  \noindent (3) Moreover, in addition to the previous part, if
  $F^{i}: M^{i} \to M'^{i}$ is a $C_{i-1}-C_{i}$ bimodule category functor,
  then the naturally induced linear functor (cf.
  \ref{definition/skein-category})
  \[
    F^{i}:
    sk(M^{1}, C_{1}, M^{2}, C_{2}, M^{3}, \ldots M^{i} \ldots, C_{n-1}, M^{n})
    \to
    sk(M^{1}, C_{1}, M^{2}, C_{2}, M^{3}, \ldots M'^{i} \ldots, C_{n-1}, M^{n}),
  \]
  corresponds to the functor
  \[
    F^{i}:
    M^{1} \boxtimes_{C_{1}} M^{2} \boxtimes_{C_{2}} M^{3} \boxtimes_{C_{3}} \ldots M^{i} \ldots \boxtimes_{C_{n-1}} M^{n}
    \to
    M^{1} \boxtimes_{C_{1}} M^{2} \boxtimes_{C_{2}} M^{3} \boxtimes_{C_{3}} \ldots M'^{i} \ldots \boxtimes_{C_{n-1}} M^{n}.
  \]
  under the equivalence.
\end{theorem}

\begin{proof}
  The first part is proved by restricting the proof of lemma \ref{lemma/main-lemma} to $C-E$ bimodule maps.
  The second part follows directly from induction and the first part. The third part is obvious.
\end{proof}

\begin{corollary}\label{corollary/skein-category-is-abelian} (Skein Categories are Abelian)

  \noindent For any $n \in \mathbb{N}$, let $C_{0}, C_{1}, \ldots, C_{n}$ be
  finite, semisimple, pivotal tensor categories. Let
  \[
    {}_{C_{0}}M^{1}_{C_{1}}, \, {}_{C_{1}}M^{2}_{C_{2}}, \ldots, {}_{C_{n-1}}M^{n}_{C_{n}}, \,
  \]
  be finite, semisimple bimodule categories. Then the skein category
  \[
    sk(M^{1}, C_{1}, M^{2}, C_{2}, M^{3}, \ldots M^{i} \ldots, C_{n-1}, M^{n})
  \]
  is an abelian category.
\end{corollary}
\begin{proof}
  The proof for the case $n=2$ follows from the fact that
  $M^{1} \boxtimes_{C_{1}} M^{2} \simeq Z_{C_{1}}(M^{1} \boxtimes M^{2})$ (cf
  \cite{kirillov/fact-homo-4d-tqft}). The rest follows from induction.
\end{proof}

