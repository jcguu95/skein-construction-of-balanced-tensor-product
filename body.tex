\section{Introduction}


The notion of a balanced tensor product was first given in
\cite{etingof/fusion-cat-and-homotopy}, and mainly developed in
\cite{douglas/balanced-product} and \cite{douglas/dualizable-tensor-categories}. It has different interpretations in various
fields:
\begin{itemize}
  \item In the theory of topological phases, this construction corresponds to
        fusing two domains (labeled by module categories) along the domain
        wall (labeled by tensor categories) \cite{kong/topological-order}.
  \item In the theory of extended topological quantum field theory, it is the
        $1$-categorical structure of a common target $3$-category $TC$
        \cite{douglas/dualizable-tensor-categories}.
  \item In the theory of vertex operator algebras (VOA), this construction helps
        classify the full bulk CFTs (as explained in
        \cite{gannon/exotic-quantum-subgroup}), and constructing new VOAs
        \cite{gannon/sln-II}.
  \item In the theory of tensor categories itself, it closely relates to the
        categorical center construction. In particular, the Drinfeld center
        $Z(C)$ of $C$ is equivalent to $C \boxtimes_{C \boxtimes C^{op}} C$
        \cite{kirillov/string-net-tv}
        \cite{douglas/dualizable-tensor-categories}.
\end{itemize}

Before this work, several algebraic constructions for balanced tensor products
were known, including categories of modules \cite{douglas/balanced-product},
internal Hom spaces \cite{davydov/picard}, and generalized categorical centers
\cite{etingof/fusion-cat-and-homotopy} \cite{kirillov/fact-homo-4d-tqft}
\cite{hoek/master}. These approaches well-captured the algebraic aspects of
balanced tensor products.

In this paper, we provide a topological construction based on skein theory,
which strikes a better mix between algebra and topology. It has a few advantages:

\begin{enumerate}
  \item Its topological nature allows us to use skein diagrams to calculate the
        values of the fully extended $3$D field theories associated to a
        fusion category by the cobordism hypothesis \cite{lurie/tqft}
        \cite{douglas/dualizable-tensor-categories}
        \cite{guu/tv-as-3-functor}.
  \item It generalizes seamlessly to balanced tensor products involving more than
        two module categories (see the picture in definition
        \ref{definition/skein-category}).
      \end{enumerate}

\noindent Finally, we list some of the (potential) applications of this work.

\begin{enumerate}
  \item \textbf{(Skein Nature of eTFTs)} In an upcoming work
        \cite{guu/tv-as-3-functor}, we build on the main result presented here
        to prove the long-anticipated theorem that the Turaev-Viro state sum
        model \cite{viro/turaev-viro-model} naturally and necessarily emerges
        from the 3-functor in the classification of fully extended field
        theories \cite{lurie/tqft} \cite{douglas/dualizable-tensor-categories}.
  \item \textbf{(No-Go: Detection of Exotic Smooth Structures)} We anticipate
        that this approach can be extended to a $4$-dimensional analogue of Turaev-Viro theory, the
        Crane-Yetter model. Additionally, a similar strategy should apply to a
        broader generalization based on fusion 2-categories, as explored in
        \cite{douglas/fusion-2-cat-4d-tqft}. Proving this would lend support
        to the conjecture proposed there, which suggests that despite the
        construction's generality and power, it remains an extended
        topological field theory and therefore cannot detect exotic structures
        \cite{reutter/no-go-exotic}.
  \item \textbf{(Factorization Homology)} The main result of this paper
        provides a simple reproof of the result in
        \cite{kirillov/fact-homo-4d-tqft} that shows equivalence between the
        Crane-Yetter model and factorization homology
        \cite{ayala/factorization-homology} in $4$D (the excision property).
  \item \textbf{(Defected TFTs)} We expect work being useful in the study of
        defect field theories, such as in \cite{meusburger/defect-tv}.
  \item \textbf{(Computations in Tensor / Higher Categories)} Given that the
        Drinfeld center is a specific instance of the balanced tensor product,
        we expect this paper will contribute to recent advancements,
        particularly the explicit computation of centers as explored in
        \cite{maurer/computing-center}, and further develop the computational
        aspects of balanced tensor products.
\end{enumerate}


% \subsection{Results}

% Fix a field $k$, and assume all categories, functors and natural transformations are enriched over $\Vect_k$ [TODO: Figure out a better way to say this sentence]. Let $C$ be a rigid monoidal category, and let $M_C$ and $_{C}N$ be $C$-module categories. We define the pre-skein category $M\otimes_C N$ whose objects are pairs $(m,n)$ with $m\in M$ and $n\in N$ and whose morphisms $(m,n)\to (m',n')$ are certain skeins (see Definition \ref{pre-skein}). Then we have a $C$-balanced functor $M\otimes N\to M\otimes_C N$.

% When $M$, $N$ and their balanced tensor product $M\boxtimes_C N$ are all $k$-linear finite semisimple, we define the skein category $sk(M,C,N)$ as the Cauchy completion of $M\otimes_C N$, which can be constructed explicitly in two steps:
% \begin{enumerate}
%   \item add finite direct sums
%   \item add splittings of idempotents
% \end{enumerate}
% We then obtain a $C$-balanced functor $M\otimes N\to sk(M,C,N)$. The main result in our paper is the following.

% \begin{theorem}\label{sk_bal2}
%   Let $C$ be a tensor category [TODO: (Manuel) or do we just have to require right duals?] and let $M_C$,$_{C}N$ be $k$-linear finite semisimple $C$-module categories. Suppose $M\boxtimes_C N$ is also $k$-linear finite semisimple. Then the
%   $C$-balanced functor $M\otimes N\to sk(M,C,N)$ exhibits $sk(M,C,N)$ as the
%   balanced tensor product $M\boxtimes_C N$.
% \end{theorem}

% \noindent Using results from \cite{douglas/dualizable-tensor-categories} (see Remark \ref{semisimple_douglas/dualizable-tensor-categories}), this has the following immediate Corollary.

% \begin{corollary}
%    Suppose that $k$ has characteristic $0$, $C$ is a finite semisimple tensor category [TODO: (Manuel) or do we just have to require right duals?], and $M_C$,$_{C}N$ are $k$-linear finite semisimple $C$-module categories. Then the $C$-balanced functor $M\otimes N\to sk(M,C,N)$ exhibits
%    $sk(M,C,N)$ as the balanced tensor product $M\boxtimes_C N$.
% \end{corollary}

% [TODO (START/001): Remove this section if we turn out not to solve the non-semisimple case.]
% We then discuss the nonsemisimple case. Here one must define $sk(M,C,N)$ by taking a finite cocompletion of $M\otimes_C N$ which preserves finite colimits in $M$ and $N$ (see Definition \ref{sk_nonsemisimple}). We then obtain the following results.

% \begin{theorem}
% Let $C$ be a rigid monoidal category and let $M_C$, $_{C}N$ be finitely cocomplete module categories. Then the $C$-balanced functor $M\otimes N\to sk(M,C,N)$ presents $sk(M,C,N)$ as the balanced Kelly tensor product of $M$ and $N$ over $C$.
% \end{theorem}

% \begin{corollary}

% Let $C$ be a finite tensor category [TODO: (Manuel) or do we just have to require right duals?], $M_C$, $_{C}N$ finite $k$-linear $C$-module categories. Then $sk(M,C,N)$ is $k$-linear and the $C$-balanced functor $M\otimes N\to sk(M,C,N)$ presents $sk(M,C,N)$ as the balanced tensor product of $M$ and $N$ over $C$.
% \end{corollary}
% [TODO (END/001): Remove this section if we turn out not to solve the non-semisimple case.]

\subsection{Preliminaries}\label{subsection/preliminaries}

Throughout the whole paper, unless mentioned otherwise, we fix a field $k$ and denote by $\Vect_k$ the category of $k$-vector spaces. We assume all categories, functors and natural transformations are enriched over $\Vect_k$. We always write $\otimes$ instead of $\otimes_k$.

\begin{definition} ($M \otimes N$)
Given categories $M,N$ we denote by $M\otimes N$ the category whose objects are pairs $(m,n)$ with $m\in M$ and $n\in N$ and where $Hom_{M\otimes N}((m,n),(m',n'))=Hom_M(m,m')\otimes Hom(n,n')$.
\end{definition}


\begin{definition} (Linear functor, $Lin(M,N)$, $Bilin(M,N)$)
  [TODO: (Jin) Introduce the notation $Fun(M,N)$ in this definition. Explain somewhere the difference between Fun, Lin, and Bilin. Stress that the key difference is the right-exactness. We may consider to use LinRex instead of Lin, as I find it very confusing.]
  A $\Vect_k$-enriched functor $F:M\to N$ is called linear if it is right
  exact, i.e. if it preserves all finite colimits. We denote the category of
  linear functors $M\to N$ by $Lin(M,N)$. A $\Vect_k$-enriched functor
  $M\otimes N \to L$ is called bilinear if it is linear in each variable. We
  denote the category of bilinear functors $M\otimes N \to L$ by
  $Bilin(M\otimes N, L)$.
\end{definition}

\noindent We make the following definition following \cite{douglas/balanced-product}.

\begin{definition} [TODO: Manuel checks this new definition.]
  A $k$-linear category an abelian $\Vect_{k}$-enriched category.
\end{definition}

\begin{definition} \cite{egno/tensor-cats} An object in a $k$-linear category is simple if it has no
  nontrivial subobjects. A $k$-linear category is semisimple if every object
  splits as a direct sum of simple objects. A $k$-linear category $M$ is
  finite if
  \begin{itemize}
    \item $Hom_M(X,Y)$ is finite dimensional, for all objects $X,Y\in M$.
    \item Every object of $M$ has finite length.
    \item $M$ has enough projectives.
    \item $M$ has finitely many isomorphism classes of simple objects.
  \end{itemize}
  A tensor category is a rigid $k$-linear monoidal category, where the
  monoidal structure is given by a bilinear functor.
\end{definition}

\noindent In a finite semisimple $k$-linear category, every object splits as a
finite direct sum of simple objects. When $M,N$ are semisimple, every
$\Vect_{k}$-enriched functor $M\to L$ is automatically exact (thus linear) and
every $\Vect_{k}$-enriched functor $M\otimes N\to L$ is automatically bilinear.

\begin{assumption}
 When $C$ is a tensor category, any $C$-module category $M$ is assumed to be $k$-linear by definition and the structure map $C\otimes M\to M$ is assumed to be bilinear. 
\end{assumption}

\begin{notation} ($m \lhd c$)

  Given a right (left, resp.) $C$-module category $M_C$ ($_{C}N$, resp.), we denote by $m\lhd c$ ($n \rhd n$, resp.) the object in $M$ ($N$, resp.) that results from acting with $c\in C$ on $m\in M$.
  \end{notation}

[TODO: Jin adds a remark collecting similar notations, and give pointers to where they are defined: $M \otimes N$, $M \boxtimes N$, $M \boxtimes_{C} N$; $Fun$, $Lin$, $Bilin$, $Fun^{C-bal}$, $Lin^{C-bal}$,$Bilin^{C-bal}$.]


% [TODO: We may want to uncomment this subsection.]
% \subsection{A Note to Experts}\label{subsection/a-note-to-experts}

% The main theorem provides an algebraic model for balanced tensor products with
% mathematical rigor, utilizing the skein construction. We do not claim to
% introduce novel ideas here: This approach closely mirrors the domain-wall
% perspective in topological phase theory, as outlined in
% \cite{kong/topological-order}. In fact, the essential techniques have been
% used in many places to study the Drinfeld center
% $Z(C) \simeq C \boxtimes_{CC} C$ (e.g. \cite{kirillov/string-net-tv}). This
% work is merely a generalization of it to the context of field theories with
% defects.

% In this work, we assume ideal conditions—finiteness, semisimplicity, and
% rigidity (see \ref{subsection/preliminaries}). While it is both possible and
% beneficial to relax some of these assumptions, for simplicity, we leave such
% generalizations for future work.

\section{Balanced Tensor Product (Algebra)}\label{section/balanced-tensor-product}

\noindent In this section we recall the definition of the balanced tensor product
from \cite{douglas/balanced-product}.

\begin{definition} (Balanced Functor)

  \noindent Let $C$ be a monoidal category, $M_{C}$, $_{C}N$ $C$-module categories, and $L$ a category. \quad Then a
  $C$-balanced functor $M \otimes N\to L$
  is a pair
  \[(M \otimes N \xrightarrow{F} L, \alpha)\] where $F$ is a [TODO: (Manuel) linear functor?] functor and $\alpha$ is a natural isomorphism
  \[
    \begin{tikzcd}
      M \otimes C \otimes N \arrow[r] \arrow[dr] &
      M \otimes N \arrow[d, Rightarrow, "\alpha"] \arrow[r, "F"] &
      L \\
      & M \otimes N \arrow[ur, "F"'] \\
    \end{tikzcd}
  \]
  where the object $(m,c,n)$ in $M \otimes C \otimes N$ is sent by the first
  upper arrow to $(m \lhd c, n)$, and by the first lower arrow to $(m, c \rhd n)$.
\end{definition}

\begin{definition} (Balanced Natural Transformation)

  \noindent Let $C$ be a monoidal category, let $M_{C}$, $_{C}N$ be $C$-module categories and let $L$ be a category. Let $(F,\alpha), (G,\beta)$ be $C$-balanced functors from $M \otimes N$ to $L$. A $C$-balanced natural transformation
  from $(F,\alpha)$ to $(G,\beta)$ is a natural transformation $ F \xrightarrow{\eta} G$ such
  that the following diagram commutes.

  \[
    \begin{tikzcd}
      F(m \lhd c, n) \arrow[r, "\alpha_{m,c,n}"] \arrow[d, "\eta_{m \lhd c, n}"'] &
      F(m, c \rhd n) \arrow[d, "\eta_{m, c \rhd n}"] \\
      G(m \lhd c, n) \arrow[r, "\beta_{m,c,n}"'] &
      G(m, c \rhd n)
    \end{tikzcd}
  \]
\end{definition}

\begin{definition} ($Fun^{C-bal}(M \otimes N, L)$ and $Bilin^{C-bal}(M \otimes N, L)$)

  \noindent Let $C$ be a monoidal category, let $M_{C}$, $_{C}N$ be $C$-module categories and let $L$ be a category. \quad Then the category of $C$-balanced functors
  $Fun^{C{\text -}bal}(M \otimes N, L)$ is defined to be the category whose
  objects are $C$-balanced functors
  $M \otimes N \to L$ and whose morphisms are $C$-balanced natural transformations.

 We denote by $Bilin^{C{\text -}bal}(M \otimes N, L)$ the full subcategory of $Fun^{C{\text -}bal}(M \otimes N, L)$ whose objects are bilinear (thus right exact) $C$-balanced functors.
\end{definition}

\begin{definition}\label{definition/balanced-tensor-product} (Balanced Tensor Product)
  \noindent Let $C$ be a tensor category [TODO: (Manuel) or do we just have to require right duals?], and let $M_C$, $_{C}N$ be $k$-linear module categories over $C$.
  \quad Then the balanced tensor product $M \boxtimes_{C} N$ is a $k$-linear
  category together with a $C$-balanced bilinear functor
  $M\otimes N\to M\boxtimes_{C} N$ such
  that \[Lin(M \boxtimes_{C} N, L) \to Bilin^{C-bal}(M \otimes N, L)\] is an
  equivalence, for any $k$-linear category $L$.
\end{definition}

In \cite{douglas/balanced-product} the authors construct $M\boxtimes_C N$ for any finite tensor category $C$ and any finite $C$-module categories $M, N$, thus establishing existence in this setting. They do this by using the fact that the module categories $M,N$ can be realized as categories of modules over algebra objects $A_M$ and $A_N$ in $C$ and then showing that the category of $A_M-A_N$-bimodule objects in $C$ is $k$-linear and satisfies the required universal property. In the present paper, we will give an alternative construction of $M\boxtimes_C N$ when $M$, $N$, and $M \boxtimes_{C} N$ are semisimple.

\section{Skein Construction (Topology)}\label{section/skein-construction}

\subsection{Pre-skein Category $M\otimes_C N$}

\noindent In this section, we present the pre-skein category $M \otimes_{C} N$
and develop some of its properties.

[TODO: Should we include ``right-dual'' in the definition of preskein category
as Manuel suggested in
https://github.com/jcguu95/skein-construction-of-balanced-tensor-product/issues/7
? Or should we just add a remark explain why the exchanged particle only goes
one way?

Jin thinks we need bi-dual.]

\begin{definition}\label{pre-skein} (Pre-Skein Category)
  For $C$ a monoidal category and $M_C$, $_{C}N$ module categories, we define
  the pre-skein category $M\otimes_C N$ as follows. Its objects are pairs $(m,n)$ with
  $m\in M$ and $n\in N$, which we denote by $m\boxtimes n$. The hom space
  $Hom_{M\otimes_C N}(m\boxtimes n, m'\boxtimes n')$, as a $k$-vector space, is the quotient
  of the $k$-vector space $$V((m,n),(m',n')):=\bigoplus_{c,\overline{c} \in Obj(C)} Hom_{M}(m, m' \lhd c) \otimes Hom_{C}(c,\overline{c}) \otimes Hom_{N} (\overline{c} \rhd n, n')$$
  by the $k$-linear subspace spanned by
  \begin{align}
    \phi \otimes (\overline{\pi} \circ \pi) \otimes \psi &- (\phi_{\pi} \otimes \overline{\pi} \otimes \psi) \label{relation/a} \\
    \phi \otimes (\overline{\pi} \circ \pi) \otimes \psi &- (\phi \otimes \pi \otimes {}_{\overline{\pi}}\psi) \label{relation/b},
  \end{align}
  where
  \begin{align}
    \phi_{\pi}  &:= \left( m \xrightarrow{\phi} m' \lhd c \xrightarrow{1_{m'} \lhd \pi} m' \lhd c' \right) \\
    {}_{\overline{\pi}}\psi &:= \left( \overline{c} \rhd n \xrightarrow{\overline{\pi} \rhd 1_{n}} \overline{\overline{c}} \rhd n \xrightarrow{\psi} n' \right)
  \end{align}
  
    \begin{center}
    \includesvg[width=18cm]{drawing-1}
  \end{center}
  
  \noindent The composition $(\phi' \otimes \pi' \otimes \psi' ) \circ (\phi \otimes \pi \otimes \psi)$ is defined to be $(\phi'' \otimes \pi'' \otimes \psi'')$ where
  \begin{itemize}
    \item
    $\pi'' \in Hom_{C}(c' \otimes c, \overline{c}' \otimes \overline{c})$ is equal to
    \[
      c' \otimes c \xrightarrow{\pi' \otimes \pi} \overline{c'} \otimes \overline{c},
    \]
    \item
    \noindent $\phi'' \in Hom_{M}(m, m'' \lhd (c' \otimes c))$ is equal to
    \[
      m \xrightarrow{\phi} m' \lhd c \xrightarrow{\phi' \lhd 1_{c}} (m'' \lhd c') \lhd c \xrightarrow[\sim]{\alpha} m'' \lhd (c' \otimes c),
    \]
    \item
    \noindent $\psi'' \in Hom_{N}((\overline{c}' \otimes \overline{c}) \rhd n, n'')$ is equal to
    \[
      (\overline{c}' \otimes \overline{c}) \rhd n \xrightarrow[\sim]{\alpha} \overline{c}' \rhd (\overline{c} \rhd n) \xrightarrow{1_{\overline{c}'} \rhd \psi} \overline{c}' \rhd n' \xrightarrow{\psi'} n''.
    \]
  \end{itemize}
  
    \begin{center}
    \includesvg[width=18cm]{drawing-2}
  \end{center}
  It is straightforward to check that this composition law respects the relations defining $Hom_{M\otimes_C N}$.
\end{definition}

\begin{remark}
 
We abuse notation by writing $(\phi\otimes\pi\otimes\psi)$ to denote the corresponding equivalence class in $Hom_{M\otimes_C N}((m,n),(m',n'))$. We say that the morphisms in $M\otimes_C N$ are skeins. [TODO: Manuel: Maybe skeins are just the "pure tensors" and not sums of those?]\end{remark}


\begin{remark}\label{absorb}
  
  Using the defining relations, we see
  that
  \[
    (\phi\otimes\pi\otimes\psi) =
    (\phi_{\pi}\otimes\id_{\bar{c}}\otimes\psi) =
    (\phi\otimes\id_c\otimes{_{\pi}\psi}).
  \]
\end{remark}

\begin{lemma}\label{decompose}

  Any morphism in $M\otimes_C N$ can be written as a sum of composites of
  morphisms of the form $\phi\otimes\id_{1}\otimes\psi:(m,n)\to(m',n')$ and
  $\id_{m\lhd c}\otimes\id_c\otimes\id_{c\rhd n}:(m\lhd c,n)\to(m,c\rhd n).$
\end{lemma}

\begin{proof}
  First, any morphism is a sum of morphisms of the form
  $\phi\otimes\pi\otimes\psi$. By Remark \ref{absorb}, we can take
  $\pi=\id_c$. Now
  $(\phi\otimes\id_c\otimes\psi)
  =
  (\id_{m'}\otimes\id_1\otimes\psi)
  \circ
  (\id_{m'\lhd c}\otimes\id_c\otimes\id_{c\rhd n})
  \circ
  (\phi\otimes\id_1\otimes\id_n)$.
\end{proof}
    
\begin{definition}\label{definition/preskeinification}.
  We define a functor $M\otimes N\to M\otimes_C N$ by the identity on objects,
  and $\phi\otimes\psi\mapsto\phi\otimes\id_1\otimes\psi$ on morphisms. We
  define also morphisms $\beta_{m,c,n}:(m\lhd c,n)\to (m, c \rhd n)$ by
  $\beta_{m,c,n}=\id_{m\lhd c}\otimes\id_c\otimes\id_{c\lhd n}.$
\end{definition}

\begin{lemma}
  The collection of morphisms $\beta_{m,c,n}$ defines a natural transformation
  of functors $M\otimes C \otimes N\to M\otimes_C N$.
\end{lemma}

\begin{proof}
  Given $\phi:m\to m'$, $\psi: n\to n'$ and $\pi:c\to \bar{c}$ we must check
  that the following diagram commutes.
  \[
    \begin{tikzcd}
      {(m\lhd c, n)} & {(m,c\rhd n)} \\
      {(m'\lhd \bar{c}, n')} & {(m',\bar{c}\rhd n')}
      \arrow["{\beta_{m,c,n}}", from=1-1, to=1-2]
      \arrow["{(\phi\lhd\pi)\otimes\id_1\otimes\psi}"', from=1-1, to=2-1]
      \arrow["{\phi\otimes\id_1\otimes(\pi\rhd\psi)}", from=1-2, to=2-2]
      \arrow["{\beta_{m',\bar{c},n'}}"', from=2-1, to=2-2]
    \end{tikzcd}
  \]
  Using the definition of composition and the relations in the definition of
  $Hom_{M\otimes_C N}$ one can check that both composites equal
  $(\phi\lhd\id_c)\otimes\pi\otimes(\id_{\bar{c}}\rhd\psi)$.
\end{proof}

\begin{lemma}\label{balanced}
  If every object in $C$ has a right dual, then the natural transformation
  $\beta$ described above is an isomorphism, so it provides a balancing for
  the functor $M\otimes N\to M\otimes_C N$ given in \ref{definition/preskeinification}.
\end{lemma}

\begin{proof}
  Given $(m,c,n)$ we provide an explicit inverse for $\beta_{m,c,n}$. Let
  $(c^*,\eta,\epsilon)$ denote a right dual for $c$. We define
  $\beta^{-1}_{m,c,n}=(\id_m\lhd\eta)\otimes\id_{c^*}\otimes (\epsilon\rhd\id_n)$.
  To prove that this is inverse to $\beta_{m,c,n}$ we use only one of the
  snake equations. [TODO: Manuel explains it in more details.]
\end{proof}

[TODO: (Manuel) Jin proposes to not make this global assumption, but to call it every
time it's needed. If we end up making this assumption, we need to at least
mention it in the convention section.]
From now on, we assume that any monoidal category $C$ has right duals. The following universal
property characterizes [TODO: (Manuel) Jin is confused what this means.] $M\otimes N \to M\otimes_C N$.

\begin{proposition}\label{univ_bal}
  Let $L$ be a category, and suppose $C$ has right duals. Then composing with
  $M\otimes N \to M\otimes_C N$ induces an equivalence of categories
  $Fun(M\otimes_C N,L)\to Fun^{C-bal}(M\otimes N,L)$.
\end{proposition}
\begin{proof}
  It follows from lemma \ref{surjective}, \ref{faithful} and \ref{full}.
\end{proof}

\begin{remark}
  Hence the pre-skein category $M \otimes_{C} N$ almost satisfies the universal property that defines the balanced tensor product $M \boxtimes_{C} N$, except that it is not abelian (thus not $k$-linear). We fix this by defining the skein category $sk(M,C,N)$ as a certain completion of the pre-skein category.
\end{remark}

\begin{lemma}\label{surjective}
  $Fun(M\otimes_C N,L)\to Fun^{C-bal}(M\otimes N,L)$ is surjective on
  objects.
\end{lemma}

\begin{proof}
  [TODO: Jin will check this proof.]
  Given a $C$-balanced functor $F:M\otimes N \to L$ with balancing $\beta$, we
  must define $\tilde{F}:M\otimes_C N \to L$ such that its composite with
  $M\otimes N\to M\otimes_C N$ is $F$. We must take $\tilde{F}=F$ on objects.
  To define $\tilde{F}$ on morphisms, let $\phi\in Hom_M(m,m'\lhd c)$,
  $\pi\in Hom_C(c,\bar{c})$ and $\psi\in Hom_N(\bar{c}\rhd n,n')$. We define
  $\tilde{F}(\phi\otimes\pi\otimes\psi)$ to be the composite
  \[
    \begin{tikzcd}
      {(m,n)} & {(m'\lhd c,n)} & {(m'\lhd \bar{c},n)} & {(m',\bar{c}\rhd n)} & {(m',n')}
      \arrow["{F(\phi,n)}", from=1-1, to=1-2]
      \arrow["{F(m'\lhd\pi,n)}", from=1-2, to=1-3]
      \arrow["{\beta_{m',\bar{c},n}}", from=1-3, to=1-4]
      \arrow["{F(m',\psi)}", from=1-4, to=1-5]
    \end{tikzcd}.
  \]
  The naturality of $\beta$ with respect to morphisms in $C$ implies that this
  is equal to the composite
  \[
    \begin{tikzcd}
      {(m,n)} & {(m'\lhd c,n)} & {(m', c\rhd n)} & {(m',\bar{c}\rhd n)} & {(m',n')}
      \arrow["{F(\phi,n)}", from=1-1, to=1-2]
      \arrow["{\beta_{m',c,n}}", from=1-2, to=1-3]
      \arrow["{F(m',\pi\rhd n)}", from=1-3, to=1-4]
      \arrow["{F(m',\psi)}", from=1-4, to=1-5]
    \end{tikzcd}.
  \]
  From these two expressions it is clear that this respects the relations in
  the definition of $Hom_{M\otimes_C N}$. The proof that this respects
  composition is a straightforward calculation, using the naturality of
  $\beta$ with respect to morphisms in $M$ and $N$.
  \end{proof}

  \begin{remark}
    Using Lemma \ref{decompose}, the functor $\tilde{F}:M\otimes_C N\to L$
    whose composite with $M\otimes N\to M\otimes_C N$ is the $C$-balanced
    functor $F:M\otimes N\to L$ with balancing $\beta$ is also determined by
    $\tilde{F}(m,n)=F(m,n)$,
    $\tilde{F}(\phi\otimes\id_1\otimes\psi)=F(\phi,\psi)$ and
    $\tilde{F}(\id_{m\lhd c}\otimes\id_c\otimes\id_{c\rhd n})=\beta_{m,c,n}.$
  \end{remark}

\begin{lemma}\label{faithful}
  $Fun(M\otimes_C N,L)\to Fun^{C-bal}(M\otimes N,L)$ is faithful.
\end{lemma}

\begin{proof}
  [TODO: Jin will check this proof.]
  This is immediate from the fact that $M\otimes N \to M\otimes_C N$ is
  surjective on objects.
\end{proof}

\begin{lemma}\label{full}
  $Fun(M\otimes_C N,L)\to Fun^{C-bal}(M\otimes N,L)$ is full.
\end{lemma}

\begin{proof}
  [TODO: Jin will check this proof.]
  Given $\eta:F\to G$ in $Fun^{C-bal}(M\otimes N,L)$ we must define
  $\tilde{\eta}:\tilde{F}\to\tilde{G}$ in $Fun(M\otimes_C N,L)$. Denote by
  $\alpha$ and $\beta$ the balancing of $F$ and $G$, respectively. Since
  $M\otimes N\to M\otimes_C N$ is surjective on objects, we are forced to
  define $\tilde{\eta}_{m,n}=:\eta_{m,n}$ for every object
  $(m,n)\in M\otimes_C N$. But we need to check that this defines a natural
  transformation. By Lemma \ref{decompose}, it's enough to check that it is
  natural with respect to maps of the form $(\phi\otimes\id_1\otimes\psi)$ and
  $(\id_{m\lhd c}\otimes\id_c\otimes\id_{c\rhd n})$. Naturality with respect
  to $(\phi\otimes\id_1\otimes\psi)$ follows directly from the naturality of
  $\eta$. Now $\eta$ is balanced, which means that
  \[
    \begin{tikzcd}
      {F(m\lhd c, n)} & {G(m\lhd c, n)} \\
      {F(m,c\rhd n)} & {G(m,c\rhd n)}
      \arrow["{\eta_{m\lhd c, n}}", from=1-1, to=1-2]
      \arrow["{\alpha_{m,c,n}}"', from=1-1, to=2-1]
      \arrow["{\beta_{m,c,n}}", from=1-2, to=2-2]
      \arrow["{\eta_{m,c\rhd n}}"', from=2-1, to=2-2]
    \end{tikzcd}\]
  commutes. Therefore
  \[
    \begin{tikzcd}
      {\tilde{F}(m\lhd c, n)} & {\tilde{G}(m\lhd c, n)} \\
      {\tilde{F}(m,c\rhd n)} & {\tilde{G}(m,c\rhd n)}
      \arrow["{\tilde{\eta}_{m\lhd c, n}}", from=1-1, to=1-2]
      \arrow["{\tilde{F}(\id_{m\lhd c}\otimes\id_c\otimes\id_{c\rhd n})}"', from=1-1, to=2-1]
      \arrow["{\tilde{G}(\id_{m\lhd c}\otimes\id_c\otimes\id_{c\rhd n})}", from=1-2, to=2-2]
      \arrow["{\tilde{\eta}_{m,c\rhd n}}"', from=2-1, to=2-2]
    \end{tikzcd}
  \]
  commutes, i.e. $\eta$ is natural with respect to
  $(\id_{m\lhd c}\otimes\id_c\otimes\id_{c\rhd n})$.
\end{proof}

\subsection{Skein Category sk(M,C,N)}

\begin{definition} (Skein Category, $sk(M,C,N)$) Let $C$ be a monoidal category with
  right duals, and let $M_C$, $_{C}N$ be module categories over $C$. Define
  the skein category $sk(M,C,N)$ to be the Cauchy completion $\Cau(M \otimes_{C} N)$
  (cf \ref{definition/cauchy-completion/abstract},
  \ref{definition/cauchy-completion/explicit}) of the pre-skein category
  $M \otimes_{C} N$.
\end{definition}

\noindent It has a $C$-balanced functor $M\otimes N \to sk(M,C,N)$, namely the
composite $M\otimes N\to M\otimes_C N\to sk(M,C,N)$ by [TODO: Jin adds reference].

\begin{remark}
An object in $sk(M,C,N)$ is an idempotent matrix $A:(m_1,n_1)\oplus\cdots\oplus (m_k,n_k)\to (m_1,n_1)\oplus\cdots\oplus (m_k,n_k)$ whose entries are skeins, i.e. elements in (???) [TODO: Manuel fixes this part]\end{remark}

\begin{proposition}\label{univ_sk}
  Let $L$ be Cauchy complete and suppose $C$ has right duals. Then composition
  with the $C$-balanced functor $M\otimes N\to sk(M,C,N)$ induces an
  equivalence of categories $$Fun(sk(M,C,N),L)\to Fun^{C-bal}(M\otimes N,L).$$
\end{proposition}

\begin{proof}
  \noindent [TODO: Manuel rewrites the proof. (Jin: I know it is immediate,
  but from \ref{univ_bal} and a fact that you mentioned in the definition of
  Cauchy completion. However, I hope you can separate that fact to a easily
  referable entry ``X'' (so the unfamiliar readers can quickly get to the
  statement)), and say explicitly that \ref{univ_bal} and ``X'' prove this
  statement directly.]
  % Old sketch: From Proposition \ref{univ_bal} and the universal property of
  % the Cauchy completion, we immediately obtain the following universal
  % property characterizing $sk(M,C,N)$.
\end{proof}

\begin{remark}
  Comparing \ref{univ_sk} with \ref{definition/balanced-tensor-product}, the defining universal property of balanced tensor product, we know we are closed to proving that the skein category $sk(M,C,N)$ is indeed equivalent to the balanced tensor product. What is missing so far are
  \begin{enumerate}
    \item $sk(M,C,N)$ is $k$-linear (in particular, abelian) (cf \ref{semisimple}).
    \item The equivalence in \ref{univ_sk} has to be restricted to their
          right-exact counterparts:
          $Lin(M \boxtimes_{C} N, L) \to Bilin^{C-bal}(M \otimes N, L)$ (cf \ref{sk_bal}).
  \end{enumerate}
\end{remark}

\subsection{Proof of the Main Theorem}

We will show that $M\otimes N\to sk(M,C,N)$ presents
$sk(M,C,N)$ as the balanced tensor product $M\boxtimes_C N$ whenever $M$, $N$
and $M\boxtimes_C N$ are finite semisimple.

In \cite{douglas/balanced-product}, the existence of the balanced tensor
product is established, for the cases where $M,N,C$ are finite. The authors also remark that
in this case the universal property described above still holds when $L$ is
just a finitely cocomplete category. We record this in the following lemma.

\begin{lemma}\label{univ_box}
  Let $C$ be a finite tensor category [TODO: (Manuel) or do we just have to require right duals?], $M_C$, $_{C}N$ finite $k$-linear $C$-module
  categories, and $L$ a finitely cocomplete category. Then the restriction
  functor is an equivalence: \[Lin(M \boxtimes_{C} N, L) \xrightarrow{\sim} Bilin^{C-bal}(M \otimes N, L).\]
\end{lemma}

\begin{proof}
  \cite[Remark 3.4]{douglas/balanced-product} [TODO: Manuel adds a clearer proof for this.]
\end{proof}

Recall that \ref{definition/preskeinification} and \ref{balanced} provide a $C$-balanced functor
$M\otimes N \to M\otimes_C N$ provided that $C$ has right duals. Compose it with
$M\otimes_C N \to Fin(M\otimes_C N)$, and we obtain a $C$-balanced functor
$M\otimes N\to Fin(M\otimes_C N)$. [TODO: Jin adds the $\beta$ data explicitly.]

\begin{lemma}\label{univ_finbal}

  Let $C$ be a monoidal category having right duals, let $M_C$ and $_{C}N$ be
  module categories over $C$, and let $L$ be a finitely cocomplete category. Then the
  restriction map $$Lin(Fin(M\otimes_C N),L)\to Fun^{C-bal}(M\otimes N,L)$$ is
  an equivalence.
\end{lemma}

\begin{proof}
  That $Lin(Fin(M\otimes_C N),L) \simeq Fun(M \otimes_{C} N, L)$ follows from
  a claim in \ref{definition/finite-cocompletion}, and that
  $Fun(M \otimes_{C} N, L) \simeq Fun^{C-bal}(M \otimes N, L)$ follows from
  \ref{univ_bal}.
\end{proof}

\begin{proposition}\label{fin_eq_bal}

  Suppose $C$ is a finite tensor category [TODO: (Manuel) or do we just have to require right duals?]
  and $M_C$ and $_{C}N$ are $k$-linear finite
  semisimple module categories. Then the map
  $Fin(M\otimes_C N)\to M\boxtimes_C N$ corresponding to the $C$-balanced
  functor $M\otimes N\to M\boxtimes_C N$ via the equivalence of Lemma
  \ref{univ_finbal} is an equivalence.
\end{proposition}

\begin{proof}
  Since $M,N$ are semisimple, any functor $M\otimes N\to L$ is bilinear. So
  Lemma \ref{univ_box} states
  that \[Lin(M \boxtimes_{C} N, L) \to Fun^{C-bal}(M \otimes N, L)\] is an
  equivalence for any finitely cocomplete $L$. So the $C$-balanced functor
  $M\otimes N\to Fin(M\otimes_C N)$ induces a map
  $M\boxtimes_C N\to Fin(M\otimes_C N)$. It is now easy to check that these
  two functors are inverse to each other. [TODO: Jin makes this proof clearer. Don't say ``now it is easy''.]
\end{proof}

Now we prove that $sk(M,C,N)$ is a finite semisimple $k$-linear category,
whenever $M$, $N$ and $M\boxtimes_C N$ are finite semisimple $k$-linear
categories.

\begin{lemma}\label{semisimple}
  Let $C$ be a tensor category [TODO: (Manuel) or do we just have to require right duals?], and let $M_C$,$_{C}N$ be $k$-linear finite semisimple $C$-module categories. Suppose $M\boxtimes_C N$ is also $k$-linear finite semisimple. [TODO: Manuel wants to call it a theorem, but Jin wants to call it a lemma.] Then $sk(M,C,N)$ is
  $k$-linear finite semisimple.
\end{lemma}

\begin{proof}
  We have a fully faithful inclusion
  $sk(M,C,N)=Cau(M\otimes_C N)\hookrightarrow Fin(M\otimes_C N)$. By
  Proposition \ref{fin_eq_bal}, we have
  $Fin(M\otimes_C N)\simeq M\boxtimes_C N$, because $M$, $N$ are finite
  semisimple. Therefore $sk(M,C,N)$ is finite semisimple, by Proposition
  \ref{cau_semi}.
\end{proof}

\begin{lemma}\label{sk_bal}
  Let $C$ be a tensor category [TODO: (Manuel) or do we just have to require right duals?], and let $M_C$,$_{C}N$ be $k$-linear finite semisimple $C$-module categories. Suppose $M\boxtimes_C N$ is also $k$-linear finite semisimple. Then the
  $C$-balanced functor $M\otimes N\to sk(M,C,N)$ exhibits $sk(M,C,N)$ as the
  balanced tensor product $M\boxtimes_C N$.
\end{lemma}

\begin{proof}
  Recall that proposition \ref{univ_sk} asserts that the restriction map
  \[
   l Fun(sk(M,C,N),L) \xrightarrow{\sim} Fun^{C-bal}(M \otimes N,L)
  \]
  is an equivalence for every Cauchy complete category $L$. Compare this with
  the defining universal property of balanced tensor product (cf
  \ref{definition/balanced-tensor-product}), we need to show that $sk(M,C,N)$
  is $k$-linear (in particular, abelian), and to restrict the equivalence to
  their right-exact counterparts:
  \[
    Lin(sk(M,C,N),L) \xrightarrow{\sim} Bilin^{C-bal}(M \otimes N, L).
  \]
  By \ref{semisimple} we know $sk(M,C,N)$ is $k$-linear, so that resolves the
  first part, and we also know that it is semisimple, so
  \[
    Fun(sk(M,C,N), L) = Lin(sk(M,C,N), L).
  \]
  Finally, $M \otimes N$ is semisimple because $M, N$ are [TODO: Manuel
  explains this part.], so
  \[
    Fun^{C-bal}(M \otimes N, L) = Bilin^{C-bal}(M \otimes N, L).
  \]
\end{proof}

\begin{remark}\label{semisimple_douglas/dualizable-tensor-categories}
  Collecting results from \cite{douglas/dualizable-tensor-categories}, we can
  conclude that $M\boxtimes_C N$ is finite semisimple (so lemma \ref{sk_bal} applies) in any of the following
  situations:
  \begin{itemize}
    \item $C$ is a finite semisimple tensor category and $_{\Vect_k}M_C$,
    $_{C}N_{\Vect_k}$ are separable bimodule categories (\cite[Proposition 2.5.3,
    Theorem 2.5.5]{douglas/dualizable-tensor-categories});
    \item $k$ is perfect, $C$ is a separable tensor category and $M,N$ are
    finite semisimple (\cite[Proposition
    2.5.10]{douglas/dualizable-tensor-categories});
    \item $k$ has characteristic $0$ and $C,M,N$ are finite semisimple
    (\cite[Corollary 2.6.9]{douglas/dualizable-tensor-categories});
  \end{itemize}
\end{remark}

Since we are mostly interested in the characteristic zero case, we record the following direct consequence of lemma \ref{sk_bal}.

\begin{theorem}
   Suppose $k$ has characteristic $0$, $C$ is a finite semisimple tensor category [TODO: (Manuel) or do we just have to require right duals?], and $M_C$,$_{C}N$ are $k$-linear finite semisimple $C$-module categories. Then the $C$-balanced functor $M\otimes N\to sk(M,C,N)$ exhibits
   $sk(M,C,N)$ as the balanced tensor product $M\boxtimes_C N$.
 \end{theorem}

 [TODO: Manuel adds the definition of $sk(M^{1},..,M^{n})$, and more
 importantly, explain why such a definition is more ``intuitive'' than the
 algebraic ones (i.e. those defined by modules over algebras).]

\begin{corollary}
  Suppose that $n \in \mathbb{N}$, $k$ has characteristic $0$,
  $C_{0}, \ldots, C_{n}$ are finite semisimple tensor categories
  [TODO: (Manuel) or do we just have to require right duals?], and
  $_{C_{0}}M^{1}_{C_{1}}, _{C_{1}}M^{2}_{C_{2}}, \ldots, _{C_{n{\text -}1}}M^{n}_{C_{n}}$
  are $k$-linear finite semisimple module categories. Then we have an
  equivalence of $C_{0}{\text -}C_{n}$ module categories
  \[
    sk(M^{1}, C_{1}, M^{2}, C_{2}, \ldots, C_{n{\text -} 1}, M^{n})
    \simeq
    (M^{1} \boxtimes_{C_{1}} M^{2} \boxtimes_{C_{2}} \ldots \boxtimes_{C_{n{\text -}1}} M^{n}).
  \]
\end{corollary}

\appendix
\section{Appendix}
   
\subsection{Completions}

\noindent In this subsection, we collect results about completions of categories.

\begin{definition} \label{definition/cauchy-completion/abstract} (Cauchy Completion (Abstract))
  A category $M$ is Cauchy complete if it has all finite direct sums and all
  idempotents split [TODO: Jin is confused what this means. Manuel will explain]. Given any $M$, we denote by $\Cau(M)$ the Cauchy
  completion of $M$, the smallest subcategory of $Fun(M^{op},\Vect_k)$
  containing all the representables and closed under finite direct sums and
  retracts. Its objects can be identified as those $X\in Fun(M^{op},\Vect_k)$
  such that $\underline{Hom}(X,-):Fun(M^{op},\Vect_k)\to\Vect_k$ preserves all
  small colimits. Alternatively, they are the retracts of finite direct sums
  of representables.

  $Cau(M)$ is Cauchy complete [TODO: Manuel will explain this to Jin], and comes equipped with a functor
  $M\to \Cau(M)$. It is characterized by a universal property. Namely, given
  any Cauchy complete category $L$, the restriction
  map $$Fun(\Cau(M),L)\to Fun(M,L)$$ is an equivalence. A reference is
  \cite[Sections 5.5 and 5.7]{kelly/basic-concepts-enriched}.
\end{definition}

\noindent We recall the usual explicit construction of the Cauchy completion $Cau(M)$ of a ($\Vect_k$-enriched) category $M$ in [TODO: Manuel adds reference].

\begin{definition}

Given a category $M$, we define $Mat(M)$ to be the category whose objects are tuples of objects in $M$. We denote such an object by $m_1\oplus\cdots\oplus m_k$. We then define $$Hom_{Mat(M)}(m_1\oplus\cdots\oplus m_k,n_1\oplus\cdots\oplus n_{\ell})$$ to be the $k$-vector space of $(\ell\times k)$ matrices whose $(i,j)$-th entry is a morphism $m_j\to n_i$ in $M$. Composition is defined by matrix product. \end{definition}

\begin{definition}
  Given a category $M$, we define its Karoubi completion $Kar(M)$ to be the category whose objects are idempotent endomorphisms $m \xrightarrow{p} m$ in $M$, and the hom space $Hom_{Kar(C)}(m \xrightarrow{p} m, n \xrightarrow{q} n)$ is the $k$-linear subspace of $Hom_{M}(m,n)$ consisting of $f$ such that the following diagram commutes
  \[\begin{tikzcd}
	m & n \\
	m & n
	\arrow["f", from=1-1, to=1-2]
	\arrow["p", from=1-1, to=2-1]
	\arrow["f", from=1-1, to=2-2]
	\arrow["q", from=1-2, to=2-2]
	\arrow["f"', from=2-1, to=2-2]
\end{tikzcd}\] commutes.\end{definition}

\begin{definition} \label{definition/cauchy-completion/explicit} (Cauchy Completion, Explicit)

  The explicit construction of the Cauchy completion is given by $Cau(M)=Kar(Mat(M))$. \end{definition}

\noindent So an object in $Cau(M)$ is an idempotent matrix $$m_1\oplus\cdots\oplus m_k \xrightarrow{A} m_1\oplus\cdots\oplus m_k.$$

\noindent We introduce a few lemmas about Cauchy complete categories.

\begin{lemma}\label{direct_sum}
  Suppose we have a fully faithful functor $F:M\hookrightarrow N$ where $M$ is
  Cauchy complete. If $F(x)=a'\oplus b'$ then we have $x=a\oplus b$ and
  isomorphisms $F(a)\simeq a'$, $F(b)\simeq b'$ such that the following
  diagrams commute.
  \[
    \begin{tikzcd}
      {F(a)} & {F(x)} & {F(b)} \\
      {a'} & {F(x)} & {b'}
      \arrow[from=1-1, to=1-2]
      \arrow["\simeq"', from=1-1, to=2-1]
      \arrow["{=}", from=1-2, to=2-2]
      \arrow[from=1-3, to=1-2]
      \arrow["\simeq", from=1-3, to=2-3]
      \arrow[from=2-1, to=2-2]
      \arrow[from=2-3, to=2-2]
    \end{tikzcd} \begin{tikzcd}
      {F(a)} & {F(x)} & {F(b)} \\
      {a'} & {F(x)} & {b'}
      \arrow["\simeq"', from=1-1, to=2-1]
      \arrow[from=1-2, to=1-1]
      \arrow[from=1-2, to=1-3]
      \arrow["{=}", from=1-2, to=2-2]
      \arrow["\simeq", from=1-3, to=2-3]
      \arrow[from=2-2, to=2-1]
      \arrow[from=2-2, to=2-3]
    \end{tikzcd}
  \]
\end{lemma}

\begin{proof}
  [TODO: Jin checks this proof.]

  Denote by $i_{a'}:a'\to F(x)$, $r_{a'}:F(x)\to a'$ and $p_{a'}:F(x)\to F(x)$
  the corresponding inclusion, retraction and idempotent, so that
  $r_{a'}i_{a'}=\id_{a'}$ and $i_{a'} r_{a'}=p_{a'}$, and similarly for $b'$.
  We have additional equations $r_{a'}i_{b'}=0=r_{b'}i_{a'}$ and
  $p_{a'}+p_{b'}=\id_{F(x)}$. Now $F$ is full, so there exists $p_a:x\to x$
  such that $F(p_a)=p_{a'}$ and similarly for $p_b$. Now $M$ is idempotent
  complete, so $p_a$ splits, i.e. we obtain $i_a:a\to x$ and $r_a:x\to a$ such
  that $r_a i_a=\id_a$ and $i_a r_a=p_a$ and similarly for $b$. Now
  $F(p_a+p_b)=p_{a'}+p_{b'}=\id_{F(x)}$. Since $F$ is faithful we get
  $p_a+p_b=\id_x$. This means that $x=a\oplus b$. Finally
  $r_{a'}F(i_a):F(a)\to a'$ and $r_{b'}F(i_b):F(b)\to b'$ are the desired
  isomorphisms.
\end{proof}

\begin{lemma}\label{abelian}
  Suppose we have a fully faithful functor $F:M\hookrightarrow N$ where $M$ is
  Cauchy complete and $N$ is $k$-linear. Suppose every short exact sequence
  splits in $N$. Then $M$ is $k$-linear and every short exact sequence splits
  in $M$.
\end{lemma}

\begin{proof}
  [TODO: Jin checks this proof.]

  We start by showing that $M$ is $k$-linear (in particular, abelian). We now
  show that $M$ has kernels, and these are preserved by $F$. The proof for
  cokernels is dual. So let $x\to y$ be a morphism in $M$. Then $F(x)\to F(y)$ has
  a kernel $k'\to F(x)$ in $N$. The short exact sequence $0\to k'\to F(x)\to c'\to 0$
  splits, where $c'$ is the cokernel of $k'\to F(x)$. So we get $F(x)=k'\oplus c'$.
  Then, by Lemma \ref{direct_sum}, we have $x=k\oplus c$ where $F(k)\simeq k'$ and
  \[
    \begin{tikzcd}
      {F(k)} & {F(x)} \\
      {k'} & {F(x)} \arrow[from=1-1, to=1-2] \arrow["\simeq"', from=1-1, to=2-1] \arrow["{=}", from=1-2, to=2-2] \arrow[from=2-1, to=2-2]
    \end{tikzcd}
  \]
  commutes. This means $F(k)\to F(x)$ is also a kernel of $F(x)\to F(y)$. But
  $F$ is fully faithful, so it reflects limits, hence $k\to x$ is a kernel of
  $x\to y$.

  We will now show that every monomorphism is a kernel. The proof that every
  epimorphism is a cokernel is dual. Let $a\hookrightarrow x$ be a
  monomorphism in $M$. Since $F$ preserves kernels, we know
  $F(a)\hookrightarrow F(x)$ is a monomorphism. Moreover, if $x\to c$ is the
  cokernel of $a\hookrightarrow x$, then $F(x)\to F(c)$ is the cokernel of
  $F(a)\hookrightarrow F(x)$ (because $F$ also preserves cokernels). So $0\to F(a)\to F(x)\to F(c)\to 0$ is exact, so
  $F(x)=F(a)\oplus F(c)$. Then $x=a\oplus c$ (by Lemma \ref{direct_sum}) and
  so $a\to x$ is the kernel of $x\to c$.

  This concludes the proof that $M$ is $k$-linear. Additionally, we showed that $F:M\to N$ is exact. The fact that every short exact sequence splits in $M$ follows easily from
  Lemma \ref{direct_sum}, and the facts that $F$ is exact and every short
  exact sequence splits in $N$.
\end{proof}

\begin{proposition}\label{cau_semi}
  Suppose we have a fully faithful functor $M\hookrightarrow N$,
  where $M$ is Cauchy complete and $N$ is $k$-linear finite semisimple. Then
  $M$ is a $k$-linear finite semisimple category.
\end{proposition}

\begin{proof}
  [TODO: Jin checks this proof.]

  By Lemma \ref{abelian} the category $M$ also $k$-linear and every short exact sequence splits in $M$, which implies that every object in $M$ is projective and also that $F:M\to N$ exact. Since $F$ is faithful and $N$ is finite, all $Hom$ spaces in $M$ are finite dimensional.

  Since $F$ is exact and fully faithful, it preserves and reflects
  monomorphisms, therefore $x\in M$ is simple if and only if $F(x)\in N$ is
  simple. So in particular, given $N$ has finitely many isomorphism classes of
  simple objects, the same is true of $M$.

  If $F(x)=\bigoplus_{i=1}^n x'_i$ with $x'_i$ simple, then by induction and
  Lemma \ref{direct_sum} we have $x=\bigoplus_{i=1}^n x_i$ where
  $F(x_i)\simeq x_i'$ so $x_i$ is simple.
\end{proof}

\begin{definition} \label{definition/finite-cocompletion} ($Fin(M)$, finite cocompletion)
  A category $M$ is finitely cocomplete if it has all finite colimits. Given a
  category $M$, we denote by $Fin(M)$ its finite cocompletion, the
  smallest finitely cocomplete subcategory of $Fun(M^{op},\Vect_k)$ that
  contains all the representables. Its objects can be identified as those
  $X\in Fun(M^{op},\Vect_k)$ such that
  $\underline{Hom}(X,-):Fun(M^{op},\Vect_k)\to \Vect_k$ preserves filtered
  colimits. Alternatively, they are the coequalisers of pairs of morphisms
  between finite coproducts of representables. [TODO: Manuel explains what underlined Hom is. Jin cannot understand this. Manuel says ``Just remove the underline. It means we take the vector space of morphisms and not the set.'' Jin: ``So should it be $Hom_{Fun(M^{op}, Vect_{k})}$?''] [TODO: Manuel provides better references for each claim given here.]

  There is fully
  faithful functor $M\to Fin(M)$ induced by the Yoneda embedding $M \to Fun(M^{op}, \Vect_{k})$. It is characterized by a universal
  property: for any finitely cocomplete $L$, the restriction map
  $Lin(Fin(M),L)\to Fun(M,L)$ is an equivalence. References include
  \cite[Section 5.7]{kelly/basic-concepts-enriched} and \cite[Section
  2.2.1]{lopezfranco/tensor-products}.
\end{definition}

\begin{remark}
  From our descriptions of $Cau(M)$
  (\ref{definition/cauchy-completion/abstract},
  \ref{definition/cauchy-completion/explicit}) and $Fin(M)$
  (\ref{definition/finite-cocompletion}) it is immediate that we have fully
  faithful functors $M\hookrightarrow Cau(M)\hookrightarrow Fin(M)$.
\end{remark}

\subsection{The Non-semisimple Case}

We explain how one can adapt the construction of $sk(M,C,N)$ to the non-semisimple case. 

\begin{definition}
Suppose $C$ is a rigid monoidal category and $M_C, _{C}N$ are finitely cocomplete $C$-module categories. Their balanced Kelly tensor product is a finitely cocomplete category $M\boxtimes^K_C N$ together with a $C$-balanced bilinear functor $M\otimes N\to M\boxtimes^K_C N$ such that precomposition defines an equivalence $$Lin(M\boxtimes^K_C N,L)\to Bilin^{C-bal}(M\otimes N, L)$$ for any finitely cocomplete category $L$.\end{definition}

Suppose $C$ is a rigid monoidal category and $M_C, _{C}N$ are finitely cocomplete $C$-module categories. We explain how to construct $sk(M,C,N)$ so that $M\otimes N\to sk(M,C,N)$ presents it as the balanced Kelly tensor product of $M$ and $N$ over $C$. We follow very closely the procedure adopted in \cite{lopezfranco/tensor-products} to construct the unbalanced Kelly tensor product, starting from $M\otimes N$.

\begin{definition}

A functor $(M\otimes_C N)^{op}\to \Vect_k$ is said to be left exact in each variable if the composite $M^{op}\otimes N^{op}\to (M\otimes_C N)^{op}\to \Vect_k$ is left exact in each variable. In other words it sends finite colimits in $M$ and finite colimits in $N$ to finite limits in $\Vect_k$. We denote by $Bilex[(M\otimes_C N)^{op},\Vect_k]\subset [(M\otimes_C N)^{op},\Vect_k]$ the full subcategory whose objects are those functors which are left exact in each variable.
\end{definition}
 
This subcategory is reflective (see e.g. \cite[Theorem 6.5]{kelly/basic-concepts-enriched}) so that we an adjunction 

\[\begin{tikzcd}
	{[(M\otimes_C N)^{op},\Vect_k]} & {Bilex[(M\otimes_C N)^{op},\Vect_k]}
	\arrow[""{name=0, anchor=center, inner sep=0}, "R", curve={height=-6pt}, from=1-1, to=1-2]
	\arrow[""{name=1, anchor=center, inner sep=0}, "i", curve={height=-6pt}, hook', from=1-2, to=1-1]
	\arrow["\dashv"{anchor=center, rotate=-90}, draw=none, from=0, to=1]
\end{tikzcd}.\] In particular, $Bilex[(M\otimes_C N)^{op},\Vect_k]$ is cocomplete.


\begin{lemma}\label{right_exact_0}
 
The composite $M\otimes N\to M\otimes_C N \to [(M\otimes_C N)^{op},\Vect_k]\to Bilex[(M\otimes_C N)^{op},\Vect_k]$ is right exact in each variable.\end{lemma}

\begin{proof}
 
This is follows from assertion (5.51) in \cite{kelly/basic-concepts-enriched}.\end{proof}

Denote by $K$ the composite $$M\otimes_C N \to [(M\otimes_C N)^{op},\Vect_k]\to Bilex[(M\otimes_C N)^{op},\Vect_k].$$ 

\begin{definition}\label{sk_nonsemisimple} We define $sk(M,C,N)$ as the closure of the image of the functor $$K:M\otimes_C N \to Bilex[(M\otimes_C N)^{op},\Vect_k]$$ under finite colimits. Thus $sk(M,C,N)$ is finitely cocomplete and comes equipped with a functor $M\otimes_C N \to sk(M,C,N)$.\end{definition}

\begin{lemma}\label{right_exact}
The functor $M\otimes N\to M\otimes_C N \to sk(M,C,N)$ is right exact in each variable.
\end{lemma}
\begin{proof}
 This follows directly from Lemma \ref{right_exact_0} and the fact that $sk(M,C,N)\hookrightarrow Bilex[(M\otimes_C N)^{op},\Vect_k]$ reflects colimits.\end{proof}

\begin{definition}
We say that a functor $M\otimes_C N \to L$ is bilinear if the composite $M\otimes N\to M\otimes_C N \to L$ is bilinear (i.e. right exact in each variable). We denote by $Bilin(M\otimes_C N, L)\subset Fun(M\otimes_C N, L)$ the full subcategory whose objects are the biliner functors. \end{definition}
 
 
\begin{proposition}\label{univ_sk_nonsemisimple}
Given a finitely cocomplete category $L$, the precomposition functor $Lin(sk(M,C,N),L)\to Bilin(M\otimes_C N, L)$ is an equivalence.\end{proposition}
\begin{proof} This follows from \cite[Theorem 6.23]{kelly/basic-concepts-enriched}.\end{proof}

\begin{lemma}\label{univ_bilin_bal}
The functor $Bilin(M\otimes_C N,L)\to Bilin^{C-bal}(M\otimes N, L)$ is an equivalence, for any category $L$. \end{lemma}
\begin{proof}
By Lemma \ref{univ_bal} we know that $Fun(M\otimes_C N,K)\to Fun^{C-bal}(M\otimes N, L)$ is an equivalence. By definition, a functor $M\otimes_C N\to L$ is bilinear if and only if the composite $M\otimes N\to M\otimes_C N\to L$ is bilinear, so we obtain an equivalence between the two subcategories.\end{proof}

\begin{theorem}\label{main_nonsemisimple}
 Let $C$ be a rigid monoidal category and let $M_C$, $_{C}N$ be finitely cocomplete module categories. Then the $C$-balanced functor $M\otimes N\to sk(M,C,N)$ presents $sk(M,C,N)$ as the balanced Kelly tensor product of $M$ and $N$ over $C$.\end{theorem}
 \begin{proof}
  
This follows from Proposition \ref{univ_sk_nonsemisimple} and Lemma \ref{univ_bilin_bal}.\end{proof}

\begin{corollary}
 
Let $C$ be a finite tensor category [TODO: (Manuel) or do we just have to require right duals?], $M_C$, $_{C}N$ finite $k$-linear $C$-module categories. Then $sk(M,C,N)$ is $k$-linear and the $C$-balanced functor $M\otimes N\to sk(M,C,N)$ presents $sk(M,C,N)$ as the balanced tensor product of $M$ and $N$ over $C$.
\end{corollary}
\begin{proof}
 
This follows fro Theorem \ref{main_nonsemisimple} and Lemma \ref{univ_box}.\end{proof}
  


 
 

 
 
 
\subsection{Other}
   
    
\begin{notation} (${}^{m'}_{n'}I^{m}_{n}$)

  \noindent Denote the equivalence class of the vector
  $(\phi \otimes \pi \otimes \psi) \in V((m,m'),(n,n'))$ by
  \[\III{m'}{n'}{m}{n}{\phi}{\psi}{c}{\pi}{c'}.\]
  When $m' = m$ and $n' = n$, we omit the primed symbols and write
  \[
    \III{}{}{m}{n}{\phi}{\psi}{c}{\pi}{c'} :=
    \III{m}{n}{m}{n}{\phi}{\psi}{c}{\pi}{c'}.
  \]
  When $\pi = 1_{c}$ (thus $c' = c$), we abbreviate it further to
  \[
    \II{m'}{n'}{m}{n}{\phi}{\psi}{c} :=
    \III{m'}{n'}{m}{n}{\phi}{\psi}{c}{1_{c}}{c},
  \]
  and
  \[
    \II{}{}{m}{n}{\phi}{\psi}{c} :=
    \III{m}{n}{m}{n}{\phi}{\psi}{c}{1_{c}}{c}.
  \]
\end{notation}

\begin{remark}\label{remark/skein-nature-of-the-notation-I} (Skein Nature of the Notation ${}^{m'}_{n'}I^{m}_{n}$)

  \noindent Informally yet instructively, it is helpful to view
  $\III{m'}{n'}{m}{n}{\phi}{\psi}{c}{\pi}{c'}$ as a skein flowing from the right to
  the left, starting from $m \boxtimes n$ to $m' \boxtimes n'$, passing through $\phi \boxtimes \psi$;
  during the process, the upper strain emits a particle $c$ at $\phi$, which
  transforms to via $\pi$ to $\overline{c}$, and hits the lower strain at
  $\psi$. Hence under the defined composition rule (see the equation for
  $\phi'' \otimes \pi'' \otimes \psi''$ above), we have
  \[
    \III{m''}{n''}{m'}{n'}{\phi'}{\psi'}{c'}{\pi'}{\overline{c}'} \circ
    \III{m'}{n'}{m}{n}{\phi}{\psi}{c}{\pi}{\overline{c}} =
    \III{m''}{n''}{m}{n}{\phi''}{\psi'}{c' \otimes c}{\pi''}{\overline{c}' \otimes \overline{c}}.
  \]
\end{remark}

\begin{remark}\label{remark/hom-space-reduction} (Hom Space Reduction)
  \noindent By the relations in the definition, the transformation
  \[
    \pi = 1_{\overline{c}} \circ \pi = \pi \circ 1_{c}
  \]
  can be absorbed into each of the strands, so
  \[
    \II{m'}{n'}{m}{n}{\phi}{{}_{\pi}\psi}{c} =
    \III{m'}{n'}{m}{n}{\phi}{{}_{\pi}\psi}{c}{1_{c}}{c} =
    \III{m'}{n'}{m}{n}{\phi}{\psi}{c}{\pi}{c'} =
    \III{m'}{n'}{m}{n}{\phi_{\pi}}{\psi}{c'}{1_{c'}}{c'} =
    \II{m'}{n'}{m}{n}{\phi_{\pi}}{\psi}{c'}.
  \]

  \noindent Furthermore, if $c \simeq \oplus_{i=1}^{l} c_{i}$, then
  \[
    \II{m'}{n'}{m}{n}{\phi}{\psi}{c} = \II{m'}{n'}{m}{n}{\phi}{\psi}{\oplus_{i=1}^{l} c_{i}} = \sum_{j=1}^{l} \III{m'}{n'}{m}{n}{\phi_{j}}{\psi}{c_{j}}{\iota_{j}}{\oplus_{i=1}^{l} c_{i}} =
    \sum_{j=1}^{l}\II{m'}{n'}{m}{n}{\phi_{j}}{\psi_{j}}{c_{j}},
  \]
  where $\iota_{j}$ denotes the $j$-th embedding map from $c_{j}$ to
  $\oplus_{i=1}^{l}c_{i}$, and the $\phi_{j}, \psi_{j}$'s denote the $j$-th
  projection of $\phi, \psi$ respectively. In particular, when
  $c \simeq x^{\oplus l}$, the this gives a reduction from
  \[
    Hom_{M}(m, m' \rhd x^{\oplus l}) \otimes Hom_{C}(x^{\oplus l}, x^{\oplus l}) \otimes Hom_{N} (x^{\oplus l} \lhd n, n')
  \]
  to
  \[
    Hom_{M}(m, m' \rhd x) \otimes Hom_{C}(x, x) \otimes Hom_{N} (x \lhd n, n').
  \]
  It is helpful to regard the result as the ``inner product'' of $\phi$ and $\psi$.
\end{remark}

\noindent The hom vector spaces are finite dimensional. An explicit basis is
constructed in \ref{proposition/basis-theorem}.

\begin{definition} \label{definition/karoubi-completion} (Karoubi Completion)

  \noindent Let $C$ be a category. \quad The Karoubi completion $Kar(C)$ of
  $C$ is defined to be the category with
  \[
    Obj(Kar(C)) = \{(c, f) \,|\, c \in Obj(C), f \in End_{C}(c), f = f^{2}\}
  \] and
  \[
    Mor_{Kar(C)}((c,f), (c', f')) = \{\overline{f} \in Hom_{C}(c,c') \,|\, \overline{f}f = \overline{f} = f'\overline{f}\},
  \]
  with the obvious composition rule.
\end{definition}

\begin{remark} \label{remark/karoubi-retract} (Karoubi Retracts)

  \noindent It is straightforward to check that every object $(c,f)$ in
  $Kar(C)$ is a retract of the original object $c = (c, 1_{c})$. So we have
  \[
    (c, f) \xrightarrow{\iota} c = (c, 1_{c}) \xrightarrow{\pi} (c, f).
  \]
\end{remark}

\begin{definition}\label{definition/skein-category} (Skein Category)

  \noindent Let $C$ be a tensor category [TODO: (Manuel) or do we just have to require right duals?]. Let $M_{C}$ and $_{C}N$ be module
  categories. \quad Define the skein category $sk(M,C,N)$ to the Karoubi
  completion of its pre-skein category
  \[
    sk(M,C,N) := Kar(p.sk(M,C,N)).
  \]
  \noindent Thus, a typical object of the skein category $sk(M,C,N)$ is an
  idempotent matrix of skeins (the typical object is an idempotent
  endomorphism of a direct sum $\oplus_{i} (m_{i} \boxtimes n_{i})$, and an
  endomorphism of such a direct sum is just a matrix whose entries are maps
  $(m_{i} \boxtimes n_{i}) \to (m_{j} \boxtimes n_{j})$, i.e. skeins like
  $\II{m_{j}}{n_{j}}{m_{i}}{n_{i}}{\phi}{\psi}{c}$). The skein category is
  obviously enriched over $\Vect_{\mathbb{C}}$.

  More generally, for any $n \in \mathbb{N}$, let
  $C_{0}, C_{1}, \ldots, C_{n}$ be tensor categories [TODO: (Manuel) or do we just have to require right duals?]. Let
  ${}_{C_{0}}M^{1}_{C_{1}}, \, {}_{C_{1}}M^{2}_{C_{2}}, \ldots, {}_{C_{n{\text -}1}}M^{n}_{C_{n}}$
  be bimodule categories. One can in a similar way define the
  $C_{0}{\text -}C_{n}$ bimodule category
  \[
    sk(M^{1}, C_{1}, M^{2}, C_{2}, M^{3}, \ldots, C_{n{\text -}1}, M^{n}).
  \]
  \begin{center}
    \includesvg[width=12cm]{drawing-3}
  \end{center}

\end{definition}

\begin{definition} \label{definition/induced-functor-on-skein-category} (Induced Functor on Skein Category)

  \noindent Assume the notation in \ref{definition/skein-category}. Let
  $1 \leq i \leq n$, and let $F: M^{i} \to M'^{i}$ be a
  $C_{i{\text -}1}{\text -}C_{i}$ bimodule category functor. \quad Then $F$
  naturally induces a linear functor between the skein categories
  \[
    sk(M^{1}, C_{1}, M^{2}, C_{2}, M^{3}, \ldots M^{i} \ldots, C_{n{\text -}1}, M^{n})
    \xrightarrow{F}
    sk(M^{1}, C_{1}, M^{2}, C_{2}, M^{3}, \ldots M'^{i} \ldots, C_{n{\text -}1}, M^{n}).
  \]
  For example, if $n=2$, $M^{1} = M$, $M^{2} = N$, $C^{1} = C$, $N \xrightarrow{F} N'$
  (with the left $C$-module structure given by $\alpha$), then we have an
  induced linear functor $sk(M,C,N) \to sk(M,C,N')$ sending the objects and morphisms via the map:
  \[
    \II{m'}{n'}{m}{n}{\phi}{\psi}{c}
    \quad \mapsto \quad
    \II{m'}{F(n')}{m}{F(n)}{\phi}{F(\psi) \circ \alpha}{c}.
  \]
\end{definition}

\noindent The main result of this paper is to show that the skein category
$sk(M,C,N)$ is equivalent to $M \boxtimes_{C} N$, and that the induced functor $M \boxtimes_{C} N \to M \boxtimes_{C} N'$
coincides with the one in \ref{definition/induced-functor-on-skein-category}
(proven in \ref{lemma/main-lemma}, \ref{theorem/main-theorem}). A necessary
ingredient is the canonical map $\boxtimes_{C}$ given in the defining universal
property.

\begin{definition}\label{definition/canonical-map} (Canonical Map $\boxtimes_{C}$)

  \noindent Let $C$ be a tensor category [TODO: (Manuel) or do we just have to require right duals?], and $M_{C}, _{C}N$ be module
  categories. \quad Define the functor
  \[
    M \times N \xrightarrow{\boxtimes_{C}} sk(M,C,N)
  \]
  to send the object $(m,n)$ to the object $\II{m}{n}{m}{n}{1_{m}}{1_{n}}{1_{1}}$, and the morphism
  \[
    (m,n) \xrightarrow{(\phi, \psi)} (m', n')
  \]
  to the morphism $\II{m'}{n'}{m}{n}{\phi}{\psi}{1_{1}}$.
\end{definition}

\noindent From the main theorem, we must have $sk(M,C,C) \simeq M$. To quickly
convince the reader that the main theorem is true before it is proven, we
provide another direct proof for this equivalence in the appendix (cf
\ref{proposition/degenerated-main-theorem}).

\hfill\break
\noindent We prove another lemma that will be useful later.

\begin{lemma}\label{lemma/I-provides-subobject} (Objects are Retracts)

  \noindent Let $C$ be a tensor category [TODO: (Manuel) or do we just have to require right duals?], and $M_{C}, _{C}N$ be module
  categories. \quad Then any typical object $\II{}{}{m}{n}{\phi}{\psi}{c}$ in
  $sk(M,C,N)$ is a retract (in particular, a subobject) of the canonical
  object $\boxtimes_{C}((m,n)) = \II{}{}{m}{n}{1_{m}}{1_{n}}{1}$.
\end{lemma}
\begin{proof}
  This directly follows from \ref{remark/karoubi-retract}.
\end{proof}



\subsection{Proof of the Main Equivalence Theorem}\label{section/proof-of-equivalence}

Unless specified otherwise, throughout this section, let $C, D, E$ be tensor
categories, let $M_{C}$ and $_{C}N$ be module categories, and let $L$ be a
linear category. We prove our main theorem (\ref{theorem/main-theorem}) in this section, justifying
that the skein construction $sk(M,C,N)$ is isomorphic to $M \boxtimes_{C} N$, and the
obvious generalization to the case of more bimodule categories.

\begin{lemma}\label{lemma/construction-of-theta} (Construction of $\Theta$)

  \noindent There exists a linear functor
  \[
    \Theta: Fun(sk(M,C,N), L) \to Fun^{C{\text -}bal}(M \times N, L).
  \]
\end{lemma}

\noindent We construct $\Theta$ explicitly in the proof.

\begin{proof}
  \noindent (Object) Let $G$ be an object of the domain. Define $\Theta(G)$ to
  be $F := G \circ \boxtimes_{C} \in Fun(M \times N, L)$. We shall provide the
  balanced structure $\alpha$ for $F$, so that $(F, \alpha)$ is $C$-balanced. 
  We need to provide the $C$-balanced data for $F$:
  \[
    \alpha_{m,c,n}: F(m \lhd c, n) =
    G(\II{}{}{mc}{n}{1}{1}{1})
    \xrightarrow{\sim} G(\II{}{}{m}{cn}{1}{1}{1})
    = F(m, c \rhd n),
  \]
  which is clearly satisfied by $G(\II{m}{cn}{mc}{n}{1}{1}{c}).$ So defined
  $\alpha$ is clearly natural, and it is invertible by using the right dual of $c$.

  \noindent (Morphism) Let $G \xrightarrow{\eta} G'$ be a morphism in
  $Fun(sk(M,C,N), L)$. Its image under $\Theta$ is simply the horizontal
  composition $\eta \star (1_{\boxtimes_{C}})$. The remaining commutativity to be checked
  % cf p11 of Jin's note 20240906-110000
  is a direct consequence of $\eta$'s naturality.
\end{proof}

\begin{lemma}\label{lemma/theta-is-faithful} ($\Theta$ is faithful)

  \noindent The linear functor $\Theta$ (cf. \ref{lemma/construction-of-theta}) is faithful.
\end{lemma}

\begin{proof}
  (We use the same notation found in this section.) This amounts to showing
  that the following map is an injective linear map:
  \[
    (G \xrightarrow{\eta} G') \mapsto (F \xrightarrow{\Theta(\eta) = \eta \star (1_{\boxtimes_{C}})} F').
  \]
  It is clearly linear. For injectivity, we notice that whenever we have
  linear functors
  \[
    X \xrightarrow{f} Y,\quad Y \xrightarrow{g, g'} Z,
  \]
  and a linear natural transformation $\eta: g \to g'$, then the map $(\eta \mapsto \eta \star 1_{f})$ is injective is equivalent to
  \[
    (\forall x \in Obj(X), \eta_{f(x)} = 0) \Rightarrow (\forall y \in Obj(Y), \eta_{y} = 0).
  \]
  This holds if $f$ is surjective on objects. However, in our case
  $f = \boxtimes_{C}$ is not as strong. Fortunately, clearly it also holds if
  $f$ is almost-surjective, in the sense that each $y \in Obj(Y)$ has an
  $x \in Obj(X)$ such that $y$ is a retract of $f(X)$. Indeed,
  \[
    1_{g(y)} = g(1_{y}) = g(\pi_{y} \circ \iota_{y}),
  \]
  so
  \[
    (g(y) \xrightarrow{\eta_{y}} g'(y)) = \eta_{y} \circ 1_{g(y)} = \eta_{y} \circ g(\pi \circ \iota) = g(\pi) \circ \eta_{f(x)} \circ g(\iota) = g(\pi) \circ 0 \circ g(\iota) = 0.
  \]
  This applies to our case by putting $f = \boxtimes_{C}$ and $g = G$, because
  each object $\II{}{}{m}{n}{\phi}{\psi}{c}$ is clearly a retract of
  $\boxtimes_{C}(m,n) = \II{}{}{m}{n}{1}{1}{1}$. Therefore, $\Theta$ is faithful.
\end{proof}

\begin{lemma}\label{lemma/theta-is-full} ($\Theta$ is full)

  \noindent The linear functor $\Theta$ (cf. \ref{lemma/construction-of-theta}) is full.
\end{lemma}

\begin{proof}
  We need to show that for any $C$-balanced natural transformation
  \[
    \nu: G \circ \boxtimes_{C} = \Theta(G) \to \Theta(G') = G' \circ \boxtimes_{C}
  \]
  there is $\mu: G \to G'$ such that $\nu = \mu \star 1_{\boxtimes_{C}}$. The data $\nu$ are the maps
  \[
    \nu_{(m,n)}: G(\II{}{}{m}{n}{1}{1}{1}) \to G'(\II{}{}{m}{n}{1}{1}{1}).
  \]
  We only need to extend these data to all objects in $sk(M,C,N)$, i.e. define compatible maps
  \[
    \nu_{\II{}{}{m}{n}{\phi}{\psi}{c}}: G(\II{}{}{m}{n}{\phi}{\psi}{c}) \to G'(\II{}{}{m}{n}{\phi}{\psi}{c}).
  \]
  It is straightforward to check that the following works:
  \[
    \nu_{\II{}{}{m}{n}{\phi}{\psi}{c}}:= G(\II{}{}{m}{n}{\phi}{\psi}{c})
    \xrightarrow{G(\iota)}
    G(\II{}{}{m}{n}{1}{1}{1})
    \xrightarrow{\nu_{(m,n)}}
    G'(\II{}{}{m}{n}{1}{1}{1})
    \xrightarrow{G'(\pi)}
    G'(\II{}{}{m}{n}{\phi}{\psi}{c}),
  \]
  where $\iota$ and $\pi$ are the inclusion and projection (see
  Lemma \ref{lemma/I-provides-subobject}).
\end{proof}

\noindent To prove that $\Theta$ is essentially surjective, we need the following
lemma.

\begin{lemma} (Images of skeins) \label{lemma/image-of-skein}
  % Discussion: Can we relax conditions? We need to use the basis theorem,
  % so it seems that we must require semisimplicity here.
  \noindent
  Let $(F,\alpha) \in Fun^{C{\text -}bal}(M \times N, L)$. For each morphism
  in $sk(M,C,N)$ of the form $\III{m'}{n'}{m}{n}{\phi}{\psi}{c}{\pi}{c'}$,
  define the image of it under $\tilde{F}$ to be the composed morphism
  \[
    F(m,n)
    \xrightarrow{F(\phi \times 1)}
    F(m' \lhd c, n)
    \xrightarrow{F((1 \lhd \pi) \times 1)}
    F(m' \lhd c', n)
    \xrightarrow[\sim]{\alpha}
    F(m', c' \rhd n)
    \xrightarrow{F(1 \times \psi)}
    F(m',n').
  \]
  Suppose we have two skeins $\II{m'}{n'}{m}{n}{\phi}{\psi}{c}$ and
  $ \II{m'}{n'}{m}{n}{\phi'}{\psi'}{c'}$ as identical morphisms in $sk(M,C,N)$
  (recall the definitional relations (\ref{relation/a}) (\ref{relation/b})).

  \noindent Then
  \begin{equation} \label{eqn/a}
    \tilde{F}(\II{m'}{n'}{m}{n}{\phi}{\psi}{c}) = \tilde{F}(\II{m'}{n'}{m}{n}{\phi'}{\psi'}{c'}).
  \end{equation}
  Moreover, $\tilde{F}$ preserves compositions, i.e.
  \begin{equation} \label{eqn/b}
    \tilde{F}(\II{m''}{n''}{m'}{n'}{\overline{\phi}}{\overline{\psi}}{\overline{c}} \circ \II{m'}{n'}{m}{n}{\phi}{\psi}{c})
    = \tilde{F}(\II{m''}{n''}{m'}{n'}{\overline{\phi}}{\overline{\psi}}{\overline{c}})
    \circ
    \tilde{F}(\II{m'}{n'}{m}{n}{\phi}{\psi}{c}).
  \end{equation}
\end{lemma}

\begin{proof}
  To prove the first statement (\ref{eqn/a}), check the equality against the
  definitional relations (\ref{relation/a}) (\ref{relation/b}).

  \noindent To prove the second statement (\ref{eqn/b}), note that the
  left-hand-side is
  $\tilde{F}(\II{m''}{n'}{m}{n}{\overline{\phi} \otimes \phi}{\overline{\psi} \otimes \psi}{\overline{c} \otimes c}),$
  which is
  \begin{multline*}
    F(m,n)
    \xrightarrow{F(\phi \times 1)}
    F(m' \lhd c, n)
    \xrightarrow{F((\overline{\phi} \lhd 1) \times 1)} \\
    F((m'' \lhd \overline{c}) \lhd c, n)
    \xrightarrow[\sim]{}
    F((m'' \lhd (\overline{c} \otimes c)), n)
    \xrightarrow[\sim]{\alpha}
    F(m'', (\overline{c} \otimes c) \rhd n)
    \xrightarrow[\sim]{}
    F(m'', \overline{c} \rhd (c \rhd n)) \\
    \xrightarrow{F(1 \times (1 \rhd \psi))}
    F(m'', \overline{c} \rhd n')
    \xrightarrow{F(1 \times \overline{\psi})}
    F(m'',n'').
  \end{multline*}
  On the other hand, the right-hand-side is
  \begin{multline*}
    F(m,n)
    \xrightarrow{F(\phi \times 1)}
    F(m' \lhd c, n)
    \xrightarrow[\sim]{\alpha} \\
    F(m', c \rhd n)
    \xrightarrow{F(1 \times \psi)}
    F(m', n')
    \xrightarrow{F(\overline{\phi} \times 1)}
    F(m'' \rhd \overline{c}, n') \\
    \xrightarrow[\sim]{\alpha}
    F(m'', \overline{c} \lhd n')
    \xrightarrow{F(1 \times \overline{\psi})}
    F(m'',n'').
  \end{multline*}
  To prove that they are equal, we can omit their first and their last arrows. Note that the composed arrow in left-hand-side $F(m' \lhd c, n) \to F(m', \overline{c} \rhd (c \rhd n))$ is, by the naturality of $\alpha$, equal to
  \[
    F(m' \lhd c, n)
    \xrightarrow[\sim]{\alpha}
    F(m', c \rhd n)
    \xrightarrow{F(\overline{\phi} \times 1)}
    F(m'' \lhd \overline{c}, c \rhd n)
    \xrightarrow[\sim]{\alpha}
    F(m'', \overline{c} \rhd (c \rhd n)).
  \]
  Compose this with
  \[
    F(m'', \overline{c} \rhd (c \rhd n))
    \xrightarrow{F(1 \times (1 \rhd \psi))}
    F(m', \overline{c} \rhd n'),
  \]
  then we get
  \[
    F(m' \lhd c, n)
    \xrightarrow[\sim]{\alpha}
    F(m', c \rhd n)
    \xrightarrow{F(\overline{\phi} \times 1)}
    F(m'' \lhd \overline{c}, c \rhd n)
    \xrightarrow{F(1 \times \psi)}
    F(m'' \lhd \overline{c}, n'),
  \]
  which is equal to
  \[
    F(m' \lhd c, n)
    \xrightarrow[\sim]{\alpha}
    F(m', c \rhd n)
    \xrightarrow{F(\overline{\phi} \times \psi)}
    F(m'' \lhd \overline{c}, n').
  \]
  So both sides are equal.
\end{proof}

\begin{lemma}\label{lemma/theta-is-essentially-surjective} ($\Theta$ is essentially surjective)

  \noindent The linear functor $\Theta$ (cf. \ref{lemma/construction-of-theta}) is essentially surjective.
\end{lemma}

\begin{proof}
  Let $(F, \alpha) \in Fun^{C{\text -}bal}(M \times N, L)$. It suffices to construct
  $G \in Fun(sk(M,C,N), L)$ such $\Theta(G) \simeq (F,\alpha)$. Recall that $L$ is an abelian
  category (so each $L$-morphism has an image), $F: M \times N \to L$ is a linear
  functor, and that
  \[
    F(m \lhd c, n) \xrightarrow[\sim]{\alpha_{m,c,n}} F(m, c \rhd n).
  \]

  \noindent ($G$ on objects) Recall the definition and properties of
  $\tilde{F}$ in \ref{lemma/image-of-skein}. Define
  $G(\II{}{}{m}{n}{\phi}{\psi}{c})$ to be the image (in $L$) of the
  $L$-morphism $\tilde{F}(\II{}{}{m}{n}{\phi}{\psi}{c})$. In particular, the
  image is a subobject and a quotient of $F(m,n)$.

  \noindent ($G$ on morphisms) We use $\tilde{F}$ again. Let
  $\II{m'}{n'}{m}{n}{\overline{\phi}}{\overline{\psi}}{\overline{c}}$ be a morphism
  from $\II{}{}{m}{n}{\phi}{\psi}{c}$ to $\II{}{}{m'}{n'}{\phi'}{\psi'}{c'}$. Define
  $G(\II{m'}{n'}{m}{n}{\overline{\phi}}{\overline{\psi}}{\overline{c}})$ to be the
  map induced by
  \[
    \tilde{F}(\II{m'}{n'}{m}{n}{\overline{\phi}}{\overline{\psi}}{\overline{c}}): F(m,n) \to F(m',n').
  \]
  To justify this definition, we must show that
  \[
    ker(
    \tilde{F}(\II{}{}{m'}{n'}{\phi'}{\psi'}{c'})
    \circ
    \tilde{F}(\II{m'}{n'}{m}{n}{\overline{\phi}}{\overline{\psi}}{\overline{c}})
    )
    \supseteq
    ker(\tilde{F}(\II{}{}{m}{n}{\phi}{\psi}{c})).
  \]
  By lemma \ref{lemma/image-of-skein}, $\tilde{F}$ respects compositions, so
  \[
    ker(
    \tilde{F}(\II{}{}{m'}{n'}{\phi'}{\psi'}{c'})
    \circ
    \tilde{F}(\II{m'}{n'}{m}{n}{\overline{\phi}}{\overline{\psi}}{\overline{c}})
    )
    =
    ker(
    \tilde{F}(\II{}{}{m'}{n'}{\phi'}{\psi'}{c'}
    \circ
    \II{m'}{n'}{m}{n}{\overline{\phi}}{\overline{\psi}}{\overline{c}})
    ).
  \]
  Then by the definition of $sk(M,N,C)$ and Karoubi completion,
  \[
    ker(
    \tilde{F}(\II{}{}{m'}{n'}{\phi'}{\psi'}{c'}
    \circ
    \II{m'}{n'}{m}{n}{\overline{\phi}}{\overline{\psi}}{\overline{c}})
    )
    =
    ker(
    \tilde{F}(
    \II{m'}{n'}{m}{n}{\overline{\phi}}{\overline{\psi}}{\overline{c}}
    \circ
    \II{}{}{m}{n}{\phi}{\psi}{c}
    )
    ).
  \]
  The final step is completed by using the composing property of $\tilde{F}$ again and the fact that
  $ker(a \circ b) \supseteq ker(b)$.
\end{proof}

\begin{lemma} (Main Lemma) \label{lemma/main-lemma}

  \noindent The canonical map (\ref{definition/canonical-map})
  $\boxtimes_{C}$: $M \times N \to sk(M,C,N)$ satisfies the universal property
  in the definition of $M \boxtimes_{C} N$. In particular, we have an
  equivalence of categories
  \[
    sk(M,C,N) \simeq M \boxtimes_{C} N.
  \]
\end{lemma}

\begin{proof}
  We only need to show that $sk(M,C,N)$ is a $k$-linear category (in
  particular, an abelian category, following the definition in
  \cite{douglas/balanced-product}), and that $\boxtimes_{C}$ induces an equivalence of
  categories
  \[
    Fun(sk(M,C,N), L) \xrightarrow[\sim]{\Theta} Fun^{C{\text -}bal}(M \times N, L).
  \]
  For $k$-linearity, the crux is to show that the skein category is abelian.
  We postpone the proof to the appendix \ref{semisimple}. For the second
  statement, we constructed $\Theta$ in (\ref{lemma/construction-of-theta}), proved that $\Theta$ is faithfulness in
  (\ref{lemma/theta-is-faithful}), is full in (\ref{lemma/theta-is-full}), and is essentially surjective in (\ref{lemma/theta-is-essentially-surjective}).
\end{proof}


\begin{theorem} (Main Theorem: Skein Construction of Balanced Tensor Product) \label{theorem/main-theorem}

  \noindent (1) Let ${}_{C}M_{D}, \, {}_{D}N_{E}$ be, bimodule categories.
  \quad Then the canonical map (\ref{definition/canonical-map})
  $\boxtimes_{D}$: $M \times N \to sk(M,D,N)$ satisfies the universal property
  in the definition of ${}_{C}M_{D} \boxtimes_{D} {}_{D}N_{E}$. In particular,
  we have an equivalence of $C{\text -}E$ bimodule categories.
  \[
    {}_{C}sk(M,D,N)_{E} \simeq {}_{C}M_{D} \boxtimes_{D} {}_{D}N_{E}.
  \]

  \noindent (2) More generally, for any $n \in \mathbb{N}$, let
  $C_{0}, C_{1}, \ldots, C_{n}$ be tensor categories [TODO: (Manuel) or do we just have to require right duals?]. Let
  ${}_{C_{0}}M^{1}_{C_{1}}, \, {}_{C_{1}}M^{2}_{C_{2}}, \ldots, {}_{C_{n{\text -}1}}M^{n}_{C_{n}}, \, $
  be bimodule categories. \quad Then we have an equivalence of
  $C_{0}{\text -}C_{n}$ bimodule categories.
  \[
    {}_{C_{0}}sk(M^{1},C_{1},M^{2},C_{2}, \ldots, C_{n{\text -}1}, M^{n})_{C_{n}}
    \simeq
    {}_{C_{0}}(M^{1}
    \boxtimes_{C_{1}}
    M^{2}
    \boxtimes_{C_{2}}
    M^{3}
    \ldots
    \boxtimes_{C_{n{\text -}1}}
    M^{n})_{C_{n}}.
  \]
  \noindent (3) Moreover, in addition to the previous part, if
  $F^{i}: M^{i} \to M'^{i}$ is a $C_{i{\text -}1}{\text -}C_{i}$ bimodule category
  functor, then the naturally induced linear functor (cf.
  \ref{definition/induced-functor-on-skein-category})
  \[
    F^{i}:
    sk(M^{1}, C_{1}, M^{2}, C_{2}, M^{3}, \ldots M^{i} \ldots, C_{n{\text -}1}, M^{n})
    \to
    sk(M^{1}, C_{1}, M^{2}, C_{2}, M^{3}, \ldots M'^{i} \ldots, C_{n{\text -}1}, M^{n}),
  \]
  corresponds to the functor
  \[
    F^{i}: M^{1} \boxtimes_{C_{1}} M^{2} \boxtimes_{C_{2}} M^{3} \boxtimes_{C_{3}} \ldots M^{i} \ldots \boxtimes_{C_{n{\text -}1}} M^{n} \to M^{1} \boxtimes_{C_{1}} M^{2} \boxtimes_{C_{2}} M^{3} \boxtimes_{C_{3}} \ldots M'^{i} \ldots \boxtimes_{C_{n{\text -}1}} M^{n}.
  \]
  under the equivalence.
\end{theorem}

\begin{proof}
  The first part is proved by restricting the proof of lemma \ref{lemma/main-lemma} to $C{\text -}E$ bimodule maps.
  The second part follows directly from induction and the first part. The third part is obvious.
\end{proof}

\begin{remark} (Application on the Turaev-Viro model)

  \noindent That the induced functor on the skein category coincides with the
  algebraic one is the key for computing values of the Turaev-Viro model in
  dimensions $(1+1)$ \cite{guu/tv-as-3-functor}.
\end{remark}

% The following corollary is commented out. The statement itself is not
% incorrect, but the proof is wrong (circular). We must have proven that they
% are abelian before showing that skein categories are actually balanced
% tensor products.
% \begin{corollary}\label{corollary/skein-category-is-abelian} (Skein Categories are Abelian)
%   \noindent For any $n \in \mathbb{N}$, let $C_{0}, C_{1}, \ldots, C_{n}$ be
%   tensor categories [TODO: (Manuel) or do we just have to require right duals?]. Let
%   \[
%     {}_{C_{0}}M^{1}_{C_{1}}, \, {}_{C_{1}}M^{2}_{C_{2}}, \ldots, {}_{C_{n{\text -}1}}M^{n}_{C_{n}}, \,
%   \]
%   be bimodule categories. Then the skein category (cf
%   \ref{definition/skein-category})
%   \[
%     sk(M^{1}, C_{1}, M^{2}, C_{2}, M^{3}, \ldots M^{i} \ldots, C_{n{\text -}1}, M^{n})
%   \]
%   is an abelian category.
% \end{corollary}
% \begin{proof}
%   The proof for the case $n=2$ follows from the fact that
%   $M^{1} \boxtimes_{C_{1}} M^{2} \simeq Z_{C_{1}}(M^{1} \boxtimes M^{2})$ is
%   abelian (cf \cite{kirillov/fact-homo-4d-tqft}). The rest follows from
%   induction.
% \end{proof}


\subsection{Misc. Results}
\noindent This subsection gathers results, proofs, and arguments that would
disrupt the flow of the main text. The content is unstructured, so readers
should approach it as a collection of standalone items.

\noindent From the main theorem, we must have $sk(M,C,C) \simeq M$. To quickly
convince the reader that the main theorem is true before it is proven, we
provide another direct proof for this equivalence:

\begin{proposition} \label{proposition/degenerated-main-theorem}

  \noindent Let $C$ be a tensor category [TODO: (Manuel) or do we just have to require right duals?]. Let $M_{C}$ be a right $C$-module
  category. \quad Then we have an equivalence of categories
  $sk(M,C,C) \simeq M$.
\end{proposition}

\begin{proof}
  We provide two proofs. The first proof is to use the main theorem
  \ref{theorem/main-theorem}, and the known fact that
  $M \boxtimes_{C} C \simeq M$. The second proof is direct, without using the
  main theorem:

  Construct the functor
  $M \xrightarrow{1_{M} \times 1} M \times C \xrightarrow{\boxtimes_{C}} sk(M,C,C)$,
  where $\boxtimes_{C}$ is the canonical map (\ref{definition/canonical-map}).
  We contend that this is an equivalence of categories. It is straightforward
  to see that it is indeed fully faithful, so it suffices to show that it is
  essentially surjective. A typical object in the codomain $sk(M,C,C)$ is some
  idempotent skein $\II{}{}{m}{c}{\phi}{\psi}{d}$ (without loss of generality,
  assume $m$ to be simple). We contend that this object is isomorphic to
  $\II{}{}{m \lhd c}{1}{\mu_{\phi, \psi}}{1}{1}$, where
  \[
    \mu_{\phi,\psi} :=
    m \lhd c
    \xrightarrow{\phi \lhd c}
    (m \lhd d) \lhd c
    \xrightarrow[\sim]{\alpha}
    m \lhd (d \otimes c)
    \xrightarrow{m \lhd \psi}
    m \lhd c.
  \]
  Indeed, the isomorphism is provided by the following two morphisms
  \begin{align*}
    \II{}{}{m \lhd c}{1}{\mu_{\phi, \psi}}{1}{1}
    \xleftarrow{
    \II{}{}{m \lhd c}{1}{\mu_{\phi, \psi}}{1}{1}
    \, \circ \,
    \II{mc}{1}{m}{c}{u}{n}{c^{\star}}
    \, \circ \,
    \II{}{}{m}{c}{\phi}{\psi}{\overline{c}}}
    \II{}{}{m}{c}{\phi}{\psi}{\overline{c}}
    \\
    \II{}{}{m}{c}{\phi}{\psi}{\overline{c}}
    \xleftarrow{
    \II{}{}{m}{c}{\phi}{\psi}{\overline{c}}
    \, \circ \,
    \II{m}{c}{mc}{1}{1}{1}{c}
    \, \circ \,
    \II{}{}{m \lhd c}{1}{\mu_{\phi, \psi}}{1}{1}
    }
    \II{}{}{m \lhd c}{1}{\mu_{\phi, \psi}}{1}{1},
  \end{align*}
  where $c^{\star}$ denotes the right dual of $c$, $n$ denotes the counit, and
  $u$ denotes the unit for $c$. The two given morphisms may seem unnecessarily
  long, but they have to be so by the definition of of Karoubi completion (in
  which objects must be ``absorbed'' into morphism). The following pictures
  show that they compose to the identities.

  \begin{center}
    \includesvg[width=18cm]{drawing-4}
  \end{center}
\end{proof}

\noindent For computation purpose (cf. one of the applications in the introduction), one
may find the following basis theorem useful.

\begin{proposition} (Basis Theorem) \label{proposition/basis-theorem}
  \noindent Let $C$ be a finite, semisimple tensor category, and
  $M_{C}, _{C}N$ be finite, semisimple module categories over $C$. \quad Thus
  the vector spaces $Hom_{M}(m, m' \lhd c), Hom_{N}(c \rhd n, n')$ have finite bases
  $\beta(m, m' \lhd c), \beta(c \rhd n, n')$ respectively. \quad Then the hom space
  $Hom_{p.sk(M,C,N)}(m \boxtimes n, m' \boxtimes n')$ has a linear basis
  \[
    \bigsqcup_{c \in Irr(C)} \beta(m, m' \lhd c) \times \beta(c \rhd n, n').
  \]
\end{proposition}
\begin{proof}
  This follows immediately from the reductions given in \ref{remark/hom-space-reduction}.
\end{proof}
