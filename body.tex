\section{Introduction}

The notion of a balanced tensor product was first given in
\cite{etingof/fusion-cat-and-homotopy}, and mainly developed in
\cite{douglas/balanced-product} and \cite{douglas/dualizable-tensor-categories}.
It has different interpretations in various fields:
\begin{itemize}
  \item In the theory of topological phases, this construction corresponds to
        fusing two domains (labeled by module categories) along the domain
        wall (labeled by tensor categories) \cite{kong/topological-order}.
  \item In the theory of extended topological quantum field theory, it is the
        $1$-categorical structure of a common target $3$-category $TC$
        \cite{douglas/dualizable-tensor-categories}.
  \item In the theory of vertex operator algebras (VOA), this construction helps
        classify the full bulk CFTs (as explained in
        \cite{gannon/exotic-quantum-subgroup}), and constructing new VOAs
        \cite{gannon/sln-II}.
  \item In the theory of tensor categories itself, it closely relates to the
        categorical center construction. In particular, the Drinfeld center
        $Z(C)$ of $C$ is equivalent to $C \boxtimes_{C \boxtimes C^{op}} C$
        \cite{kirillov/string-net-tv}
        \cite{douglas/dualizable-tensor-categories}.
\end{itemize}

\noindent Before this work, several algebraic constructions for balanced tensor products
were known, including categories of modules \cite{douglas/balanced-product},
internal Hom spaces \cite{davydov/picard}, and generalized categorical centers
\cite{etingof/fusion-cat-and-homotopy} \cite{kirillov/fact-homo-4d-tqft}
\cite{hoek/master}. These approaches well-captured the algebraic aspects of
balanced tensor products.

In this paper, we provide a topological construction based on skein theory,
which strikes a better mix between algebra and topology. It has a few advantages:

\begin{enumerate}
  \item Its topological nature allows us to use skein diagrams to calculate the
        values of the fully extended $3$D field theories associated to a
        fusion category by the cobordism hypothesis \cite{lurie/tqft}
        \cite{douglas/dualizable-tensor-categories}
        \cite{guu/tv-as-3-functor}.
  \item It generalizes seamlessly to balanced tensor products involving more than
        two module categories: $M_1 \boxtimes_{C_1} M_2 \boxtimes_{C_2} M_3 \ldots$.
        (see the picture in \ref{iterated_remark}).
      \end{enumerate}

\noindent Some of the (potential) applications of this work are as follows.

\begin{enumerate}
  \item \textbf{(Skein Nature of eTFTs)} In an upcoming work
        \cite{guu/tv-as-3-functor}, we build on the main result presented here
        to prove the long-anticipated theorem that the Turaev-Viro state sum
        model \cite{viro/turaev-viro-model} naturally and necessarily emerges
        from the 3-functor in the classification of fully extended field
        theories \cite{lurie/tqft} \cite{douglas/dualizable-tensor-categories}.
  \item \textbf{(No-Go: Detection of Exotic Smooth Structures)} We anticipate
        that this approach can be extended to a $4$-dimensional analogue of
        Turaev-Viro theory, the Crane-Yetter model. Additionally, a similar
        strategy should apply to a broader generalization based on fusion
        2-categories, as explored in \cite{douglas/fusion-2-cat-4d-tqft}.
        Proving this would lend support to the conjecture proposed there,
        which suggests that despite the construction's generality and power,
        it remains an extended topological field theory and therefore cannot
        detect exotic structures \cite{reutter/no-go-exotic}.
  \item \textbf{(Factorization Homology)} The main result of this paper
        provides a simple reproof of the result in
        \cite{kirillov/fact-homo-4d-tqft} that shows equivalence between the
        Crane-Yetter model and factorization homology
        \cite{ayala/factorization-homology} in $4$D (the excision property).
  \item \textbf{(Defected TFTs)} We expect work being useful in the study of
        defect field theories, such as in \cite{meusburger/defect-tv}.
  \item \textbf{(Computations in Tensor / Higher Categories)} Given that the
        Drinfeld center is a specific instance of the balanced tensor product,
        we expect this paper will contribute to recent advancements,
        particularly the explicit computation of centers as explored in
        \cite{maurer/computing-center}, and further develop the computational
        aspects of balanced tensor products.
\end{enumerate}

%% % NOTE Add this section back, or remove it.
%% % \subsection{Results}

% Fix a field $k$, and assume all categories, functors and natural
% transformations are enriched over $\Vect_k$ [TODO: Figure out a better way
% to say this sentence]. Let $C$ be a rigid monoidal category, and let $M_C$
% and $_{C}N$ be $C$-module categories. We define the pre-skein category
% $M\otimes_C N$ whose objects are pairs $(m,n)$ with $m\in M$ and $n\in N$
% and whose morphisms $(m,n)\to (m',n')$ are certain skeins (see
% Definition \ref{pre-skein}). Then we have a $C$-balanced functor $M\otimes
% N\to M\otimes_C N$.

% When $M$, $N$ and their balanced tensor product $M\boxtimes_C N$ are all $k$-linear finite semisimple, we define the skein category $sk(M,C,N)$ as the Cauchy completion of $M\otimes_C N$, which can be constructed explicitly in two steps:
% \begin{enumerate}
%   \item add finite direct sums
%   \item add splittings of idempotents
% \end{enumerate}
% We then obtain a $C$-balanced functor $M\otimes N\to sk(M,C,N)$. The main result in our paper is the following.

% \begin{theorem}\label{sk_bal2}
%   Let $C$ be a tensor category [TODO: (Manuel) or do we just have to require right duals?] and let $M_C$,$_{C}N$ be $k$-linear finite semisimple $C$-module categories. Suppose $M\boxtimes_C N$ is also $k$-linear finite semisimple. Then the
%   $C$-balanced functor $M\otimes N\to sk(M,C,N)$ exhibits $sk(M,C,N)$ as the
%   balanced tensor product $M\boxtimes_C N$.
% \end{theorem}

% \noindent Using results from \cite{douglas/dualizable-tensor-categories} (see Remark \ref{semisimple_douglas/dualizable-tensor-categories}), this has the following immediate Corollary.

% \begin{corollary}
%    Suppose that $k$ has characteristic $0$, $C$ is a finite semisimple tensor category [TODO: (Manuel) or do we just have to require right duals?], and $M_C$,$_{C}N$ are $k$-linear finite semisimple $C$-module categories. Then the $C$-balanced functor $M\otimes N\to sk(M,C,N)$ exhibits
%    $sk(M,C,N)$ as the balanced tensor product $M\boxtimes_C N$.
% \end{corollary}

% [TODO (START/001): Remove this section if we turn out not to solve the
% non-semisimple case.] We then discuss the nonsemisimple case. Here one must
% define $sk(M,C,N)$ by taking a finite cocompletion of $M\otimes_C N$ which
% preserves finite colimits in $M$ and $N$ (see
% Definition \ref{sk_nonsemisimple}). We then obtain the following results.

% \begin{theorem} Let $C$ be a rigid monoidal category and let $M_C$, $_{C}N$
% be finitely cocomplete module categories. Then the $C$-balanced functor
% $M\otimes N\to sk(M,C,N)$ presents $sk(M,C,N)$ as the balanced Kelly tensor
% product of $M$ and $N$ over $C$.
% \end{theorem}

% \begin{corollary}

% Let $C$ be a finite tensor category [TODO: (Manuel) or do we just have to
% require right duals?], $M_C$, $_{C}N$ finite $k$-linear $C$-module
% categories. Then $sk(M,C,N)$ is $k$-linear and the $C$-balanced functor
% $M\otimes N\to sk(M,C,N)$ presents $sk(M,C,N)$ as the balanced tensor
% product of $M$ and $N$ over $C$. \end{corollary}

% [TODO (END/001): Remove this section if we turn out not to solve the
% non-semisimple case.]


\subsection{Outline}\label{subsection/outline}

\noindent Given a finite semisimple tensor category $C$, and finite semisimple
module categories $M_C$ and $_CN$ (over $C$), we aim to provide a mathematical
construction of their balanced tensor product $M \boxtimes_C N$ based on skein
theory. We start by defining the preskein category $M \otimes_C N$
(\ref{pre-skein}), which satisfies a similar defining universal property
(\ref{univ_bal}), but fails to be $k$-linear. We then (Cauchy) complete it to
the skein category $sk(M,C,N)$ (\ref{definition/balanced-tensor-product}), and
prove that it is indeed equivalent to $M \boxtimes_C N$ when $k$ has characteristic $0$ (Main Theorem,
\ref{main-theorem-ver-2}). We also obtain analogous results in prime characteristic (see \ref{sk_bal} and \ref{semisimple_douglas/dualizable-tensor-categories}). This generalizes immediately to the balanced tensor
product of multiple module categories: $M_1 \boxtimes_{C_1} M_2 \boxtimes_{C_2}
M_3 \ldots$ (\ref{iterated}). We then drop the semisimplicity
assumption, and prove a similar, yet more general, statement
(\ref{main_nonsemisimple}), where the balanced tensor product is generalized
to Kelly balanced tensor product.

\subsection{Preliminaries}\label{subsection/preliminaries}

Throughout the whole paper, unless mentioned otherwise, we fix a field $k$ and
denote by $\Vect_k$ the category of $k$-vector spaces. We assume all
categories, functors and natural transformations are enriched over $\Vect_k$.

\begin{definition} \label{definition/linearized-product} (Linearized Product, $M \otimes N$)

\noindent
Given categories $M,N$ we denote by $M\otimes N$ the category whose objects
are pairs $(m,n)$ with $m\in M$ and $n\in N$ and where $Hom_{M\otimes
N}((m,n),(m',n'))=Hom_M(m,m')\otimes Hom(n,n')$. Heuristically, $M \otimes N$
is $M \times N$ with bi-linearlized hom spaces.
\end{definition}

\begin{remark} (Categorical Products: $M \otimes N$, $M \otimes_C N$, $M \hat{\otimes}_C N$, $M \boxtimes N$, $M \boxtimes_C N$)

  \noindent For convenience, we collect the six kinds of categorical products
  we use throughout the paper.
  \begin{enumerate}
    \item Linearized product $M \otimes N$ (\ref{definition/linearized-product}).
    \item Pre-skein category $M\otimes_C N$ (\ref{pre-skein}).
    \item Deligne tensor product $M \boxtimes N$ (\ref{remark/deligne-tensor-product}).
    \item Balanced tensor product $M \boxtimes_C N$ (\ref{definition/balanced-tensor-product}).
    \item General Pre-skein category $M \hat{\otimes}_C N$ (\ref{pre-skein_no-duals}).
    \item Kelly Balanced tensor product $M \boxtimes_C^K N$ (\ref{definition/kelly-balanced-tensor-product}).
  \end{enumerate}
  \vspace{-30pt}
\end{remark}

\begin{definition} ((Bi)linear functor, $Fun(M,N)$, $Lin(M,N)$, $Bilin(M \otimes N, L)$)

  \noindent Given categories $M, N, L$, denote by $Fun(M,N)$ the category of
  ($\Vect_k$-enriched) functors from $M$ to $N$. A functor $F \in Fun(M,N)$ is
  linear if it is right exact (i.e. if it preserves all finite colimits). We
  denote the category of linear functors $M \to N$ by $Lin(M,N)$. A functor
  $F \in Fun(M \otimes N, L)$ is called bilinear if it is linear in each
  variable. We denote the category of bilinear functors $M\otimes N \to L$ by
  $Bilin(M\otimes N, L)$.
\end{definition}

\begin{definition} \label{definition/k-linear-category} ($k$-linear category) \cite{douglas/balanced-product}

\noindent A $k$-linear category is an abelian category with a compatible $\Vect_{k}$-enrichment.
\end{definition}

\begin{assumption}
When $C$ is a $k$-linear monoidal category (e.g. a tensor category), any
$C$-module category $M$ is assumed to be $k$-linear and the
structure map $C \otimes M\to M$ is assumed to be bilinear.

Given a right (left, resp.) $C$-module category $M_C$ ($_{C}N$, resp.), we
denote by $m\lhd c$ ($n \rhd n$, resp.) the object in $M$ ($N$, resp.) that
results from acting with $c\in C$ on $m\in M$.
\end{assumption}

\begin{definition} \cite{egno/tensor-cats}
  An object in a $k$-linear category is simple if it has no nontrivial
  subobjects. A $k$-linear category is semisimple if every object splits as a
  direct sum of simple objects. A $k$-linear category $M$ is finite if

  \begin{itemize}
    \item $Hom_M(X,Y)$ is finite dimensional, for all objects $X,Y\in M$.
    \item Every object of $M$ has finite length.
    \item $M$ has enough projectives.
    \item $M$ has finitely many isomorphism classes of simple objects.
  \end{itemize}
  A $k$-linear monoidal category is a $k$-linear category $C$ with a monoidal
  structure where the functor $C\otimes C\to C$ is bilinear. A tensor category
  is a rigid $k$-linear monoidal category.
\end{definition}

\noindent In a finite semisimple $k$-linear category, every object splits as a
finite direct sum of simple objects. When $M,N$ are semisimple, every
$\Vect_{k}$-enriched functor $M\to L$ is automatically exact (thus linear) and
every $\Vect_{k}$-enriched functor $M\otimes N\to L$ is automatically
bilinear.

%% % NOTE Add this section back, or remove it.
%% % [TODO: We may want to uncomment this subsection.]
% \subsection{A Note to Experts}\label{subsection/a-note-to-experts}

% The main theorem provides an algebraic model for balanced tensor products with
% mathematical rigor, utilizing the skein construction. We do not claim to
% introduce novel ideas here: This approach closely mirrors the domain-wall
% perspective in topological phase theory, as outlined in
% \cite{kong/topological-order}. In fact, the essential techniques have been
% used in many places to study the Drinfeld center
% $Z(C) \simeq C \boxtimes_{CC} C$ (e.g. \cite{kirillov/string-net-tv}). This
% work is merely a generalization of it to the context of field theories with
% defects.

% In this work, we assume ideal conditions—finiteness, semisimplicity, and
% rigidity (see \ref{subsection/preliminaries}). While it is both possible and
% beneficial to relax some of these assumptions, for simplicity, we leave such
% generalizations for future work.


\subsection{Balanced Tensor Product}\label{subsection/balanced-tensor-product}

\noindent In this section we recall the definition of the balanced tensor product
from \cite{douglas/balanced-product}.

\begin{definition} (Balanced Functor)

  \noindent Let $C$ be a monoidal category, $M_{C}$, $_{C}N$ $C$-module
  categories, and $L$ a category. \quad Then a $C$-balanced functor $M \otimes
  N\to L$ is a pair

  \[(M \otimes N \xrightarrow{F} L, \alpha)\]
  where $F$ is a ($\Vect_k$-enriched) functor and $\alpha$ is a natural isomorphism
  \[
    \begin{tikzcd}
      M \otimes C \otimes N \arrow[r] \arrow[dr] &
      M \otimes N \arrow[d, Rightarrow, "\alpha"] \arrow[r, "F"] &
      L \\
      & M \otimes N \arrow[ur, "F"'] \\
    \end{tikzcd}
  \]
  where the object $(m,c,n)$ in $M \otimes C \otimes N$ is sent by the first
  upper arrow to $(m \lhd c, n)$, and by the first lower arrow to $(m, c \rhd n)$.
\end{definition}

\begin{definition} (Balanced Natural Transformation)

  \noindent Let $C$ be a monoidal category, let $M_{C}$, $_{C}N$ be $C$-module
  categories and let $L$ be a category. Let $(F,\alpha), (G,\beta)$ be
  $C$-balanced functors from $M \otimes N$ to $L$. A $C$-balanced natural
  transformation from $(F,\alpha)$ to $(G,\beta)$ is a natural transformation
  $ F \xrightarrow{\eta} G$ such that the following diagram commutes.

  \[
    \begin{tikzcd}
      F(m \lhd c, n) \arrow[r, "\alpha_{m,c,n}"] \arrow[d, "\eta_{m \lhd c, n}"'] &
      F(m, c \rhd n) \arrow[d, "\eta_{m, c \rhd n}"] \\
      G(m \lhd c, n) \arrow[r, "\beta_{m,c,n}"'] &
      G(m, c \rhd n)
    \end{tikzcd}
  \]
\end{definition}

\begin{definition} ($Fun^{C-bal}(M \otimes N, L)$ and $Bilin^{C-bal}(M \otimes N, L)$)

  \noindent Let $C$ be a monoidal category, let $M_{C}$, $_{C}N$ be $C$-module
  categories and let $L$ be a category. \quad Then the category of
  $C$-balanced functors $Fun^{C{\text -}bal}(M \otimes N, L)$ is defined to be
  the category whose objects are $C$-balanced functors $M \otimes N \to L$ and
  whose morphisms are $C$-balanced natural transformations.

  We denote by $Bilin^{C{\text -}bal}(M \otimes N, L)$ the full subcategory of
  $Fun^{C{\text -}bal}(M \otimes N, L)$ whose objects are bilinear (thus right
  exact) $C$-balanced functors.
\end{definition}

\begin{definition}\label{definition/balanced-tensor-product} (Balanced Tensor Product)

  \noindent Let $C$ be a $k$-linear monoidal category, and let $M_C$, $_{C}N$
  be $k$-linear module categories over $C$. \quad Then the balanced tensor
  product $M \boxtimes_{C} N$ is a $k$-linear category together with a
  $C$-balanced bilinear functor $M\otimes N\to M\boxtimes_{C} N$ such
  that \[Lin(M \boxtimes_{C} N, L) \to Bilin^{C-bal}(M \otimes N, L)\] is an
  equivalence, for any $k$-linear category $L$.
\end{definition}

\noindent In \cite{douglas/balanced-product} the authors construct $M\boxtimes_C N$ for
any finite tensor category $C$ and any finite $C$-module categories $M, N$,
thus establishing existence in this setting. They do this by using the fact
that the module categories $M,N$ can be realized as categories of modules over
algebra objects $A_M$ and $A_N$ in $C$ and then showing that the category of
$A_M-A_N$-bimodule objects in $C$ is $k$-linear and satisfies the required
universal property. In the present paper, we will give an alternative
construction of $M\boxtimes_C N$ when $M$, $N$, and $M \boxtimes_{C} N$ are
semisimple.

\begin{remark} \label{remark/deligne-tensor-product}
  Notice that the Deligne tensor product $M \boxtimes N$ is just
  $M \boxtimes_{\Vect_k} N$.
\end{remark}

\section{Skein Construction}\label{section/skein-construction}

\subsection{Pre-skein Category $M\otimes_C N$}

\noindent In this section, we present the pre-skein category $M \otimes_{C} N$
and develop some of its properties.

\begin{definition}\label{pre-skein} (Pre-Skein Category)
  For $C$ a monoidal category having right duals
  and $M_C$, $_{C}N$ module categories, we define
  the pre-skein category $M\otimes_C N$ as follows. Its objects are pairs
  $(m,n)$ with $m\in M$ and $n\in N$, which we denote by $m\boxtimes n$. The
  hom space $Hom_{M\otimes_C N}(m\boxtimes n, m'\boxtimes n')$, as a
  $k$-vector space, is the quotient of the $k$-vector
  space $$V((m,n),(m',n')):=\bigoplus_{c,\overline{c} \in Obj(C)} Hom_{M}(m,
  m' \lhd c) \otimes Hom_{C}(c,\overline{c}) \otimes Hom_{N}
  (\overline{c} \rhd n, n')$$ by the $k$-linear subspace spanned by
  \begin{align}
    \phi \otimes (\overline{\pi} \circ \pi) \otimes \psi &- (\phi_{\pi} \otimes \overline{\pi} \otimes \psi) \label{relation/a} \\
    \phi \otimes (\overline{\pi} \circ \pi) \otimes \psi &- (\phi \otimes \pi \otimes {}_{\overline{\pi}}\psi) \label{relation/b},
  \end{align}
  where
  \begin{align}
    \phi_{\pi}  &:= \left( m \xrightarrow{\phi} m' \lhd c \xrightarrow{1_{m'} \lhd \pi} m' \lhd c' \right) \\
    {}_{\overline{\pi}}\psi &:= \left( \overline{c} \rhd n \xrightarrow{\overline{\pi} \rhd 1_{n}} \overline{\overline{c}} \rhd n \xrightarrow{\psi} n' \right)
  \end{align}
  
  \begin{center}
    \includesvg[width=15cm]{drawing-1}

    {\tiny (The $1$-morphisms are drawn from right to left, and
      the tensor products are drawn from top to bottom through this paper.)}
  \end{center}

  \noindent The composition $(\phi' \otimes \pi' \otimes \psi' ) \circ
  (\phi \otimes \pi \otimes \psi)$ is defined to be
  $(\phi'' \otimes \pi'' \otimes \psi'')$ where

  \begin{itemize}
    \item
    $\pi'' \in Hom_{C}(c' \otimes c, \overline{c}' \otimes \overline{c})$ is equal to
    \[
      c' \otimes c \xrightarrow{\pi' \otimes \pi} \overline{c'} \otimes \overline{c},
    \]
    \item
    \noindent $\phi'' \in Hom_{M}(m, m'' \lhd (c' \otimes c))$ is equal to
    \[
      m \xrightarrow{\phi} m' \lhd c \xrightarrow{\phi' \lhd 1_{c}} (m'' \lhd c') \lhd c \xrightarrow[\sim]{\alpha} m'' \lhd (c' \otimes c),
    \]
    \item
    \noindent $\psi'' \in Hom_{N}((\overline{c}' \otimes \overline{c}) \rhd n, n'')$ is equal to
    \[
      (\overline{c}' \otimes \overline{c}) \rhd n \xrightarrow[\sim]{\alpha} \overline{c}' \rhd (\overline{c} \rhd n) \xrightarrow{1_{\overline{c}'} \rhd \psi} \overline{c}' \rhd n' \xrightarrow{\psi'} n''.
    \]
  \end{itemize}
  
    \begin{center}
    \includesvg[width=18cm]{drawing-2}
  \end{center}

  \noindent It is straightforward to check that this composition law respects the
  relations defining $Hom_{M\otimes_C N}$.
\end{definition}

\begin{remark}
 
  \noindent We abuse notation by writing $(\phi\otimes\pi\otimes\psi)$ to
  denote the corresponding equivalence class in $Hom_{M\otimes_C
    N}((m,n),(m',n'))$. We refer to a morphism $(\phi\otimes\pi\otimes\psi)$
  as a skein.
\end{remark}

\begin{remark}

  \noindent In definition \ref{pre-skein}, we require $C$ to have right duals.
  The construction makes sense even without this assumption. To clarify the
  role played by the duals, we define the general preskein category cor $C$
  not necessarily having duals at all in \ref{pre-skein_no-duals} and
  \ref{pre-skein_no-duals_2}.
\end{remark}

\begin{remark}\label{absorb}
  
  Using the defining relations, we see
  that
  \[
    (\phi\otimes\pi\otimes\psi) =
    (\phi_{\pi}\otimes\id_{\bar{c}}\otimes\psi) =
    (\phi\otimes\id_c\otimes{_{\pi}\psi}).
  \]
\end{remark}

\begin{lemma}\label{decompose}
  Any morphism in $M\otimes_C N$ can be written as a sum of composites of
  morphisms of the form $\phi\otimes\id_{1}\otimes\psi:(m,n)\to(m',n')$ and
  $\id_{m\lhd c}\otimes\id_c\otimes\id_{c\rhd n}:(m\lhd c,n)\to(m,c\rhd n).$
\end{lemma}

\begin{proof}
  First, any morphism is a sum of morphisms of the form
  $\phi\otimes\pi\otimes\psi$. By remark \ref{absorb}, we can take
  $\pi=\id_c$. Now
  $(\phi\otimes\id_c\otimes\psi)
  =
  (\id_{m'}\otimes\id_1\otimes\psi)
  \circ
  (\id_{m'\lhd c}\otimes\id_c\otimes\id_{c\rhd n})
  \circ
  (\phi\otimes\id_1\otimes\id_n)$.
\end{proof}


\begin{definition}\label{definition/preskeinification}.
  We define a functor $M\otimes N\to M\otimes_C N$ by the identity on objects,
  and $\phi\otimes\psi\mapsto\phi\otimes\id_1\otimes\psi$ on morphisms. We
  define also morphisms $\beta_{m,c,n}:(m\lhd c,n)\to (m, c \rhd n)$ by
  $\beta_{m,c,n}=\id_{m\lhd c}\otimes\id_c\otimes\id_{c\lhd n}.$

  \begin{center}\includesvg{graphics/beta}\end{center}

\end{definition}

\begin{lemma}\label{beta_natural}
  The collection of morphisms $\beta_{m,c,n}$ defines a natural transformation
  \[
    \begin{tikzcd}
      {M\otimes C\otimes N} & {M\otimes_C N}
      \arrow[""{name=0, anchor=center, inner sep=0}, "{\lhd\otimes\id_N}", curve={height=-12pt}, from=1-1, to=1-2]
      \arrow[""{name=1, anchor=center, inner sep=0}, "{\id_M\otimes\rhd}"', curve={height=12pt}, from=1-1, to=1-2]
      \arrow["\beta", shorten <=3pt, shorten >=3pt, Rightarrow, from=0, to=1]
    \end{tikzcd}.
    \]
\end{lemma}

\begin{proof}
  Given $\phi:m\to m'$, $\psi: n\to n'$ and $\pi:c\to \bar{c}$ we must check
  that the following diagram commutes.
  \[
    \begin{tikzcd}
      {(m\lhd c, n)} & {(m,c\rhd n)} \\
      {(m'\lhd \bar{c}, n')} & {(m',\bar{c}\rhd n')}
      \arrow["{\beta_{m,c,n}}", from=1-1, to=1-2]
      \arrow["{(\phi\lhd\pi)\otimes\id_1\otimes\psi}"', from=1-1, to=2-1]
      \arrow["{\phi\otimes\id_1\otimes(\pi\rhd\psi)}", from=1-2, to=2-2]
      \arrow["{\beta_{m',\bar{c},n'}}"', from=2-1, to=2-2]
    \end{tikzcd}
  \]
  Using the definition of composition and the relations in the definition of
  $Hom_{M\otimes_C N}$ one can check that both composites equal
  $(\phi\lhd\id_c)\otimes\pi\otimes(\id_{\bar{c}}\rhd\psi)$.
\end{proof}

\begin{lemma}\label{beta_invertible}
  If every object in $C$ has a right dual, then the natural transformation
  $\beta$ described above is an isomorphism.
\end{lemma}

\begin{proof}
  Given $(m,c,n)$ we provide an explicit inverse for $\beta_{m,c,n}$. Let
  $(c^*,\eta,\epsilon)$ denote a right dual for $c$. We define
  $\beta^{-1}_{m,c,n}=(\id_m\lhd\eta)\otimes\id_{c^*}\otimes
  (\epsilon\rhd\id_n)$. The following is a proof that
  $\beta_{m,c,n}\circ\beta_{m,c,n}^{-1}=\id_{m,c\rhd n}$.
  \begin{center}\includesvg{graphics/calculation1}\end{center}

  \noindent The proof that $\beta_{m,c,n}^{-1}\circ\beta_{m,c,n}=\id_{m\lhd c,n}$ is
  similar, using the same snake equation.
\end{proof}

\begin{proposition}\label{is_balanced}

\noindent Suppose $C$ is a monoidal category with right duals and $M_C$, $_{C}N$ are
$C$-module categories. Then
\ref{definition/preskeinification} provides a $C$-balanced
functor $M\otimes N\to M\otimes_C N$.
\end{proposition}

\begin{proof}
  This is the content of lemmas \ref{beta_natural} and \ref{beta_invertible}.
\end{proof}

\noindent Therefore, it is natural to assume $C$ to have right duals.
We will do this frequently from now on. Recall that by definition any tensor
category has right and left duals.

\begin{proposition}\label{univ_bal}
  Suppose $C$ is a monoidal category with right duals and $M_C$, $_{C}N$ are
  $C$-module categories. Let $L$ be a category. Then composition with
  $M\otimes N \to M\otimes_C N$ induces an equivalence of categories
  $Fun(M\otimes_C N,L)\to Fun^{C-bal}(M\otimes N,L)$.
\end{proposition}

\begin{proof}
  It follows from lemma \ref{surjective}, \ref{faithful} and \ref{full}.
\end{proof}

\begin{remark}
  Hence the pre-skein category $M \otimes_{C} N$ almost satisfies (and is characterized by) the
  universal property that defines the balanced tensor product $M \boxtimes_{C}
  N$, except that it is not abelian (thus not $k$-linear). We will fix this by
  defining the skein category $sk(M,C,N)$ as a certain completion of the
  pre-skein category.
\end{remark}

\begin{lemma}\label{surjective}
  $Fun(M\otimes_C N,L)\to Fun^{C-bal}(M\otimes N,L)$ is surjective on
  objects.
\end{lemma}

\begin{proof}
  Given a $C$-balanced functor $F:M\otimes N \to L$ with balancing $\beta$, we
  will construct $\tilde{F}:M\otimes_C N \to L$ such that its composite with
  $M\otimes N\to M\otimes_C N$ is $F$. Define $\tilde{F} = F$ on objects.
  For morphisms, let $\phi\in Hom_M(m,m'\lhd c)$, $\pi\in Hom_C(c,\bar{c})$
  and $\psi\in Hom_N(\bar{c}\rhd n,n')$. We define
  $\tilde{F}([\phi\otimes\pi\otimes\psi])$ to be the composite

  \[
    \begin{tikzcd}
      {(m,n)} & {(m'\lhd c,n)} & {(m'\lhd \bar{c},n)} & {(m',\bar{c}\rhd n)} & {(m',n')}
      \arrow["{F(\phi,n)}", from=1-1, to=1-2]
      \arrow["{F(m'\lhd\pi,n)}", from=1-2, to=1-3]
      \arrow["{\beta_{m',\bar{c},n}}", from=1-3, to=1-4]
      \arrow["{F(m',\psi)}", from=1-4, to=1-5]
    \end{tikzcd}.
  \]

  \noindent The naturality of $\beta$ with respect to morphisms in $C$ implies
  that this is equal to the composite

  \[
    \begin{tikzcd}
      {(m,n)} & {(m'\lhd c,n)} & {(m', c\rhd n)} & {(m',\bar{c}\rhd n)} & {(m',n')}
      \arrow["{F(\phi,n)}", from=1-1, to=1-2]
      \arrow["{\beta_{m',c,n}}", from=1-2, to=1-3]
      \arrow["{F(m',\pi\rhd n)}", from=1-3, to=1-4]
      \arrow["{F(m',\psi)}", from=1-4, to=1-5]
    \end{tikzcd}.
  \]
  From these two expressions it is clear that this respects the relations in
  the definition of $Hom_{M\otimes_C N}$. The proof that this respects
  composition is a straightforward calculation, using the naturality of
  $\beta$ with respect to morphisms in $M$ and $N$.
  \end{proof}

  \begin{remark}
    Using lemma \ref{decompose}, the functor $\tilde{F}:M\otimes_C N\to L$
    whose composite with $M\otimes N\to M\otimes_C N$ is the $C$-balanced
    functor $F:M\otimes N\to L$ with balancing $\beta$ is also determined by
    $\tilde{F}(m,n)=F(m,n)$,
    $\tilde{F}(\phi\otimes\id_1\otimes\psi)=F(\phi,\psi)$ and
    $\tilde{F}(\id_{m\lhd c}\otimes\id_c\otimes\id_{c\rhd n})=\beta_{m,c,n}.$
  \end{remark}

\begin{lemma}\label{faithful}
  $Fun(M\otimes_C N,L)\to Fun^{C-bal}(M\otimes N,L)$ is faithful.
\end{lemma}

\begin{proof}
  This is immediate from the fact that $M\otimes N \to M\otimes_C N$ is
  surjective on objects.
\end{proof}

\begin{lemma}\label{full}
  $Fun(M\otimes_C N,L)\to Fun^{C-bal}(M\otimes N,L)$ is full.
\end{lemma}

\begin{proof}
  Given $\eta:F\to G$ in $Fun^{C-bal}(M\otimes N,L)$ we must define
  $\tilde{\eta}:\tilde{F}\to\tilde{G}$ in $Fun(M\otimes_C N,L)$. Denote by
  $\alpha$ and $\beta$ the balancing of $F$ and $G$, respectively. Since
  $M\otimes N\to M\otimes_C N$ is surjective on objects, we are forced to
  define $\tilde{\eta}_{m,n}=:\eta_{m,n}$ for every object $(m,n)\in
  M\otimes_C N$. But we need to check that this defines a natural
  transformation. By lemma \ref{decompose}, it is enough to check that it is
  natural with respect to maps of the form $(\phi\otimes\id_1\otimes\psi)$ and
  $(\id_{m\lhd c}\otimes\id_c\otimes\id_{c\rhd n})$. Naturality with respect
  to $(\phi\otimes\id_1\otimes\psi)$ follows directly from the naturality of
  $\eta$. Now $\eta$ is balanced, which means that

  \[
    \begin{tikzcd}
      {F(m\lhd c, n)} & {G(m\lhd c, n)} \\
      {F(m,c\rhd n)} & {G(m,c\rhd n)}
      \arrow["{\eta_{m\lhd c, n}}", from=1-1, to=1-2]
      \arrow["{\alpha_{m,c,n}}"', from=1-1, to=2-1]
      \arrow["{\beta_{m,c,n}}", from=1-2, to=2-2]
      \arrow["{\eta_{m,c\rhd n}}"', from=2-1, to=2-2]
    \end{tikzcd}\]
  commutes. Therefore
  \[
    \begin{tikzcd}
      {\tilde{F}(m\lhd c, n)} & {\tilde{G}(m\lhd c, n)} \\
      {\tilde{F}(m,c\rhd n)} & {\tilde{G}(m,c\rhd n)}
      \arrow["{\tilde{\eta}_{m\lhd c, n}}", from=1-1, to=1-2]
      \arrow["{\tilde{F}(\id_{m\lhd c}\otimes\id_c\otimes\id_{c\rhd n})}"', from=1-1, to=2-1]
      \arrow["{\tilde{G}(\id_{m\lhd c}\otimes\id_c\otimes\id_{c\rhd n})}", from=1-2, to=2-2]
      \arrow["{\tilde{\eta}_{m,c\rhd n}}"', from=2-1, to=2-2]
    \end{tikzcd}
  \]

  \noindent commutes, i.e. $\eta$ is natural with respect to $(\id_{m\lhd
  c}\otimes\id_c\otimes\id_{c\rhd n})$.
\end{proof}

\subsection{Skein Category sk(M,C,N)}

\begin{definition} (Skein Category, $sk(M,C,N)$)

  \noindent Let $C$ be a monoidal category with right duals, and let $M_C$,
  $_{C}N$ be module categories over $C$. Define the skein category $sk(M,C,N)$
  to be the Cauchy completion $\Cau(M \otimes_{C} N)$ (cf
  \ref{definition/cauchy-completion/abstract},
  \ref{definition/cauchy-completion/explicit}) of the pre-skein category $M
  \otimes_{C} N$.
\end{definition}

\begin{remark}\label{canonical_map}
  There is a $C$-balanced functor $M\otimes N \to sk(M,C,N)$ obtained by the
  composite $M\otimes N\to M\otimes_C N\to sk(M,C,N)$, where the first arrow
  is the $C$-balanced functor in \ref{is_balanced} (thus the composite is
  automatically $C$-balanced), and the second is the canonical functor to
  (\ref{definition/cauchy-completion/abstract}).
\end{remark}

\begin{remark}
  An object in $sk(M,C,N)$ is an idempotent matrix
  $A:(m_1,n_1)\oplus\cdots\oplus (m_k,n_k)\to (m_1,n_1)\oplus\cdots\oplus
  (m_k,n_k)$ whose $(i,j)$-entry is a morphism $(m_j,n_j)\to (m_i,n_i)$ in
  $M\otimes_C N$.
\end{remark}

\begin{proposition}\label{univ_sk}
  Let $L$ be Cauchy complete and suppose $C$ has right duals. Then composition
  with the $C$-balanced functor $M\otimes N\to sk(M,C,N)$ induces an
  equivalence of categories $$Fun(sk(M,C,N),L)\to Fun^{C-bal}(M\otimes N,L).$$
\end{proposition}

\begin{proof}
  We wish to show the composite $$Fun(sk(M,C,N),L)\to Fun(M\otimes_C N, L)\to
  Fun^{C-bal}(M\otimes N,L)$$ is an equivalence. The first functor is an
  equivalence by the universal property of the Cauchy completion
  (\ref{univ_prop_cau}) and the second one is an equivalence by proposition
  \ref{univ_bal}.
\end{proof}

\begin{remark}
  Comparing \ref{univ_sk} with \ref{definition/balanced-tensor-product}, the
  defining universal property of balanced tensor product, we know we are
  closed to proving that the skein category $sk(M,C,N)$ is indeed equivalent
  to the balanced tensor product. What is missing so far are

  \begin{enumerate}
    \item $sk(M,C,N)$ is $k$-linear (in particular, abelian) (cf \ref{semisimple}).
    \item The equivalence in \ref{univ_sk} has to be restricted to their
          right-exact counterparts:
          $Lin(M \boxtimes_{C} N, L) \to Bilin^{C-bal}(M \otimes N, L)$ (cf \ref{sk_bal}).
  \end{enumerate}
\end{remark}

\subsection{Proof of the Main Theorem}

We will show that $M\otimes N\to sk(M,C,N)$ presents $sk(M,C,N)$ as the
balanced tensor product $M\boxtimes_C N$ whenever $C$ is a finite tensor
category and $M$, $N$ and $M\boxtimes_C N$ are finite semisimple. To quickly
convince the reader that such result is correct, we also proved a degenerated
statement $sk(M,C,C) \simeq M$ directly in \ref{proposition/degenerated-main-theorem}.

\begin{remark} In \cite{douglas/balanced-product}, the existence of the balanced tensor
product is established, when $C$ is a finite tensor category and $M_C,_{C}N$ 
are finite $k$-linear $C$-module categories. That is the reason for the appearance of the finiteness conditions
on $M$, $C$, $N$ and the rigidity assumption on $C$ in the statements of all results in
the present paper which involve the balanced tensor product.\end{remark}

In \cite{douglas/balanced-product} the authors remark that their construction of $M\boxtimes_C N$ 
allows them to extend its universal property (as described in \ref{definition/balanced-tensor-product})
to the case when the target category $L$ is just a finitely cocomplete category.
We record this in the following lemma.

\begin{lemma}\label{univ_box}

  \noindent
  Let $C$ be a finite tensor category, $M_C$, $_{C}N$ finite $k$-linear
  $C$-module categories, and $L$ a finitely cocomplete category. (Thus the
  balanced tensor product $M \boxtimes_C N$ exists by
  \cite{douglas/balanced-product}.) Then the restriction functor is an
  equivalence:
  \[Lin(M \boxtimes_{C} N, L) \xrightarrow{\sim} Bilin^{C-bal}(M \otimes N, L).\]
\end{lemma}

\begin{proof}
  If $L$ is assumed to be $k$-linear, this follows from the universal property
  of the balanced tensor product. Here, $L$ is only assumed to be finitely
  cocomplete, and a proof is outlined in \cite[Remark 3.4]{douglas/balanced-product}.
  % NOTE This argument is given by Manuel. Jin does not understand it well.
\end{proof}

\begin{notation}
  Given a category $M$, we denote by $Fin(M)$ its finite cocompletion
  (\ref{definition/finite-cocompletion}).
\end{notation}

\noindent Recall from \ref{is_balanced} that we have a $C$-balanced functor $M
\otimes N \to M\otimes_C N$ provided that $C$ has right duals. Composing it
with $M\otimes_C N \to Fin(M\otimes_C N)$ provides a $C$-balanced functor
$M\otimes N\to Fin(M\otimes_C N)$ (because the first one is $C$-balanced).

\begin{lemma}\label{univ_finbal}

  \noindent Let $C$ be a monoidal category having right duals, let $M_C$ and $_{C}N$ be
  module categories over $C$, and let $L$ be a finitely cocomplete category.
  Then the restriction map $$Lin(Fin(M\otimes_C N),L)\to Fun^{C-bal}(M\otimes
  N,L)$$ is an equivalence.
\end{lemma}

\begin{proof}
  That $Lin(Fin(M\otimes_C N),L) \simeq Fun(M \otimes_{C} N, L)$ follows from
  a claim in \ref{definition/finite-cocompletion}, and that $Fun(M \otimes_{C}
  N, L) \simeq Fun^{C-bal}(M \otimes N, L)$ follows from \ref{univ_bal}.
\end{proof}

\begin{proposition}\label{fin_eq_bal}

  \noindent Suppose $C$ is a finite tensor category and $M_C$ and $_{C}N$ are $k$-linear
  finite semisimple module categories. Then we have a canonical equivalence of
  categories $Fin(M\otimes_C N) \simeq M\boxtimes_C N$.
  % Then the map $Fin(M\otimes_C N)\to M\boxtimes_C N$ corresponding to the
  % $C$-balanced functor $M\otimes N\to M\boxtimes_C N$ via the equivalence of
  % lemma \ref{univ_finbal} is an equivalence.
\end{proposition}

\begin{proof}

  \noindent Since $M,N$ are semisimple, any functor $M\otimes N\to L$ is
  bilinear (right exact in each variable).
  So by \ref{univ_finbal} the map \[Lin(Fin(M \otimes_{C} N), L) \to
  Bilin^{C-bal}(M \otimes N, L)\] defined by composing with $M\otimes N\to
  Fin(M\otimes_C N)$ is an equivalence for any finitely cocomplete $L$.
  By \ref{univ_box}, composing with $M\otimes N\to M\boxtimes_C N$ gives
  an equivalence \[Lin(M \boxtimes_{C} N, L) \xrightarrow{\sim}
  Bilin^{C-bal}(M \otimes N, L)\] for any finitely cocomplete category $L$.

  Thus, $Fin(M\otimes_C N)$ and $M \boxtimes_{C} N$ are both finitely cocomplete
  categories and are both characterized by the same universal property.
  Therefore, they are canonically equivalent.
\end{proof}

Now we prove that $sk(M,C,N)$ is a finite semisimple $k$-linear category,
whenever $M$, $N$ and $M\boxtimes_C N$ are finite semisimple $k$-linear
categories.

\begin{lemma}\label{semisimple}
  Let $C$ be a finite tensor category, and let $M_C$,$_{C}N$ be $k$-linear finite
  semisimple $C$-module categories. Suppose $M\boxtimes_C N$ is also
  $k$-linear finite semisimple.
  Then $sk(M,C,N)$ is $k$-linear finite
  semisimple.
\end{lemma}

\begin{proof}
  We have a fully faithful inclusion $sk(M,C,N)=Cau(M\otimes_C
  N)\hookrightarrow Fin(M\otimes_C N)$. By proposition \ref{fin_eq_bal}, we
  have $Fin(M\otimes_C N)\simeq M\boxtimes_C N$, because $M$, $N$ are finite
  semisimple. Therefore $sk(M,C,N)$ is finite semisimple, by
  proposition \ref{cau_semi}.
\end{proof}

\begin{lemma}\label{sk_bal}
  Let $C$ be a finite tensor category, and let $M_C$,$_{C}N$ be $k$-linear finite
  semisimple $C$-module categories. Suppose $M\boxtimes_C N$ is also
  $k$-linear finite semisimple. Then the $C$-balanced functor $M\otimes N\to
  sk(M,C,N)$ exhibits $sk(M,C,N)$ as the balanced tensor product $M\boxtimes_C
  N$.
\end{lemma}

\begin{proof}
  Recall that proposition \ref{univ_sk} asserts that the restriction map

  \[
   Fun(sk(M,C,N),L) \xrightarrow{\sim} Fun^{C-bal}(M \otimes N,L)
  \]

  is an equivalence for every Cauchy complete category $L$. Comparing this
  with the defining universal property of balanced tensor product
  (cf \ref{definition/balanced-tensor-product}), we need to show that
  $sk(M,C,N)$ is $k$-linear (in particular, abelian), and to restrict the
  equivalence to their right-exact counterparts:

  \[
    Lin(sk(M,C,N),L) \xrightarrow{\sim} Bilin^{C-bal}(M \otimes N, L).
  \]

  By \ref{semisimple} we know $sk(M,C,N)$ is $k$-linear, so that resolves the
  first part, and we also know that it is semisimple, so

  \[
    Fun(sk(M,C,N), L) = Lin(sk(M,C,N), L).
  \]

  Finally,

  \[
    Fun^{C-bal}(M \otimes N, L) = Bilin^{C-bal}(M \otimes N, L)
  \]
  because $M$ and $N$ are semisimple.
\end{proof}

\begin{remark}\label{semisimple_douglas/dualizable-tensor-categories}

  \noindent Collecting results
  from \cite{douglas/dualizable-tensor-categories}, we can conclude that
  $M\boxtimes_C N$ is finite semisimple (so lemma \ref{sk_bal} applies) in any
  of the following situations (in decreasing level of generality):

  \begin{itemize}

    \item $C$ is a finite semisimple tensor category and $_{\Vect_k}M_C$,
    $_{C}N_{\Vect_k}$ are separable bimodule categories (\cite[proposition
    2.5.3, theorem 2.5.5]{douglas/dualizable-tensor-categories});

    \item $k$ is perfect, $C$ is a separable tensor category and $M_C,_{C}N$ 
    are finite semisimple $C$-module categories (\cite[proposition
    2.5.10]{douglas/dualizable-tensor-categories});

    \item $k$ has characteristic $0$, $C$ is a finite semisimple tensor category 
    and $M_C,_{C}N$ are finite semisimple $C$-module categories
    (\cite[Corollary 2.6.9]{douglas/dualizable-tensor-categories});

  \end{itemize}
\end{remark}

\noindent Since we are mostly interested in the characteristic zero case, we record the
following direct consequence of lemma \ref{sk_bal}.

\begin{theorem} \label{main-theorem-ver-2}
   Suppose $k$ has characteristic $0$, $C$ is a finite semisimple tensor
   category, and $M_C$,$_{C}N$ are $k$-linear finite semisimple $C$-module
   categories. Then the $C$-balanced functor $M\otimes N\to sk(M,C,N)$
   exhibits $sk(M,C,N)$ as the balanced tensor product $M\boxtimes_C
   N$.
\end{theorem}

\begin{remark}\label{iterated_remark}
One can immediately generalize \ref{pre-skein} to obtain a definition of $M^{1} \otimes_{C_{1}} M^{2} \otimes_{C_{2}} \ldots \otimes_{C_{n{\text -}1}} M^{n}$ where morphisms are skeins as in the picture below. \begin{center}\includesvg{drawing-3.svg}\end{center}

Then one can define $sk(M^{1}, C_{1}, M^{2}, C_{2}, \ldots, C_{n{\text -} 1}, M^{n})$ by Cauchy completion and thus extend Theorem \ref{main-theorem-ver-2} to iterated products.

\end{remark}


\begin{corollary}\label{iterated}
  Suppose $k$ is a field of characteristic $0$. Let $C_{1}, \ldots,
  C_{n-1}$ be finite semisimple tensor categories, and let $M^{1}_{C_{1}},
  _{C_{1}}M^{2}_{C_{2}}, \ldots, _{C_{n{\text -}1}}M^{n}$ be
  $k$-linear finite semisimple bimodule categories. Then we have an equivalence
  of categories

  \[
    sk(M^{1}, C_{1}, M^{2}, C_{2}, \ldots, C_{n{\text -} 1}, M^{n})
    \simeq
    M^{1} \boxtimes_{C_{1}} M^{2} \boxtimes_{C_{2}} \ldots \boxtimes_{C_{n{\text -}1}} M^{n}.
  \]
\end{corollary}

% \begin{proof}
% 
% Note that $M^{1} \boxtimes_{C_{1}} M^{2} \boxtimes_{C_{2}} \ldots \boxtimes_{C_{n{\text -}1}} M^{n}$ is an abelian category, characterized by a universal property: the functor $Lin(M^{1} \boxtimes_{C_{1}} M^{2} \boxtimes_{C_{2}} \ldots \boxtimes_{C_{n{\text -}1}} M^{n},L)\to Multilin^{C_i-bal}(M^{1} \otimes_{C_{1}} M^{2} \otimes_{C_{2}} \ldots \otimes_{C_{n{\text -}1}} M^{n},L)$ given by precomposition with $M^{1} \otimes_{C_{1}} M^{2} \otimes_{C_{2}} \ldots \otimes_{C_{n{\text -}1}} M^{n}\to M^{1} \boxtimes_{C_{1}} M^{2} \boxtimes_{C_{2}} \ldots \boxtimes_{C_{n{\text -}1}} M^{n}$ is an equivalence.
% 
% We can show that $sk(M^{1}, C_{1}, M^{2}, C_{2}, \ldots, C_{n{\text -} 1}, M^{n})$ satisfies the same universal property by first establishing a universal property of $(M^{1} \otimes_{C_{1}} M^{2} \otimes_{C_{2}} \ldots \otimes_{C_{n{\text -}1}} M^{n})$ analogous to \ref{univ_bal}, and then using the universal property of the Cauchy completion (\ref{univ_prop_cau}). 
% 
% Finally, to show that $sk(M^{1}, C_{1}, M^{2}, C_{2}, \ldots, C_{n{\text -} 1}, M^{n})$ is abelian we exploit the fact that  $M^{1} \boxtimes_{C_{1}} M^{2} \boxtimes_{C_{2}} \ldots \boxtimes_{C_{n{\text -}1}} M^{n}$ is semisimple and argue as in \ref{semisimple}.
% \end{proof}

\begin{remark}
Just like theorem \ref{main-theorem-ver-2}, corollary \ref{iterated} also holds in any of the more general settings described in remark \ref{semisimple_douglas/dualizable-tensor-categories}.
\end{remark}


\appendix
\section{Appendix}
   
\subsection{Completions}

\noindent In this subsection, we collect results about completions of categories.

\begin{definition} (Cauchy Complete)

  \noindent A ($\Vect_k$-enriched) category $M$ is Cauchy complete if it has
  all finite direct sums and all idempotents split.
\end{definition}

% <<Acknowledgement of Gap>> Below we collect well known facts about Cauchy
% completion in the $\Vect_k$-enriched setting. Unfortunately, we have not
% been able to find satisfactory references for all of this. The main
% reference we use is from Kelly's work, which only mentions the Ab-enriched
% statements (instead of Vect_k) without proofs.

\begin{definition} \label{definition/cauchy-completion/abstract} (Cauchy Completion (abstract version)) \cite[Sections 5.5 and 5.7]{kelly/basic-concepts-enriched}.

  \noindent Given a category $M$, we denote by $\Cau(M)$ the Cauchy completion
  of $M$, the smallest subcategory of $Fun(M^{op},\Vect_k)$ containing
  all the representables and closed under finite direct sums and retracts. Its
  objects can be identified as those $X \in Fun(M^{op},\Vect_k)$ such that
  $$Hom_{Fun(M^{op},\Vect_k)}(X,-): Fun(M^{op},\Vect_k) \to \Vect_k$$ preserves all
  small colimits. Alternatively, they are the retracts of finite direct sums
  of representables.
\end{definition}
 
\begin{lemma}\label{univ_prop_cau}

  \noindent For any category $M$, the category $Cau(M)$ is Cauchy complete.
  The Yoneda embedding induces a fully faithful functor $M \hookrightarrow
  Cau(M)$, restriction of which gives an equivalence: $$Fun(\Cau(M),L)
  \xrightarrow{\sim} Fun(M,L)$$ for any Cauchy complete category $L$
\end{lemma}

\begin{proof}
  See \cite[Sections 5.5 and 5.7]{kelly/basic-concepts-enriched}.
\end{proof}

\noindent Next, we recall the usual explicit construction of the Cauchy
completion $Cau(M)$ of a ($\Vect_k$-enriched) category $M$ in
\ref{definition/cauchy-completion/explicit}.

\begin{definition} (Matrix Category)

  \noindent Given a category $M$, we define $Mat(M)$ to be the category whose
  objects are tuples of objects in $M$. We denote such an object by
  $m_1\oplus\cdots\oplus m_k$. We then
  define $$Hom_{Mat(M)}(m_1\oplus\cdots\oplus m_k,n_1\oplus\cdots\oplus
  n_{\ell})$$ to be the $k$-vector space of $(\ell\times k)$ matrices whose
  $(i,j)$-th entry is a morphism $m_j\to n_i$ in $M$. Composition is defined
  by matrix product.
\end{definition}

\begin{definition} (Karoubi Completion)

  \noindent Given a category $M$, we define its Karoubi completion $Kar(M)$ to
  be the category whose objects are idempotent endomorphisms $m
  \xrightarrow{p} m$ in $M$, and the hom space $Hom_{Kar(C)}(m \xrightarrow{p}
  m, n \xrightarrow{q} n)$ is the $k$-linear subspace of $Hom_{M}(m,n)$
  consisting of $f$ such that the following diagram commutes.
  \[
    \begin{tikzcd}
    m & n \\
    m & n
    \arrow["f", from=1-1, to=1-2]
    \arrow["p", from=1-1, to=2-1]
    \arrow["f", from=1-1, to=2-2]
    \arrow["q", from=1-2, to=2-2]
    \arrow["f"', from=2-1, to=2-2]
    \end{tikzcd}
  \]
\end{definition}

\begin{definition} \label{definition/cauchy-completion/explicit} (Cauchy Completion (explicit version))

  \noindent The explicit construction of the Cauchy completion is given by
  $Cau(M)=Kar(Mat(M))$.
\end{definition}

\noindent So an object in $Cau(M)$ is an idempotent
matrix $$m_1\oplus\cdots\oplus m_k \xrightarrow{A} m_1\oplus\cdots\oplus
m_k.$$

\noindent We introduce a few lemmas about Cauchy complete categories.

\begin{lemma} \label{direct_sum}
  Suppose we have a fully faithful functor $F:M\hookrightarrow N$ where $M$ is
  Cauchy complete. If $F(x)=a'\oplus b'$ then we have $x=a\oplus b$ and
  isomorphisms $F(a)\simeq a'$, $F(b)\simeq b'$ such that the following
  diagrams commute.
  \[
    \begin{tikzcd}
      {F(a)} & {F(x)} & {F(b)} \\
      {a'} & {F(x)} & {b'}
      \arrow[from=1-1, to=1-2]
      \arrow["\simeq"', from=1-1, to=2-1]
      \arrow["{=}", from=1-2, to=2-2]
      \arrow[from=1-3, to=1-2]
      \arrow["\simeq", from=1-3, to=2-3]
      \arrow[from=2-1, to=2-2]
      \arrow[from=2-3, to=2-2]
    \end{tikzcd} \begin{tikzcd}
      {F(a)} & {F(x)} & {F(b)} \\
      {a'} & {F(x)} & {b'}
      \arrow["\simeq"', from=1-1, to=2-1]
      \arrow[from=1-2, to=1-1]
      \arrow[from=1-2, to=1-3]
      \arrow["{=}", from=1-2, to=2-2]
      \arrow["\simeq", from=1-3, to=2-3]
      \arrow[from=2-2, to=2-1]
      \arrow[from=2-2, to=2-3]
    \end{tikzcd}
  \]
\end{lemma}

\begin{proof}
  Denote by $i_{a'}:a'\to F(x)$, $r_{a'}:F(x)\to a'$ and $p_{a'}:F(x)\to F(x)$
  the corresponding inclusion, retraction and idempotent, so that
  $r_{a'}i_{a'}=\id_{a'}$ and $i_{a'} r_{a'}=p_{a'}$, and similarly for $b'$.
  We have additional equations $r_{a'}i_{b'}=0=r_{b'}i_{a'}$ and
  $p_{a'}+p_{b'}=\id_{F(x)}$. Now $F$ is full, so there exists $p_a:x\to x$
  such that $F(p_a)=p_{a'}$ and similarly for $p_b$. Now $M$ is idempotent
  complete, so $p_a$ splits, i.e. we obtain $i_a:a\to x$ and $r_a:x\to a$ such
  that $r_a i_a=\id_a$ and $i_a r_a=p_a$ and similarly for $b$. Now
  $F(p_a+p_b)=p_{a'}+p_{b'}=\id_{F(x)}$. Since $F$ is faithful we get
  $p_a+p_b=\id_x$. This means that $x=a\oplus b$. Finally
  $r_{a'}F(i_a):F(a)\to a'$ and $r_{b'}F(i_b):F(b)\to b'$ are the desired
  isomorphisms.
\end{proof}

\begin{lemma}\label{abelian}

  \noindent Suppose we have a fully faithful functor $F:M\hookrightarrow N$
  where $M$ is Cauchy complete and $N$ is $k$-linear. Suppose every short
  exact sequence splits in $N$. Then $M$ is $k$-linear and every short exact
  sequence splits in $M$.
\end{lemma}

\begin{proof}
  We start by showing that $M$ is $k$-linear (essentially, we need to check
  that $M$ is abelian). We now show that $M$ has kernels, and they are
  preserved by $F$. (The proof for cokernels is dual to this proof).

  So let $x\to y$ be a morphism in $M$. Then $F(x)\to F(y)$ has a kernel
  $k'\to F(x)$ in $N$. The short exact sequence $0\to k'\to F(x)\to c'\to 0$
  splits, where $c'$ is the cokernel of $k'\to F(x)$. So we get $F(x)=k'\oplus
  c'$. Then, by lemma \ref{direct_sum}, we have $x=k\oplus c$ where
  $F(k)\simeq k'$ and

  \[
    \begin{tikzcd}
      {F(k)} & {F(x)} \\
      {k'} & {F(x)} \arrow[from=1-1, to=1-2]
      \arrow["\simeq"', from=1-1, to=2-1]
      \arrow["{=}", from=1-2, to=2-2]
      \arrow[from=2-1, to=2-2]
    \end{tikzcd}
  \]

  commutes. This means $F(k)\to F(x)$ is also a kernel of $F(x)\to F(y)$. But
  $F$ is fully faithful, so it reflects limits, hence $k\to x$ is a kernel of
  $x\to y$.

  Next, we show that every monomorphism is a kernel. (The proof that every
  epimorphism is a cokernel is dual to this proof). Let $a\hookrightarrow x$
  be a monomorphism in $M$. Since $F$ preserves kernels, we know
  $F(a)\hookrightarrow F(x)$ is a monomorphism. Moreover, if $x\to c$ is the
  cokernel of $a\hookrightarrow x$, then $F(x)\to F(c)$ is the cokernel of
  $F(a)\hookrightarrow F(x)$ (because $F$ also preserves cokernels). So $0\to
  F(a)\to F(x)\to F(c)\to 0$ is exact, so $F(x)=F(a)\oplus F(c)$. Then
  $x=a\oplus c$ (by lemma \ref{direct_sum}) and so $a\to x$ is the kernel of
  $x\to c$.

  This concludes the proof that $M$ is $k$-linear, and that $F$ is exact.

  Finally, we need that $F:M \to N$ is exact. The fact that every short exact
  sequence splits in $M$ follows easily from lemma \ref{direct_sum}, and the
  facts that $F$ is exact and every short exact sequence splits in $N$.
\end{proof}

\begin{proposition}\label{cau_semi}

  \noindent Suppose we have a fully faithful functor $M\hookrightarrow N$,
  where $M$ is Cauchy complete and $N$ is $k$-linear finite semisimple. Then
  $M$ is a $k$-linear finite semisimple category.
\end{proposition}

\begin{proof}
  By lemma \ref{abelian} the category $M$ also $k$-linear and every short
  exact sequence splits in $M$, which implies that every object in $M$ is
  projective and also that $F:M\to N$ exact. Since $F$ is faithful and $N$ is
  finite, all hom spaces in $M$ are finite dimensional.

  Since $F$ is exact and fully faithful, it preserves and reflects
  monomorphisms, therefore $x\in M$ is simple if and only if $F(x)\in N$ is
  simple. So in particular, given $N$ has finitely many isomorphism classes of
  simple objects, the same is true of $M$.

  If $F(x)=\bigoplus_{i=1}^n x'_i$ with $x'_i$ simple, then by induction and
  lemma \ref{direct_sum} we have $x=\bigoplus_{i=1}^n x_i$ where
  $F(x_i)\simeq x_i'$ so $x_i$ is simple.
\end{proof}

\begin{definition}\label{definition/finitely-cocomplete} (Finitely Cocomplete)

  \noindent A category $M$ is finitely cocomplete if it has all finite colimits.
\end{definition}


\begin{definition} \label{definition/finite-cocompletion}
  (Finite Cocompletion, $Fin(M)$) \cite[Section 5.7]{kelly/basic-concepts-enriched}, \cite[Section 2.2.1]{lopezfranco/tensor-products}.

  \noindent Given a category $M$, we denote by $Fin(M)$ its finite
  cocompletion. It is the smallest finitely cocomplete subcategory of
  $Fun(M^{op},\Vect_k)$ that contains all the representables. Its objects can
  be identified as those $X\in Fun(M^{op},\Vect_k)$ such that
  $Hom_{Fun(M^{op},\Vect_k)}(X,-):Fun(M^{op},\Vect_k)\to \Vect_k$ preserves
  filtered colimits. Alternatively, they are the coequalisers of pairs of
  morphisms between finite coproducts of representables. The Yoneda embedding
  induces a fully faithful functor $M\hookrightarrow Fin(M)$.
\end{definition}

\begin{lemma}\label{univ_prop_fin}

  \noindent For any finitely cocomplete $L$, the restriction map $Lin(Fin(M),L)\to
  Fun(M,L)$ is an equivalence.
\end{lemma}

\begin{proof}
 See \cite[Section 5.7]{kelly/basic-concepts-enriched} or \cite[Section
   2.2.1]{lopezfranco/tensor-products}.
\end{proof}

\begin{remark}
  From our descriptions of $Cau(M)$
  (\ref{definition/cauchy-completion/abstract}) and $Fin(M)$
  (\ref{definition/finite-cocompletion}), it is immediate that we have fully
  faithful functors $M\hookrightarrow Cau(M)\hookrightarrow Fin(M)$.
\end{remark}

\subsection{Rigidity}

\noindent This section explains the role of duals in the theory of the
balanced Kelly tensor product (of finitely cocomplete categories). Namely, the
existence of left or right duals in $C$ allows for the category $M \otimes_C
N$ to have finite dimensional hom spaces, whenever $M$, $N$ and $C$ have
finite dimensional hom spaces. The relevance of having both left and right
duals (rigidity) remains unclear to us from this perspective, although
rigidity is a standing assumption in the most general known construction of
the balanced Deligne tensor product of finite $k$-linear categories (see
\cite{douglas/balanced-product}).

Similar to the preskein category $M \otimes_C N$, we define the general
preskein category $M \hat{\otimes}_C N$ that works for monoidal categories $C$
not necessarily with duals in \ref{pre-skein_no-duals},
\ref{pre-skein_no-duals_2}. We then prove that they are in fact isomorphic
categories when $C$ has weak duals
(\ref{lemma/equivalence-of-two-preskein-cats}).

\begin{definition} \label{pre-skein_no-duals} (General Preskein Category, $M \hat{\otimes}_C N$)

  \noindent Let $C$ be a monoidal category (not necessarily with duals) and let $M_C$ and $_{C}N$ be
  $C$-module categories. We define $M \hat{\otimes}_C N$ as the category whose
  objects are pairs $(m,n)$ and whose morphism spaces are the spaces of
  alternating skeins of arbitrary (finite) length, as depicted below.

  \begin{center}
    \includesvg{graphics/up_down_skein}
  \end{center}

  This should of course be interpreted as a suitable direct sum of tensor
  products of morphism spaces in $M$, $C$ and $N$. We impose relations
  analogous to those in definition \ref{pre-skein}, plus the extra relations
  below.

  \begin{center}
    \begin{tabular}{l}
      \includesvg{graphics/relation2} \\
      \includesvg{graphics/relation2b} \\
      \includesvg{graphics/relation3} \\
      \includesvg{graphics/relation3b} \\
      \includesvg{graphics/relation4} \\
      \includesvg{graphics/relation4b}
    \end{tabular}
  \end{center}

  Each of these relations may be applied locally to any finite length
  alternating skein. In the case of the last two relations, after applying the
  relation one must apply the composition rule from definition \ref{pre-skein}
  to obtain an alternating skein again.

  Composition is defined by horizontally stacking the two diagrams and then
  applying the composition rule from definition \ref{pre-skein} if there are
  parallel diagonal wires.
\end{definition}

\begin{definition} \label{pre-skein_no-duals_2} (General Preskein Category (alternative))
 
  \noindent Let $C$ be a monoidal category and let $M_C$ and $_{C}N$ be
  $C$-module categories. We define $M \hat{\otimes}_C N$ by starting with $M \otimes
  N$, then freely adjoining morphisms $(m\lhd c, n)\to (m,c\rhd n)$ and
  $(m,c\rhd n)\to (m\lhd c, n)$ for each $m\in M$, $n\in N$ and $c\in C$
  (depicted below)

  \begin{center}
    \begin{tabular}{ll}
    \includesvg{graphics/generator1} & \includesvg{graphics/generator2}

    \end{tabular}
  \end{center}

  \noindent and finally imposing the relations below (where we denote composition by
  concatenation).

  \begin{center}
    \begin{tabular}{l}

    \includesvg{graphics/relation6} \\ \\
    \includesvg{graphics/relation7} \\ \\
    \includesvg{graphics/relation8} \\ \\
    \includesvg{graphics/relation9} where $\pi:c\to\bar{c}$ is a morphism in $C$ \\ \\
    \includesvg{graphics/relation10} where $\pi:c\to\bar{c}$ is a morphism in $C$
  \end{tabular}
\end{center}

\end{definition}

\begin{definition} \label{definition/weak-right-dual}
 
  \noindent Given an object $c$ in a monoidal category $C$, a weak right dual
  for $c$ is a triple $(c^*,\eta,\epsilon)$, where $c^*$ is an object in $C$
  and $\eta:1\to c\otimes c^*$ and $\epsilon:c^*\otimes c\to 1$ are morphisms
  in $c$, such that the following equation holds.

  \begin{center}
    \includesvg{graphics/snake_equation}
  \end{center}

  A monoidal category with weak right duals is a monoidal category where every
  object has a weak right dual.
\end{definition}

\begin{lemma}\label{up_skein}
  
  \noindent Let $C$ be a monoidal category with weak right duals and let $M_C$
  and $_{C}N$ be $C$-module categories. Suppose $(c^*,\eta,\epsilon)$ is a
  weak right dual for $c$.
  Then \begin{center}\includesvg{graphics/up_equals_dual_down}\end{center} in
  $M \hat{\otimes}_C N$.
\end{lemma}

\begin{proof}
  Both sides of the equation are inverse
  to \begin{center}\includesvg{down_skein}.\end{center} For the left hand
  side, this follows from the relations in definition
  \ref{pre-skein_no-duals}. For the right hand side, it follows from the proof
  of lemma \ref{beta_invertible}, where only the fact that
  $(c^*,\eta,\epsilon)$ is a weak right dual for $c$ is used.
\end{proof}

\begin{lemma} \label{lemma/equivalence-of-two-preskein-cats}
  
  \noindent Let $C$ be a monoidal category with weak right duals and let $M_C$
  and $_{C}N$ be $C$-module categories. Then the preskein category $M
  \otimes_C N$ and the general preskein category $M \hat{\otimes}_C N$ are
  isomorphic.
\end{lemma}

\begin{proof}
  There is an obvious functor $M \otimes_C N \to M \hat{\otimes}_C N$ which is a
  bijection on objects. To show that it is an isomorphism on morphism spaces,
  we use lemma \ref{up_skein} to reduce every morphism in $M \hat{\otimes}_C N$ to a
  morphism in $M\otimes_C N$.
\end{proof}

\begin{proposition}

  \noindent Let $C$ be a monoidal category, let $M_C$ and $_{C}N$ be
  $C$-module categories and let $L$ be a category. Then composition with
  $M \otimes N \to M \otimes_C N$ induces an equivalence of categories
  $Fun(M \hat{\otimes}_C N,L) \to Fun^{C-bal}(M \otimes N,L)$.
\end{proposition}

\begin{proof}
  The only difference between this and the proof of proposition \ref{univ_bal}
  is that now providing an inverse to the balancing $\beta_{m,c,n}:(m\lhd c,
  n)\to (m,c\rhd n)$ (see lemma \ref{beta_invertible}) does not require duals,
  as we can simply use the skein with a $c$ wire going up.
\end{proof}
  
\begin{remark}
  While read from right to left, the wires in the definition of preskein
  category go down, while the wires in the definition of the general preskein
  category go both ways. We could also define the ``opposite'' preskein
  category where the diagonal wires go up (read from right to left). This
  category is isomorphic to the general preskein category $M \hat{\otimes}_C
  N$ of definition \ref{pre-skein_no-duals} when $C$ has weak left duals
  (defined in an analogous way to weak right duals). When $C$ has both weak
  left and right duals (in particular, when $C$ is rigid), then all three
  categories are isomorphic.
\end{remark}

\subsection{The Non-semisimple Case} \label{section-nonsemisimple}

In this subsection, we extend the construction of the skein category
$sk(M,C,N)$ to the non-semisimple skein category $M \boxtimes_C^{sk} N$ for
$C, M, N$ that are not necessarily semisimple. While this generalizes what we
did in the main text of the paper, we put it in the appendix because the tools
we use here rely on several results in
\cite{kelly/basic-concepts-enriched}. For convenience, we include the
necessary results from \cite{kelly/basic-concepts-enriched} in section \ref{review_kelly})

\noindent The general idea is to start from $M\otimes_C N$ and then add finite
colimits, to obtain a finitely cocomplete category $M\boxtimes_C^{sk} N$.
However, we also want the functor $M\otimes N \to M\boxtimes^{sk}_C N$ to be
right exact in each variable.

When $M$, $N$ are semisimple, this is automatic, so we can simply take
$M\boxtimes_C^{sk} N = Fin(M\otimes_C N)$, i.e. the finite cocompletion of
$M\otimes_C N$, i.e. the closure of the image of $$M\otimes_C N\hookrightarrow
Fun((M\otimes_C N)^{op},\Vect_k)$$ under finite colimits.

When $M$, $N$ are not necessarily semisimple, we must simultaneously complete
$M\otimes_C N$ under finite colimits and also force diagrams coming from
colimit diagrams in $M, N$ to become colimits in $M \boxtimes_C^{sk} N$. This
is achieved by changing from $Fun$ to $Bilex$ (both are the same in the
semisimple case). More precisely, we define $M\boxtimes_C^{sk} N$ as the
closure under finite colimits of the image of
\[
  M \otimes_C N
  \hookrightarrow
  Fun((M\otimes_C N)^{op},\Vect_k)
  \xrightarrow{R}
  Bilex((M\otimes_C N)^{op},\Vect_k),
\]
where, the first functor is the Yoneda embedding, and the second functor $R$
is right adjoint to the inclusion $Bilex((M\otimes_C N)^{op},\Vect_k)
\hookrightarrow Fun((M\otimes_C N)^{op},\Vect_k)$ of the full subcategory. The
existence of $R$ is the most substantial result from
\cite{kelly/basic-concepts-enriched} that we rely on.

\begin{definition} \label{definition/kelly-balanced-tensor-product} (Kelly Balanced Tensor Product)

  \noindent Suppose $C$ is a monoidal category and $M_C, _{C}N$ are finitely
  cocomplete $C$-module categories. Their balanced Kelly tensor product is a
  finitely cocomplete category $M\boxtimes^K_C N$ together with a $C$-balanced
  bilinear functor $M\otimes N\to M\boxtimes^K_C N$ such that precomposition
  defines an equivalence $$Lin(M\boxtimes^K_C N,L)\to Bilin^{C-bal}(M\otimes
  N, L)$$ for any finitely cocomplete category $L$.
\end{definition}

\noindent Suppose $C$ is a monoidal category and $M_C, _{C}N$ are finitely cocomplete
$C$-module categories. We explain how to construct a category
$M\boxtimes_C^{sk}N$ so that $M\otimes N\to M\boxtimes_C^{sk}N$ presents it as
the balanced Kelly tensor product of $M$ and $N$ over $C$. We follow very
closely the procedure adopted in \cite{lopezfranco/tensor-products} to
construct the unbalanced Kelly tensor product, starting from $M\otimes N$.

\begin{definition}

  \noindent A functor $(M\otimes_C N)^{op}\to L$ is said to be left exact in
  each variable if the composite $M^{op}\otimes N^{op}\to (M\otimes_C
  N)^{op}\to L$ is left exact in each variable. In other words it sends finite
  colimits in $M$ and finite colimits in $N$ to finite limits in $L$. We
  denote by $Bilex((M\otimes_C N)^{op},\Vect_k)\subset Fun(M\otimes_C
  N)^{op},\Vect_k)$ the full subcategory whose objects are those functors
  which are left exact in each variable.
\end{definition}

\begin{definition}

  \noindent We say that a functor $M\otimes_C N \to L$ is bilinear if the
  composite $M\otimes N\to M\otimes_C N \to L$ is bilinear (i.e. right exact
  in each variable). We denote by $Bilin(M\otimes_C N, L)\subset
  Fun(M\otimes_C N, L)$ the full subcategory whose objects are the bilinear
  functors.
\end{definition}

\noindent The subcategory $Bilex((M\otimes_C N)^{op},\Vect_k)\subset Fun((M\otimes_C
N)^{op},\Vect_k)$ is reflective (see \cite[theorem
  6.5]{kelly/basic-concepts-enriched} or theorem \ref{reflective}) so that we
have an adjunction

\[\begin{tikzcd}
            {Bilex((M\otimes_C N)^{op},\Vect_k)} & {Fun((M\otimes_C N)^{op},\Vect_k)}
            \arrow[""{name=0, anchor=center, inner sep=0}, "i"', shift right=3, hook, from=1-1, to=1-2]
            \arrow[""{name=1, anchor=center, inner sep=0}, "R"', shift right=3, from=1-2, to=1-1]
            \arrow["\dashv"{anchor=center, rotate=-90}, draw=none, from=1, to=0]
\end{tikzcd}.\]
In particular, $Bilex((M\otimes_C N)^{op},\Vect_k)$ is cocomplete.

\begin{definition}\label{def_K}

  \noindent We denote by $K:M\otimes_C N\to Bilex((M\otimes_C N)^{op}$ the composite
  \[
  \begin{tikzcd}
    {M\otimes_C N} & {Fun((M\otimes_C N)^{op},\Vect_k)} & {Bilex((M\otimes_C N)^{op},\Vect_k)}
    \arrow["Y", from=1-1, to=1-2]
    \arrow["R", from=1-2, to=1-3]
  \end{tikzcd},
  \]
  where $Y$ is the Yoneda embedding.\end{definition}
\begin{lemma}\label{right_exact_0}

The functor $K:M\otimes_C N\to Bilex((M\otimes_C N)^{op}$ is bilinear.\end{lemma}

\begin{proof}

  This follows from assertion (5.51) in \cite{kelly/basic-concepts-enriched}.
  But we can give the following proof. We must show that $Bilex((M\otimes_C
  N)^{op},\Vect_k)(RY-,\phi):(M\otimes_C N)^{op}\to\Vect_k$ is left exact in
  each variable, for any $\phi\in Bilex((M\otimes_C N)^{op},\Vect_k)$. This
  follows from the calculation $$Bilex((M\otimes_C
  N)^{op},\Vect_k)(RY-,\phi)\simeq Fun((M\otimes_C
  N)^{op},\Vect_k)(Y-,\phi)\simeq \phi$$ where the first step uses the
  adjunction $R\dashv i$ and the second step is the Yoneda lemma.
\end{proof}

\begin{definition}\label{sk_nonsemisimple}

  \noindent Let $C$ be a monoidal category and let $M_C$, $_{C}N$ be finitely
  cocomplete $C$-module categories. We define $M\boxtimes_C^{sk}N$ as the
  closure of the full image of the functor $$K:M\otimes_C N \to
  Bilex((M\otimes_C N)^{op},\Vect_k)$$ under finite colimits. This means
  $M\boxtimes_C^{sk}N$ is the smallest replete full subcategory of
  $Bilex((M\otimes_C N)^{op},\Vect_k)$ which contains every object of the form
  $K(m,n)$ and is closed under finite colimits. Thus $M\boxtimes_C^{sk}N$ is
  finitely cocomplete and comes equipped with a functor $Z:M\otimes_C N \to
  M\boxtimes_C^{sk}N$.
\end{definition}

\begin{lemma}\label{right_exact}
  The functor $Z:M\otimes_C N \to M\boxtimes_C^{sk}N$ is bilinear.
\end{lemma}

\begin{proof}
  This follows from lemma \ref{right_exact_0} and the fact that the fully
  faithful functor $$M\boxtimes_C^{sk}N\hookrightarrow Bilex((M\otimes_C
  N)^{op},\Vect_k)$$ reflects colimits.
\end{proof}

\begin{proposition}\label{univ_sk_nonsemisimple}

  \noindent Given a finitely cocomplete category $L$, the
  functor $$Lin(M\boxtimes_C^{sk}N,L)\to Bilin(M\otimes_C N, L)$$ given by
  precomposition with $Z$ is an equivalence.
\end{proposition}

\begin{proof}
  See \cite[theorem 6.23]{kelly/basic-concepts-enriched} or theorem
  \ref{f_theory_small_sketch}.
\end{proof}

\begin{lemma}\label{univ_bilin_bal}

  \noindent The functor $Bilin(M\otimes_C N,L)\to Bilin^{C-bal}(M\otimes N,
  L)$ is an equivalence, for any category $L$.
\end{lemma}

\begin{proof}
By lemma \ref{univ_bal} we know that $Fun(M\otimes_C N,K)\to
Fun^{C-bal}(M\otimes N, L)$ is an equivalence. By definition, a functor
$M\otimes_C N\to L$ is bilinear if and only if the composite $M\otimes N\to
M\otimes_C N\to L$ is bilinear, so we obtain an equivalence between the two
subcategories.
\end{proof}

\begin{theorem}\label{main_nonsemisimple}

  \noindent Let $C$ be a monoidal category and let $M_C$, $_{C}N$ be finitely
  cocomplete module categories. Then the $C$-balanced functor $M\otimes N\to
  M\boxtimes_C^{sk}N$ presents $M\boxtimes_C^{sk}N$ as the balanced Kelly
  tensor product of $M$ and $N$ over $C$.
\end{theorem}

\begin{proof}

This follows from proposition \ref{univ_sk_nonsemisimple} and lemma
\ref{univ_bilin_bal}.
\end{proof}

\begin{corollary}

  \noindent Let $C$ be a finite tensor category, $M_C$, $_{C}N$ finite
  $k$-linear $C$-module categories. Then $M\boxtimes_C^{sk}N$ is $k$-linear
  and the $C$-balanced functor $M\otimes N\to M\boxtimes_C^{sk}N$ presents
  $M\boxtimes_C^{sk}N$ as the balanced Deligne tensor product of $M$ and $N$
  over $C$.
\end{corollary}

\begin{proof}
  This follows from theorem \ref{main_nonsemisimple} and lemma \ref{univ_box}.
\end{proof}

\noindent The following lemma makes the connection between the construction of
$M\boxtimes_C^{sk}N$ in this section and our previous constructions in the
semisimple case.

% \begin{lemma}
% Suppose $Y(m,n)\in Bilex((M\otimes_C N)^{op},\Vect_k)$ for all $(m,n)\in M\otimes_C N$. Denote by $\tilde{Y}:M\otimes_C N\hookrightarrow Bilex((M\otimes_C N)^{op},\Vect_k)$ the resulting functor. Then we have a natural isomorphism $\tilde{Y}\simeq K$. \end{lemma}
% \begin{proof}
%  Note that we have $i\tilde{Y}=Y$. Since $i$ is fully faithful, the counit $Ri\Rightarrow \id_{Bilex((M\otimes_C N)^{op}),\Vect_k)}$ of the adjunction $R\dashv i$ is an isomorphism. Finally we have $K=Ry=Ri\tilde{Y}\simeq\tilde{Y}.$\end{proof}
%

\begin{lemma}\label{fin_boxsk}

  \noindent Let $C$ be a monoidal category, and let $M_C$ and $_{C}N$ be
  $k$-linear semisimple $C$-module categories. Then $M\boxtimes_C^{sk}N\simeq
  Fin(M\otimes_C N)$.
\end{lemma}

\begin{proof}
  In this case we have $Bilex((M\otimes_C N)^{op},\Vect_k)=Fun((M\otimes_C
  N)^{op},\Vect_k)$. So the functor $K$ in definition \ref{sk_nonsemisimple}
  is simply the Yoneda embedding $M\otimes_C N\hookrightarrow Fun((M\otimes_C
  N)^{op},\Vect_k)$ and so $M\boxtimes_C^{sk} N$ is simply the finite
  cocompletion of $M\otimes_C N$.
\end{proof}

\begin{corollary}\label{fin_kelly}
  Let $C$ be a monoidal category, and let $M_C$ and $_{C}N$ be $k$-linear
  semisimple $C$-module categories. Then $M\otimes_CN\to Fin(M\otimes_C N)$
  presents $Fin(M\otimes_C N)$ as the balanced Kelly tensor product of $M$ and
  $N$ over $C$.
\end{corollary}

\begin{remark}
  In lemma \ref{fin_boxsk} and corollary \label{fin_kelly}, semisimplicity of
  $M$ and $N$ can be replaced by the weaker condition that every short exact
  sequence splits in $M$ and $N$.
\end{remark}


% \begin{remark}
% When $C$ is a monoidal category and $M_C,_{C}N$ are $k$-linear semisimple $C$-module categories, we have $$M\boxtimes^K_{C}N\simeq M\boxtimes_C^{sk}N\simeq Fin(M\otimes_C N).$$
% 
% \noindent When, additionally, $Fin(M\otimes_C N)$ is semisimple, we have $$M\boxtimes^K_{C}N\simeq M\boxtimes_C^{sk}N\simeq Fin(M\otimes_C N)\simeq Cau(M\otimes_C N)=sk(M,C,N).$$
% 
% \noindent When $C$ is a finite tensor category and $M,N$ are $k$-linear finite semisimple $C$-module categories, we have $$M\boxtimes^{sk}_C N\simeq Fin(M\otimes_C N)\simeq M\boxtimes_C N.$$
% 
% \noindent When $C$ is a finite tensor category and $M,N, M\boxtimes_C N$ are all $k$-linear finite semisimple, we have $$M\boxtimes_C^{sk}N\simeq Fin(M\otimes_C N)\simeq Cau(M\otimes_C N)=sk(M,C,N)\simeq M\boxtimes_C N.$$ See remark \ref{semisimple_douglas/dualizable-tensor-categories} for conditions on $M,N$ and $C$ under which $M\boxtimes_C N$ is finite semisimple.
% \end{remark}

% This subsubsection `Categorical Results of G. M. Kelly' is maintained by Manuel.
\subsubsection{Review of some basic concepts of enriched category theory} \label{review_kelly}

Here we review the concepts from \cite{kelly/basic-concepts-enriched} that are
needed in \ref{section-nonsemisimple}.

First we note that the results in \cite{kelly/basic-concepts-enriched} can be
applied to categories enriched over very general kinds of symmetric monoidal
categories $V$. To be precise, the results we need apply when $V$ is a locally
bounded, closed symmetric monoidal category. We won't need to define these
terms, as we won't prove the results from
\cite{kelly/basic-concepts-enriched}. All that needs to be said is that
$\Vect_k$ satisfies these properties, and we will always take $V=\Vect_k$.

\begin{definition}

  \noindent Consider a functor $F:I\to A$. A cone over $F$ is an object $a\in
  A$ together with a natural transformation $a\Rightarrow F$ (where $a$
  denotes the constant functor $a:I\to A$). A cocone under $F$ is an object
  $b\in A$ together with a natural transformation $F\Rightarrow b$.
\end{definition}

\begin{definition}

  \noindent Let $A$ be a category. A cone in $A$ consists of a category $I$, a
  functor $F:I\to A$ and a cone over $F$. A cocone in $A$ consists of a
  category $I$, a functor $F:I\to A$ and a cocone under $F$.
\end{definition}

We often denote a cocone in $A$ simply by writing $a_i\to a$ which denotes
$a_i=F(i)$ and the component $a_i\to a$ of the natural transformation
corresponding to $i\in I$.


\begin{definition}

  \noindent A small sketch is a pair $(A^{op},\Phi)$ where $A$ is a small
  category and $\Phi$ is a small set of cocones in $A$. We often refer to a
  cocone in $\Phi$ as a $\Phi$-cocone in $A$.
\end{definition}

\begin{remark}
  Note that in \cite{kelly/basic-concepts-enriched} $\Phi$ is allowed to be a
  set of cylinders. Cocones are particular kinds of cylinders and we won't
  need the extra generality.
\end{remark}

\begin{definition}
  \noindent Given a small sketch $(A^{op},\Phi)$ and a category $B$, a
  $\Phi$-comodel in $B$ is a functor $A\to B$ sending the $\Phi$-cocones in
  $A$ to colimit cocones in $B$. We denote by $\Phi-Com(A,B)\subset Fun(A,B)$
  the full subcategory whose objects are the $\Phi$-comodels. A $\Phi$-algebra
  is a functor $A^{op}\to \Vect_k$ sending all $\Phi$-cocones in $A$ (which
  become cones in $A^{op}$) to limit cones in $\Vect_k$. We denote by
  $\Phi-Alg\subset Fun(A^{op},\Vect_k)$ the full subcategory whose objects are
  the $\Phi-algebras$.
\end{definition}


\begin{definition}

  \noindent Given a small sketch $(A^{op},\Phi)$, we define $\Theta_{\Phi}$ to
  be the set of all morphisms in $Fun(A^{op},\Vect_k)$ of the form $colim_i
  Y(a_i)\to Y(a)$ where $a_i\to a$ is a $\Phi$-cocone in $A$ and $Y$ is the
  Yoneda embedding.
\end{definition}

\begin{definition}

  \noindent Given a category $P$, we say that an object $p\in P$ is orthogonal
  to a morphism $\theta:m\to n$ in $P$ if $\theta^*:Hom_P(n,p)\to Hom_P(m,p)$
  is an isomorphism.
\end{definition}

\begin{definition}
  \noindent A full subcategory $C\subset P$ is called reflective if the fully
  faithful inclusion functor $C\hookrightarrow P$ has a left adjoint $R:P\to
  C$. The left adjoint is called the reflector.
\end{definition}

\begin{theorem}\cite[theorem 6.5]{kelly/basic-concepts-enriched}\label{reflective}

  \noindent Let $A$ be a small category and let $\Theta$ be a set of morphisms
  in $Fun(A^{op},\Vect_k)$. Let $C_{\Theta}\subset Fun(A^{op},\Vect_k)$ be the
  full subcategory whose objects are those orthogonal to every
  $\theta\in\Theta$. Then $C_{\Theta}$ is a reflective subcategory.
\end{theorem}

\begin{remark}
  The category $C_{\Theta}$ is denoted $\Theta$-Alg in
  \cite{kelly/basic-concepts-enriched}.
\end{remark}


\begin{lemma}\cite[theorem 6.11]{kelly/basic-concepts-enriched}
  Let $(A^{op},\Phi)$ be a small sketch. Then the reflective subcategory
  $C_{\Theta_{\Phi}}\subset Fun(A^{op},\Vect_k)$ is the category $\Phi$-Alg.
\end{lemma}

\begin{proof}
  An object $\phi \in Fun(A^{op},\Vect_k)$ is orthogonal to a morphism
  $colim_i Y(a_i)\to Y(a)$ in $\Theta_{\Phi}$
  if $$Fun(A^{op},\Vect_k)(Y(a),\phi)\to Fun(A^{op},\Vect_k)(colim_i
  Y(a_i),\phi)$$ is an isomorphism. That is to say $\phi(a)\to lim_i
  \phi(a_i)$ is an isomorphism, i.e. $\phi(a)\to \phi(a_i)$ is a limit cone.
  So the objects of $Fun(A^{op},\Vect_k)$ which are orthogonal to every
  $\theta\in \Theta_{\Phi}$ are exactly the $\Phi$-algebras.
\end{proof}

\begin{definition}

  \noindent Given a small sketch $(A^{op},\Phi)$ we denote by $C_{\Phi}$ the
  category $\Phi-Alg$.
\end{definition}


So we obtain an adjunction
\[\begin{tikzcd}
            {C_{\Phi}} & {Fun(A^{op},\Vect_k)}
            \arrow[""{name=0, anchor=center, inner sep=0}, "i_{\Phi}"', shift right=3, hook, from=1-1, to=1-2]
            \arrow[""{name=1, anchor=center, inner sep=0}, "R_{\Phi}"', shift right=3, from=1-2, to=1-1]
            \arrow["\dashv"{anchor=center, rotate=-90}, draw=none, from=1, to=0]
\end{tikzcd}.
\]

\begin{definition}

  \noindent Given a small sketch $(A^{op},\Phi)$, and denoting by $Y:A\to
  Fun(A^{op},\Vect_k)$ the Yoneda embedding, we define $K_{\Phi}:A\to
  C_{\Phi}$ as the composite functor
  \[
    \begin{tikzcd} A &
    {Fun(A^{op},\Vect_k)} & {C_{\Phi}} \arrow["Y", hook, from=1-1, to=1-2]
    \arrow["R_{\Phi}", from=1-2, to=1-3]
    \end{tikzcd}.
  \]
\end{definition}

\begin{lemma}

  \noindent Given a small sketch $(A^{op},\Phi)$, the functor $K_{\Phi}:A\to
  C_{\Phi}$ is a $\Phi$-comodel, i.e. it sends $\Phi$-cocones in $A$ to
  colimit cocones in $C_{\Phi}$.
\end{lemma}

\begin{proof}
  This is (a special case of) statement (5.51) in
  \cite{kelly/basic-concepts-enriched}. Since the proof is short, we can
  record it here. We must show that, given any object $\phi\in
  C_{\Phi}=\Phi-Alg$, the functor $C_{\Phi}(R_{\Phi}Y-,\phi):A^{op}\to\Vect_k$
  is a $\Phi$-algebra, i.e. sends $\Phi$-cocones in $A$ to limit cones in
  $\Vect_k$. But $C_{\Phi}(R_{\Phi}Y-,\phi)\simeq
  Fun(A^{op},\Vect_k)(Y-,\phi)\simeq\phi$.
\end{proof}


\begin{remark}

  \noindent Since $C_{\Phi}\subset Fun(A^{op},\Vect_k)$ is a reflective
  subcategory and $Fun(A^{op},\Vect_k)$ is cocomplete, $C_{\Phi}$ is also
  cocomplete.
\end{remark}

\begin{definition}
  \noindent A diagram type is just a small category. Given a diagram type $I$,
  a diagram of type $I$ in $D$ is a functor $I\to D$. Let $\mathcal{F}$ be a
  set of diagram types. An $\mathcal{F}$-colimit in a category $D$ is a
  colimit of a diagram of type $I$ where $I\in \mathcal{F}$. A functor $D\to
  B$ is $\mathcal{F}$-cocontinuous if it sends $\mathcal{F}$-colimits in $D$
  to $\mathcal{F}$-colimits in $B$. We denote by
  $\mathcal{F}-Cocts(D,B)\subset Fun(D,B)$ the full subcategory whose objects
  are the $\mathcal{F}$-cocontinuous functors.
\end{definition}

\begin{remark}
  In \cite{kelly/basic-concepts-enriched} a more general notion of indexing
  type is used. Here we only need the more restricted notion of diagram type
  defined above.
\end{remark}

\begin{definition}

  \noindent A subcategory $D\subset C$ is replete if given $d\in D$ and an
  isomorphism $f:d\to c$ in $C$, then $c$ and $f$ are also in $D$.
\end{definition}


\begin{definition}

  \noindent Let $\mathcal{F}$ be a set of diagram types and suppose $C$ is
  $\mathcal{F}$-cocomplete. Let $K\subset C$ be a full subcategory. The
  closure of $K$ under $\mathcal{F}$-colimits in $C$ is the smallest replete
  full subcategory of $C$ containing $K$ and closed under
  $\mathcal{F}$-colimits.
\end{definition}

\begin{definition}

  \noindent The type of a cocone $a\Rightarrow F$ is the type of the diagram
  $F$.
\end{definition}

\begin{definition}

  \noindent Let $\mathcal{F}$ be a small set of diagram types. A small
  $\mathcal{F}$-sketch is a small sketch $(A^{op},\Phi)$ where the type of
  every $\Phi$-cocone is in $\mathcal{F}$.
\end{definition}


\begin{definition}

  \noindent Let $\mathcal{F}$ be a small set of diagram types and let
  $(A^{op},\Phi)$ be a small $\mathcal{F}$-sketch. Define
  $D_{\Phi,\mathcal{F}}$ as the closure of the full image of $K_{\Phi}$ in
  $C_{\Phi}$ under $\mathcal{F}$-colimits. Notice that $K_{\Phi}:A\to
  C_{\Phi}$ factors through $D_{\Phi,\mathcal{F}}\hookrightarrow C_{\Phi}$, so
  we obtain a functor $Z_{\Phi,\mathcal{F}}:A\to D_{\Phi,\mathcal{F}}$.
\end{definition}

\begin{lemma}
  Let $\mathcal{F}$ be a small set of diagram types and let $(A^{op},\Phi)$ be
  a small $\mathcal{F}$-sketch. Then the functor $Z_{\Phi,\mathcal{F}}:A\to
  D_{\Phi}$ is a $\Phi$-comodel, i.e. it sends $\Phi$-cocones in $A$ to
  colimit cocones in $D_{\Phi,\mathcal{F}}$.
\end{lemma}

\begin{proof}
  This follows from the fact that $K_{\Phi}$ is a $\Phi$-comodel and the fully
  faithful inclusion $D_{\Phi,\mathcal{F}}\hookrightarrow C_{\Phi}$ reflects
  colimits.
\end{proof}

\begin{lemma}
  Let $\mathcal{F}$ be a small set of diagram types and let $(A^{op},\Phi)$ be
  a small $\mathcal{F}$-sketch. Given an $\mathcal{F}$-cocontinuous functor
  $H:D_{\Phi,\mathcal{F}}\to B$, the composite $
  \begin{tikzcd} A &
    D_{\Phi,\mathcal{F}} & B \arrow["Z_{\Phi,\mathcal{F}}", from=1-1, to=1-2]
    \arrow["H", from=1-2, to=1-3]
  \end{tikzcd}$
  is a $\Phi$-comodel.
\end{lemma}

 \begin{proof}
   Recall that $Z_{\Phi,\mathcal{F}}$ is a $\Phi$-comodel, i.e. it sends
   $\Phi$-cocones in $A$ to colimit cocones in $D_{\Phi,\mathcal{F}}$. Since
   $\mathcal{F}$ contains the type of every $\Phi$-cocone,
   $Z_{\Phi,\mathcal{F}}$ sends $\Phi$-cocones in $A$ to $\mathcal{F}$-colimit
   cocones in $D_{\Phi,\mathcal{F}}$. But $H$ sends $\mathcal{F}$-colimits in
   $D_{\Phi,\mathcal{F}}$ to $\mathcal{F}$-colimits in $B$, so finally the
   composite sends $\Phi$-cocones in $A$ to colimit cocones in $B$.
 \end{proof}

This means that precomposition with $Z_{\Phi,\mathcal{F}}$ defines a functor
$Z_{\Phi,\mathcal{F}}^*:\mathcal{F}-Cocts(D_{\Phi,\mathcal{F}},B)\to\Phi-Com(A,B)$
for any category $B$.

\begin{theorem}\cite[theorem 6.23]{kelly/basic-concepts-enriched}\label{f_theory_small_sketch}

  \noindent Let $\mathcal{F}$ be a small set of diagram types and let
  $(A^{op},\Phi)$ be a small $\mathcal{F}$-sketch. Let $B$ be an
  $\mathcal{F}$-cocomplete category.
  Then $$Z_{\Phi,\mathcal{F}}^*:\mathcal{F}-Cocts(D_{\Phi,\mathcal{F}},B)\to\Phi-Com(A,B)$$
  is an equivalence.
\end{theorem}

In section \ref{section-nonsemisimple}, we use the procedure described here to
define $M\boxtimes_C^{sk}N$. We have $A=M\otimes_C N$, which is a small
category whenever $M,N,C$ are small. We take $\mathcal{F}$ to be a set of
finite categories containing exactly one representative of each isomorphism
class (it is a small set). So $\mathcal{F}$-colimits are just finite colimits.
We take $\Phi$ to be the set of all cocones of the kinds $m_i\otimes n\to
m\otimes n$ and $m\otimes n_i\to m\otimes n$ where $m_i\to m$ is any finite
colimit cocone in $M$ (with type in $\mathcal{F}$) and $n_i\to n$ is any
finite colimit cocone in $N$ (with type in $\mathcal{F}$). The fact that we
require the types to be in the small set $\mathcal{F}$ and the fact that the
categories $M$, $N$ are small together imply that $\Phi$ is a small set. Then
one obtains $C_{\Phi}=\Phi-Alg=Bilex((M\otimes_C N)^{op},\Vect_k)$ and
$K_{\Phi}:M\otimes_C N\to Bilex((M\otimes_C N)^{op},\Vect_k)$ is the functor
$K$ of definition \ref{def_K}. Finally, the category $D_{\Phi,\mathcal{F}}$
obtained as the closure of the full image of $K_{\Phi}$ in $C_{\Phi}$ under
finite colimits is exactly $M\boxtimes_C^{sk}N$ (definition
\ref{sk_nonsemisimple}). Now $\mathcal{F}$-cocomplete is synonymous with
finitely cocomplete and $\mathcal{F}$-cocontinuous means right exact. So the
$\mathcal{F}$-cocontinuous functors $M\boxtimes_C^{sk}N\to L$ are exactly the
linear functors. Moreover, $\Phi$-comodels $M\otimes_C N\to L$ are exactly the
bilinear functors. So we obtain proposition \ref{univ_sk_nonsemisimple}.

\subsubsection{Questions}

We record here two questions which remain to be explored. 

\begin{question}Suppose $C$ is a monoidal category and $M_C$, $_{C}N$ are module categories (all enriched over $\Vect_k$). Under what conditions is the functor $M\otimes N\to M\otimes_C N$ right exact in each variable?\end{question}

This is true for example in the unbalanced case, i.e. when $C=\Vect_k$. When this holds, one can show that the Yoneda embedding $M\otimes_C N\hookrightarrow Fun((M\otimes_C N)^{op},\Vect_k)$ factors through the inclusion $Bilex((M\otimes_C N)^{op},\Vect_k)\hookrightarrow Fun((M\otimes_C N)^{op},\Vect_k)$, yielding a functor $K:M\otimes_C N\hookrightarrow Bilex((M\otimes_C N)^{op},\Vect_k)$, so we can simply define  $M\boxtimes^{sk}_C N$ as the closure of the image of this functor under finite colimits. This means we don't need to use the reflection $R:Fun((M\otimes_C N)^{op},\Vect_k)\to Bilex((M\otimes_C N)^{op},\Vect_k)$. 

We note that in \ref{section-nonsemisimple} we have not made the assumption that the action of $C$ on $M$ and $N$ is right exact in each variable. It seems like this hypothesis should play a role in answering the question above.

\begin{question}
If the balanced tensor product of $k$-linear categories $M\boxtimes_C N$ exists, then must it agree with the Kelly tensor product $M\boxtimes_C^{K}N$ (and therefore with $M\boxtimes^{sk}_C N$)?  
\end{question}

We know this is true when $C$ is a finite tensor category and $M$, $N$ are finite $k$-linear $C$-module categories (see \cite[Remark 3.4]{douglas/balanced-product}.) This also holds in the unbalanced case, i.e. when $C=\Vect_k$ (see \cite[Theorem 18]{lopezfranco/tensor-products}). 

%% % NOTE Remove this whole section?
%% %% \subsection{Other}

%% \begin{notation} (${}^{m'}_{n'}I^{m}_{n}$)

%%   \noindent Denote the equivalence class of the vector
%%   $(\phi \otimes \pi \otimes \psi) \in V((m,m'),(n,n'))$ by
%%   \[\III{m'}{n'}{m}{n}{\phi}{\psi}{c}{\pi}{c'}.\]
%%   When $m' = m$ and $n' = n$, we omit the primed symbols and write
%%   \[
%%     \III{}{}{m}{n}{\phi}{\psi}{c}{\pi}{c'} :=
%%     \III{m}{n}{m}{n}{\phi}{\psi}{c}{\pi}{c'}.
%%   \]
%%   When $\pi = 1_{c}$ (thus $c' = c$), we abbreviate it further to
%%   \[
%%     \II{m'}{n'}{m}{n}{\phi}{\psi}{c} :=
%%     \III{m'}{n'}{m}{n}{\phi}{\psi}{c}{1_{c}}{c},
%%   \]
%%   and
%%   \[
%%     \II{}{}{m}{n}{\phi}{\psi}{c} :=
%%     \III{m}{n}{m}{n}{\phi}{\psi}{c}{1_{c}}{c}.
%%   \]
%% \end{notation}

%% \begin{remark}\label{remark/skein-nature-of-the-notation-I} (Skein Nature of the Notation ${}^{m'}_{n'}I^{m}_{n}$)

%%   \noindent Informally yet instructively, it is helpful to view
%%   $\III{m'}{n'}{m}{n}{\phi}{\psi}{c}{\pi}{c'}$ as a skein flowing from the right to
%%   the left, starting from $m \boxtimes n$ to $m' \boxtimes n'$, passing through $\phi \boxtimes \psi$;
%%   during the process, the upper strain emits a particle $c$ at $\phi$, which
%%   transforms to via $\pi$ to $\overline{c}$, and hits the lower strain at
%%   $\psi$. Hence under the defined composition rule (see the equation for
%%   $\phi'' \otimes \pi'' \otimes \psi''$ above), we have
%%   \[
%%     \III{m''}{n''}{m'}{n'}{\phi'}{\psi'}{c'}{\pi'}{\overline{c}'} \circ
%%     \III{m'}{n'}{m}{n}{\phi}{\psi}{c}{\pi}{\overline{c}} =
%%     \III{m''}{n''}{m}{n}{\phi''}{\psi'}{c' \otimes c}{\pi''}{\overline{c}' \otimes \overline{c}}.
%%   \]
%% \end{remark}

%% \begin{remark}\label{remark/hom-space-reduction} (Hom Space Reduction)
%%   \noindent By the relations in the definition, the transformation
%%   \[
%%     \pi = 1_{\overline{c}} \circ \pi = \pi \circ 1_{c}
%%   \]
%%   can be absorbed into each of the strands, so
%%   \[
%%     \II{m'}{n'}{m}{n}{\phi}{{}_{\pi}\psi}{c} =
%%     \III{m'}{n'}{m}{n}{\phi}{{}_{\pi}\psi}{c}{1_{c}}{c} =
%%     \III{m'}{n'}{m}{n}{\phi}{\psi}{c}{\pi}{c'} =
%%     \III{m'}{n'}{m}{n}{\phi_{\pi}}{\psi}{c'}{1_{c'}}{c'} =
%%     \II{m'}{n'}{m}{n}{\phi_{\pi}}{\psi}{c'}.
%%   \]

%%   \noindent Furthermore, if $c \simeq \oplus_{i=1}^{l} c_{i}$, then
%%   \[
%%     \II{m'}{n'}{m}{n}{\phi}{\psi}{c} = \II{m'}{n'}{m}{n}{\phi}{\psi}{\oplus_{i=1}^{l} c_{i}} = \sum_{j=1}^{l} \III{m'}{n'}{m}{n}{\phi_{j}}{\psi}{c_{j}}{\iota_{j}}{\oplus_{i=1}^{l} c_{i}} =
%%     \sum_{j=1}^{l}\II{m'}{n'}{m}{n}{\phi_{j}}{\psi_{j}}{c_{j}},
%%   \]
%%   where $\iota_{j}$ denotes the $j$-th embedding map from $c_{j}$ to
%%   $\oplus_{i=1}^{l}c_{i}$, and the $\phi_{j}, \psi_{j}$'s denote the $j$-th
%%   projection of $\phi, \psi$ respectively. In particular, when
%%   $c \simeq x^{\oplus l}$, the this gives a reduction from
%%   \[
%%     Hom_{M}(m, m' \rhd x^{\oplus l}) \otimes Hom_{C}(x^{\oplus l}, x^{\oplus l}) \otimes Hom_{N} (x^{\oplus l} \lhd n, n')
%%   \]
%%   to
%%   \[
%%     Hom_{M}(m, m' \rhd x) \otimes Hom_{C}(x, x) \otimes Hom_{N} (x \lhd n, n').
%%   \]
%%   It is helpful to regard the result as the ``inner product'' of $\phi$ and $\psi$.
%% \end{remark}

%% \noindent The hom vector spaces are finite dimensional. An explicit basis is
%% constructed in \ref{proposition/basis-theorem}.

%% \begin{definition} \label{definition/karoubi-completion} (Karoubi Completion)

%%   \noindent Let $C$ be a category. \quad The Karoubi completion $Kar(C)$ of
%%   $C$ is defined to be the category with
%%   \[
%%     Obj(Kar(C)) = \{(c, f) \,|\, c \in Obj(C), f \in End_{C}(c), f = f^{2}\}
%%   \] and
%%   \[
%%     Mor_{Kar(C)}((c,f), (c', f')) = \{\overline{f} \in Hom_{C}(c,c') \,|\, \overline{f}f = \overline{f} = f'\overline{f}\},
%%   \]
%%   with the obvious composition rule.
%% \end{definition}

%% \begin{remark} \label{remark/karoubi-retract} (Karoubi Retracts)

%%   \noindent It is straightforward to check that every object $(c,f)$ in
%%   $Kar(C)$ is a retract of the original object $c = (c, 1_{c})$. So we have
%%   \[
%%     (c, f) \xrightarrow{\iota} c = (c, 1_{c}) \xrightarrow{\pi} (c, f).
%%   \]
%% \end{remark}

%% \begin{definition}\label{definition/skein-category} (Skein Category)

%%   \noindent Let $C$ be a tensor category. Let $M_{C}$ and $_{C}N$ be module
%%   categories. \quad Define the skein category $sk(M,C,N)$ to the Karoubi
%%   completion of its pre-skein category
%%   \[
%%     sk(M,C,N) := Kar(p.sk(M,C,N)).
%%   \]
%%   \noindent Thus, a typical object of the skein category $sk(M,C,N)$ is an
%%   idempotent matrix of skeins (the typical object is an idempotent
%%   endomorphism of a direct sum $\oplus_{i} (m_{i} \boxtimes n_{i})$, and an
%%   endomorphism of such a direct sum is just a matrix whose entries are maps
%%   $(m_{i} \boxtimes n_{i}) \to (m_{j} \boxtimes n_{j})$, i.e. skeins like
%%   $\II{m_{j}}{n_{j}}{m_{i}}{n_{i}}{\phi}{\psi}{c}$). The skein category is
%%   obviously enriched over $\Vect_{\mathbb{C}}$.

%%   More generally, for any $n \in \mathbb{N}$, let
%%   $C_{0}, C_{1}, \ldots, C_{n}$ be tensor categories. Let
%%   ${}_{C_{0}}M^{1}_{C_{1}}, \, {}_{C_{1}}M^{2}_{C_{2}}, \ldots, {}_{C_{n{\text -}1}}M^{n}_{C_{n}}$
%%   be bimodule categories. One can in a similar way define the
%%   $C_{0}{\text -}C_{n}$ bimodule category
%%   \[
%%     sk(M^{1}, C_{1}, M^{2}, C_{2}, M^{3}, \ldots, C_{n{\text -}1}, M^{n}).
%%   \]
%%   \begin{center}
%%     \includesvg[width=12cm]{drawing-3}
%%   \end{center}

%% \end{definition}

%% \begin{definition} \label{definition/induced-functor-on-skein-category} (Induced Functor on Skein Category)

%%   \noindent Assume the notation in \ref{definition/skein-category}. Let
%%   $1 \leq i \leq n$, and let $F: M^{i} \to M'^{i}$ be a
%%   $C_{i{\text -}1}{\text -}C_{i}$ bimodule category functor. \quad Then $F$
%%   naturally induces a linear functor between the skein categories
%%   \[
%%     sk(M^{1}, C_{1}, M^{2}, C_{2}, M^{3}, \ldots M^{i} \ldots, C_{n{\text -}1}, M^{n})
%%     \xrightarrow{F}
%%     sk(M^{1}, C_{1}, M^{2}, C_{2}, M^{3}, \ldots M'^{i} \ldots, C_{n{\text -}1}, M^{n}).
%%   \]
%%   For example, if $n=2$, $M^{1} = M$, $M^{2} = N$, $C^{1} = C$, $N \xrightarrow{F} N'$
%%   (with the left $C$-module structure given by $\alpha$), then we have an
%%   induced linear functor $sk(M,C,N) \to sk(M,C,N')$ sending the objects and morphisms via the map:
%%   \[
%%     \II{m'}{n'}{m}{n}{\phi}{\psi}{c}
%%     \quad \mapsto \quad
%%     \II{m'}{F(n')}{m}{F(n)}{\phi}{F(\psi) \circ \alpha}{c}.
%%   \]
%% \end{definition}

%% \noindent The main result of this paper is to show that the skein category
%% $sk(M,C,N)$ is equivalent to $M \boxtimes_{C} N$, and that the induced functor $M \boxtimes_{C} N \to M \boxtimes_{C} N'$
%% coincides with the one in \ref{definition/induced-functor-on-skein-category}
%% (proven in \ref{lemma/main-lemma}, \ref{theorem/main-theorem}). A necessary
%% ingredient is the canonical map $\boxtimes_{C}$ given in the defining universal
%% property.

%% \begin{definition}\label{definition/canonical-map} (Canonical Map $\boxtimes_{C}$)

%%   \noindent Let $C$ be a tensor category, and $M_{C}, _{C}N$ be module
%%   categories. \quad Define the functor
%%   \[
%%     M \times N \xrightarrow{\boxtimes_{C}} sk(M,C,N)
%%   \]
%%   to send the object $(m,n)$ to the object $\II{m}{n}{m}{n}{1_{m}}{1_{n}}{1_{1}}$, and the morphism
%%   \[
%%     (m,n) \xrightarrow{(\phi, \psi)} (m', n')
%%   \]
%%   to the morphism $\II{m'}{n'}{m}{n}{\phi}{\psi}{1_{1}}$.
%% \end{definition}

%% \noindent From the main theorem, we must have $sk(M,C,C) \simeq M$. To quickly
%% convince the reader that the main theorem is true before it is proven, we
%% provide another direct proof for this equivalence in the appendix (cf
%% \ref{proposition/degenerated-main-theorem}).

%% \hfill\break
%% \noindent We prove another lemma that will be useful later.

%% \begin{lemma}\label{lemma/I-provides-subobject} (Objects are Retracts)

%%   \noindent Let $C$ be a tensor category, and $M_{C}, _{C}N$ be module
%%   categories. \quad Then any typical object $\II{}{}{m}{n}{\phi}{\psi}{c}$ in
%%   $sk(M,C,N)$ is a retract (in particular, a subobject) of the canonical
%%   object $\boxtimes_{C}((m,n)) = \II{}{}{m}{n}{1_{m}}{1_{n}}{1}$.
%% \end{lemma}
%% \begin{proof}
%%   This directly follows from \ref{remark/karoubi-retract}.
%% \end{proof}



%% \subsection{Proof of the Main Equivalence Theorem}\label{section/proof-of-equivalence}

%% Unless specified otherwise, throughout this section, let $C, D, E$ be tensor
%% categories, let $M_{C}$ and $_{C}N$ be module categories, and let $L$ be a
%% linear category. We prove our main theorem (\ref{theorem/main-theorem}) in this section, justifying
%% that the skein construction $sk(M,C,N)$ is isomorphic to $M \boxtimes_{C} N$, and the
%% obvious generalization to the case of more bimodule categories.

%% \begin{lemma}\label{lemma/construction-of-theta} (Construction of $\Theta$)

%%   \noindent There exists a linear functor
%%   \[
%%     \Theta: Fun(sk(M,C,N), L) \to Fun^{C{\text -}bal}(M \times N, L).
%%   \]
%% \end{lemma}

%% \noindent We construct $\Theta$ explicitly in the proof.

%% \begin{proof}
%%   \noindent (Object) Let $G$ be an object of the domain. Define $\Theta(G)$ to
%%   be $F := G \circ \boxtimes_{C} \in Fun(M \times N, L)$. We shall provide the
%%   balanced structure $\alpha$ for $F$, so that $(F, \alpha)$ is $C$-balanced.
%%   We need to provide the $C$-balanced data for $F$:
%%   \[
%%     \alpha_{m,c,n}: F(m \lhd c, n) =
%%     G(\II{}{}{mc}{n}{1}{1}{1})
%%     \xrightarrow{\sim} G(\II{}{}{m}{cn}{1}{1}{1})
%%     = F(m, c \rhd n),
%%   \]
%%   which is clearly satisfied by $G(\II{m}{cn}{mc}{n}{1}{1}{c}).$ So defined
%%   $\alpha$ is clearly natural, and it is invertible by using the right dual of $c$.

%%   \noindent (Morphism) Let $G \xrightarrow{\eta} G'$ be a morphism in
%%   $Fun(sk(M,C,N), L)$. Its image under $\Theta$ is simply the horizontal
%%   composition $\eta \star (1_{\boxtimes_{C}})$. The remaining commutativity to be checked
%%   % cf p11 of Jin's note 20240906-110000
%%   is a direct consequence of $\eta$'s naturality.
%% \end{proof}

%% \begin{lemma}\label{lemma/theta-is-faithful} ($\Theta$ is faithful)

%%   \noindent The linear functor $\Theta$ (cf. \ref{lemma/construction-of-theta}) is faithful.
%% \end{lemma}

%% \begin{proof}
%%   (We use the same notation found in this section.) This amounts to showing
%%   that the following map is an injective linear map:
%%   \[
%%     (G \xrightarrow{\eta} G') \mapsto (F \xrightarrow{\Theta(\eta) = \eta \star (1_{\boxtimes_{C}})} F').
%%   \]
%%   It is clearly linear. For injectivity, we notice that whenever we have
%%   linear functors
%%   \[
%%     X \xrightarrow{f} Y,\quad Y \xrightarrow{g, g'} Z,
%%   \]
%%   and a linear natural transformation $\eta: g \to g'$, then the map $(\eta \mapsto \eta \star 1_{f})$ is injective is equivalent to
%%   \[
%%     (\forall x \in Obj(X), \eta_{f(x)} = 0) \Rightarrow (\forall y \in Obj(Y), \eta_{y} = 0).
%%   \]
%%   This holds if $f$ is surjective on objects. However, in our case
%%   $f = \boxtimes_{C}$ is not as strong. Fortunately, clearly it also holds if
%%   $f$ is almost-surjective, in the sense that each $y \in Obj(Y)$ has an
%%   $x \in Obj(X)$ such that $y$ is a retract of $f(X)$. Indeed,
%%   \[
%%     1_{g(y)} = g(1_{y}) = g(\pi_{y} \circ \iota_{y}),
%%   \]
%%   so
%%   \[
%%     (g(y) \xrightarrow{\eta_{y}} g'(y)) = \eta_{y} \circ 1_{g(y)} = \eta_{y} \circ g(\pi \circ \iota) = g(\pi) \circ \eta_{f(x)} \circ g(\iota) = g(\pi) \circ 0 \circ g(\iota) = 0.
%%   \]
%%   This applies to our case by putting $f = \boxtimes_{C}$ and $g = G$, because
%%   each object $\II{}{}{m}{n}{\phi}{\psi}{c}$ is clearly a retract of
%%   $\boxtimes_{C}(m,n) = \II{}{}{m}{n}{1}{1}{1}$. Therefore, $\Theta$ is faithful.
%% \end{proof}

%% \begin{lemma}\label{lemma/theta-is-full} ($\Theta$ is full)

%%   \noindent The linear functor $\Theta$ (cf. \ref{lemma/construction-of-theta}) is full.
%% \end{lemma}

%% \begin{proof}
%%   We need to show that for any $C$-balanced natural transformation
%%   \[
%%     \nu: G \circ \boxtimes_{C} = \Theta(G) \to \Theta(G') = G' \circ \boxtimes_{C}
%%   \]
%%   there is $\mu: G \to G'$ such that $\nu = \mu \star 1_{\boxtimes_{C}}$. The data $\nu$ are the maps
%%   \[
%%     \nu_{(m,n)}: G(\II{}{}{m}{n}{1}{1}{1}) \to G'(\II{}{}{m}{n}{1}{1}{1}).
%%   \]
%%   We only need to extend these data to all objects in $sk(M,C,N)$, i.e. define compatible maps
%%   \[
%%     \nu_{\II{}{}{m}{n}{\phi}{\psi}{c}}: G(\II{}{}{m}{n}{\phi}{\psi}{c}) \to G'(\II{}{}{m}{n}{\phi}{\psi}{c}).
%%   \]
%%   It is straightforward to check that the following works:
%%   \[
%%     \nu_{\II{}{}{m}{n}{\phi}{\psi}{c}}:= G(\II{}{}{m}{n}{\phi}{\psi}{c})
%%     \xrightarrow{G(\iota)}
%%     G(\II{}{}{m}{n}{1}{1}{1})
%%     \xrightarrow{\nu_{(m,n)}}
%%     G'(\II{}{}{m}{n}{1}{1}{1})
%%     \xrightarrow{G'(\pi)}
%%     G'(\II{}{}{m}{n}{\phi}{\psi}{c}),
%%   \]
%%   where $\iota$ and $\pi$ are the inclusion and projection (see
%%   Lemma \ref{lemma/I-provides-subobject}).
%% \end{proof}

%% \noindent To prove that $\Theta$ is essentially surjective, we need the following
%% lemma.

%% \begin{lemma} (Images of skeins) \label{lemma/image-of-skein}
%%   % Discussion: Can we relax conditions? We need to use the basis theorem,
%%   % so it seems that we must require semisimplicity here.
%%   \noindent
%%   Let $(F,\alpha) \in Fun^{C{\text -}bal}(M \times N, L)$. For each morphism
%%   in $sk(M,C,N)$ of the form $\III{m'}{n'}{m}{n}{\phi}{\psi}{c}{\pi}{c'}$,
%%   define the image of it under $\tilde{F}$ to be the composed morphism
%%   \[
%%     F(m,n)
%%     \xrightarrow{F(\phi \times 1)}
%%     F(m' \lhd c, n)
%%     \xrightarrow{F((1 \lhd \pi) \times 1)}
%%     F(m' \lhd c', n)
%%     \xrightarrow[\sim]{\alpha}
%%     F(m', c' \rhd n)
%%     \xrightarrow{F(1 \times \psi)}
%%     F(m',n').
%%   \]
%%   Suppose we have two skeins $\II{m'}{n'}{m}{n}{\phi}{\psi}{c}$ and
%%   $ \II{m'}{n'}{m}{n}{\phi'}{\psi'}{c'}$ as identical morphisms in $sk(M,C,N)$
%%   (recall the definitional relations (\ref{relation/a}) (\ref{relation/b})).

%%   \noindent Then
%%   \begin{equation} \label{eqn/a}
%%     \tilde{F}(\II{m'}{n'}{m}{n}{\phi}{\psi}{c}) = \tilde{F}(\II{m'}{n'}{m}{n}{\phi'}{\psi'}{c'}).
%%   \end{equation}
%%   Moreover, $\tilde{F}$ preserves compositions, i.e.
%%   \begin{equation} \label{eqn/b}
%%     \tilde{F}(\II{m''}{n''}{m'}{n'}{\overline{\phi}}{\overline{\psi}}{\overline{c}} \circ \II{m'}{n'}{m}{n}{\phi}{\psi}{c})
%%     = \tilde{F}(\II{m''}{n''}{m'}{n'}{\overline{\phi}}{\overline{\psi}}{\overline{c}})
%%     \circ
%%     \tilde{F}(\II{m'}{n'}{m}{n}{\phi}{\psi}{c}).
%%   \end{equation}
%% \end{lemma}

%% \begin{proof}
%%   To prove the first statement (\ref{eqn/a}), check the equality against the
%%   definitional relations (\ref{relation/a}) (\ref{relation/b}).

%%   \noindent To prove the second statement (\ref{eqn/b}), note that the
%%   left-hand-side is
%%   $\tilde{F}(\II{m''}{n'}{m}{n}{\overline{\phi} \otimes \phi}{\overline{\psi} \otimes \psi}{\overline{c} \otimes c}),$
%%   which is
%%   \begin{multline*}
%%     F(m,n)
%%     \xrightarrow{F(\phi \times 1)}
%%     F(m' \lhd c, n)
%%     \xrightarrow{F((\overline{\phi} \lhd 1) \times 1)} \\
%%     F((m'' \lhd \overline{c}) \lhd c, n)
%%     \xrightarrow[\sim]{}
%%     F((m'' \lhd (\overline{c} \otimes c)), n)
%%     \xrightarrow[\sim]{\alpha}
%%     F(m'', (\overline{c} \otimes c) \rhd n)
%%     \xrightarrow[\sim]{}
%%     F(m'', \overline{c} \rhd (c \rhd n)) \\
%%     \xrightarrow{F(1 \times (1 \rhd \psi))}
%%     F(m'', \overline{c} \rhd n')
%%     \xrightarrow{F(1 \times \overline{\psi})}
%%     F(m'',n'').
%%   \end{multline*}
%%   On the other hand, the right-hand-side is
%%   \begin{multline*}
%%     F(m,n)
%%     \xrightarrow{F(\phi \times 1)}
%%     F(m' \lhd c, n)
%%     \xrightarrow[\sim]{\alpha} \\
%%     F(m', c \rhd n)
%%     \xrightarrow{F(1 \times \psi)}
%%     F(m', n')
%%     \xrightarrow{F(\overline{\phi} \times 1)}
%%     F(m'' \rhd \overline{c}, n') \\
%%     \xrightarrow[\sim]{\alpha}
%%     F(m'', \overline{c} \lhd n')
%%     \xrightarrow{F(1 \times \overline{\psi})}
%%     F(m'',n'').
%%   \end{multline*}
%%   To prove that they are equal, we can omit their first and their last arrows. Note that the composed arrow in left-hand-side $F(m' \lhd c, n) \to F(m', \overline{c} \rhd (c \rhd n))$ is, by the naturality of $\alpha$, equal to
%%   \[
%%     F(m' \lhd c, n)
%%     \xrightarrow[\sim]{\alpha}
%%     F(m', c \rhd n)
%%     \xrightarrow{F(\overline{\phi} \times 1)}
%%     F(m'' \lhd \overline{c}, c \rhd n)
%%     \xrightarrow[\sim]{\alpha}
%%     F(m'', \overline{c} \rhd (c \rhd n)).
%%   \]
%%   Compose this with
%%   \[
%%     F(m'', \overline{c} \rhd (c \rhd n))
%%     \xrightarrow{F(1 \times (1 \rhd \psi))}
%%     F(m', \overline{c} \rhd n'),
%%   \]
%%   then we get
%%   \[
%%     F(m' \lhd c, n)
%%     \xrightarrow[\sim]{\alpha}
%%     F(m', c \rhd n)
%%     \xrightarrow{F(\overline{\phi} \times 1)}
%%     F(m'' \lhd \overline{c}, c \rhd n)
%%     \xrightarrow{F(1 \times \psi)}
%%     F(m'' \lhd \overline{c}, n'),
%%   \]
%%   which is equal to
%%   \[
%%     F(m' \lhd c, n)
%%     \xrightarrow[\sim]{\alpha}
%%     F(m', c \rhd n)
%%     \xrightarrow{F(\overline{\phi} \times \psi)}
%%     F(m'' \lhd \overline{c}, n').
%%   \]
%%   So both sides are equal.
%% \end{proof}

%% \begin{lemma}\label{lemma/theta-is-essentially-surjective} ($\Theta$ is essentially surjective)

%%   \noindent The linear functor $\Theta$ (cf. \ref{lemma/construction-of-theta}) is essentially surjective.
%% \end{lemma}

%% \begin{proof}
%%   Let $(F, \alpha) \in Fun^{C{\text -}bal}(M \times N, L)$. It suffices to construct
%%   $G \in Fun(sk(M,C,N), L)$ such $\Theta(G) \simeq (F,\alpha)$. Recall that $L$ is an abelian
%%   category (so each $L$-morphism has an image), $F: M \times N \to L$ is a linear
%%   functor, and that
%%   \[
%%     F(m \lhd c, n) \xrightarrow[\sim]{\alpha_{m,c,n}} F(m, c \rhd n).
%%   \]

%%   \noindent ($G$ on objects) Recall the definition and properties of
%%   $\tilde{F}$ in \ref{lemma/image-of-skein}. Define
%%   $G(\II{}{}{m}{n}{\phi}{\psi}{c})$ to be the image (in $L$) of the
%%   $L$-morphism $\tilde{F}(\II{}{}{m}{n}{\phi}{\psi}{c})$. In particular, the
%%   image is a subobject and a quotient of $F(m,n)$.

%%   \noindent ($G$ on morphisms) We use $\tilde{F}$ again. Let
%%   $\II{m'}{n'}{m}{n}{\overline{\phi}}{\overline{\psi}}{\overline{c}}$ be a morphism
%%   from $\II{}{}{m}{n}{\phi}{\psi}{c}$ to $\II{}{}{m'}{n'}{\phi'}{\psi'}{c'}$. Define
%%   $G(\II{m'}{n'}{m}{n}{\overline{\phi}}{\overline{\psi}}{\overline{c}})$ to be the
%%   map induced by
%%   \[
%%     \tilde{F}(\II{m'}{n'}{m}{n}{\overline{\phi}}{\overline{\psi}}{\overline{c}}): F(m,n) \to F(m',n').
%%   \]
%%   To justify this definition, we must show that
%%   \[
%%     ker(
%%     \tilde{F}(\II{}{}{m'}{n'}{\phi'}{\psi'}{c'})
%%     \circ
%%     \tilde{F}(\II{m'}{n'}{m}{n}{\overline{\phi}}{\overline{\psi}}{\overline{c}})
%%     )
%%     \supseteq
%%     ker(\tilde{F}(\II{}{}{m}{n}{\phi}{\psi}{c})).
%%   \]
%%   By lemma \ref{lemma/image-of-skein}, $\tilde{F}$ respects compositions, so
%%   \[
%%     ker(
%%     \tilde{F}(\II{}{}{m'}{n'}{\phi'}{\psi'}{c'})
%%     \circ
%%     \tilde{F}(\II{m'}{n'}{m}{n}{\overline{\phi}}{\overline{\psi}}{\overline{c}})
%%     )
%%     =
%%     ker(
%%     \tilde{F}(\II{}{}{m'}{n'}{\phi'}{\psi'}{c'}
%%     \circ
%%     \II{m'}{n'}{m}{n}{\overline{\phi}}{\overline{\psi}}{\overline{c}})
%%     ).
%%   \]
%%   Then by the definition of $sk(M,N,C)$ and Karoubi completion,
%%   \[
%%     ker(
%%     \tilde{F}(\II{}{}{m'}{n'}{\phi'}{\psi'}{c'}
%%     \circ
%%     \II{m'}{n'}{m}{n}{\overline{\phi}}{\overline{\psi}}{\overline{c}})
%%     )
%%     =
%%     ker(
%%     \tilde{F}(
%%     \II{m'}{n'}{m}{n}{\overline{\phi}}{\overline{\psi}}{\overline{c}}
%%     \circ
%%     \II{}{}{m}{n}{\phi}{\psi}{c}
%%     )
%%     ).
%%   \]
%%   The final step is completed by using the composing property of $\tilde{F}$ again and the fact that
%%   $ker(a \circ b) \supseteq ker(b)$.
%% \end{proof}

%% \begin{lemma} (Main Lemma) \label{lemma/main-lemma}

%%   \noindent The canonical map (\ref{definition/canonical-map})
%%   $\boxtimes_{C}$: $M \times N \to sk(M,C,N)$ satisfies the universal property
%%   in the definition of $M \boxtimes_{C} N$. In particular, we have an
%%   equivalence of categories
%%   \[
%%     sk(M,C,N) \simeq M \boxtimes_{C} N.
%%   \]
%% \end{lemma}

%% \begin{proof}
%%   We only need to show that $sk(M,C,N)$ is a $k$-linear category (in
%%   particular, an abelian category, following the definition in
%%   \cite{douglas/balanced-product}), and that $\boxtimes_{C}$ induces an equivalence of
%%   categories
%%   \[
%%     Fun(sk(M,C,N), L) \xrightarrow[\sim]{\Theta} Fun^{C{\text -}bal}(M \times N, L).
%%   \]
%%   For $k$-linearity, the crux is to show that the skein category is abelian.
%%   We postpone the proof to the appendix \ref{semisimple}. For the second
%%   statement, we constructed $\Theta$ in (\ref{lemma/construction-of-theta}), proved that $\Theta$ is faithfulness in
%%   (\ref{lemma/theta-is-faithful}), is full in (\ref{lemma/theta-is-full}), and is essentially surjective in (\ref{lemma/theta-is-essentially-surjective}).
%% \end{proof}


%% \begin{theorem} (Main Theorem: Skein Construction of Balanced Tensor Product) \label{theorem/main-theorem}

%%   \noindent (1) Let ${}_{C}M_{D}, \, {}_{D}N_{E}$ be, bimodule categories.
%%   \quad Then the canonical map (\ref{definition/canonical-map})
%%   $\boxtimes_{D}$: $M \times N \to sk(M,D,N)$ satisfies the universal property
%%   in the definition of ${}_{C}M_{D} \boxtimes_{D} {}_{D}N_{E}$. In particular,
%%   we have an equivalence of $C{\text -}E$ bimodule categories.
%%   \[
%%     {}_{C}sk(M,D,N)_{E} \simeq {}_{C}M_{D} \boxtimes_{D} {}_{D}N_{E}.
%%   \]

%%   \noindent (2) More generally, for any $n \in \mathbb{N}$, let
%%   $C_{0}, C_{1}, \ldots, C_{n}$ be tensor categories. Let
%%   ${}_{C_{0}}M^{1}_{C_{1}}, \, {}_{C_{1}}M^{2}_{C_{2}}, \ldots, {}_{C_{n{\text -}1}}M^{n}_{C_{n}}, \, $
%%   be bimodule categories. \quad Then we have an equivalence of
%%   $C_{0}{\text -}C_{n}$ bimodule categories.
%%   \[
%%     {}_{C_{0}}sk(M^{1},C_{1},M^{2},C_{2}, \ldots, C_{n{\text -}1}, M^{n})_{C_{n}}
%%     \simeq
%%     {}_{C_{0}}(M^{1}
%%     \boxtimes_{C_{1}}
%%     M^{2}
%%     \boxtimes_{C_{2}}
%%     M^{3}
%%     \ldots
%%     \boxtimes_{C_{n{\text -}1}}
%%     M^{n})_{C_{n}}.
%%   \]
%%   \noindent (3) Moreover, in addition to the previous part, if
%%   $F^{i}: M^{i} \to M'^{i}$ is a $C_{i{\text -}1}{\text -}C_{i}$ bimodule category
%%   functor, then the naturally induced linear functor (cf.
%%   \ref{definition/induced-functor-on-skein-category})
%%   \[
%%     F^{i}:
%%     sk(M^{1}, C_{1}, M^{2}, C_{2}, M^{3}, \ldots M^{i} \ldots, C_{n{\text -}1}, M^{n})
%%     \to
%%     sk(M^{1}, C_{1}, M^{2}, C_{2}, M^{3}, \ldots M'^{i} \ldots, C_{n{\text -}1}, M^{n}),
%%   \]
%%   corresponds to the functor
%%   \[
%%     F^{i}: M^{1} \boxtimes_{C_{1}} M^{2} \boxtimes_{C_{2}} M^{3} \boxtimes_{C_{3}} \ldots M^{i} \ldots \boxtimes_{C_{n{\text -}1}} M^{n} \to M^{1} \boxtimes_{C_{1}} M^{2} \boxtimes_{C_{2}} M^{3} \boxtimes_{C_{3}} \ldots M'^{i} \ldots \boxtimes_{C_{n{\text -}1}} M^{n}.
%%   \]
%%   under the equivalence.
%% \end{theorem}

%% \begin{proof}
%%   The first part is proved by restricting the proof of lemma \ref{lemma/main-lemma} to $C{\text -}E$ bimodule maps.
%%   The second part follows directly from induction and the first part. The third part is obvious.
%% \end{proof}

%% \begin{remark} (Application on the Turaev-Viro model)

%%   \noindent That the induced functor on the skein category coincides with the
%%   algebraic one is the key for computing values of the Turaev-Viro model in
%%   dimensions $(1+1)$ \cite{guu/tv-as-3-functor}.
%% \end{remark}

%% % The following corollary is commented out. The statement itself is not
%% % incorrect, but the proof is wrong (circular). We must have proven that they
%% % are abelian before showing that skein categories are actually balanced
%% % tensor products.
%% % \begin{corollary}\label{corollary/skein-category-is-abelian} (Skein Categories are Abelian)
%% %   \noindent For any $n \in \mathbb{N}$, let $C_{0}, C_{1}, \ldots, C_{n}$ be
%% %   tensor categories. Let
%% %   \[
%% %     {}_{C_{0}}M^{1}_{C_{1}}, \, {}_{C_{1}}M^{2}_{C_{2}}, \ldots, {}_{C_{n{\text -}1}}M^{n}_{C_{n}}, \,
%% %   \]
%% %   be bimodule categories. Then the skein category (cf
%% %   \ref{definition/skein-category})
%% %   \[
%% %     sk(M^{1}, C_{1}, M^{2}, C_{2}, M^{3}, \ldots M^{i} \ldots, C_{n{\text -}1}, M^{n})
%% %   \]
%% %   is an abelian category.
%% % \end{corollary}
%% % \begin{proof}
%% %   The proof for the case $n=2$ follows from the fact that
%% %   $M^{1} \boxtimes_{C_{1}} M^{2} \simeq Z_{C_{1}}(M^{1} \boxtimes M^{2})$ is
%% %   abelian (cf \cite{kirillov/fact-homo-4d-tqft}). The rest follows from
%% %   induction.
%% % \end{proof}


\subsection{Misc. Results}

\noindent This subsection gathers results, proofs, and arguments that would
disrupt the flow of the main text. The content is unstructured, so readers
should approach it as a collection of standalone items.

\noindent From the main theorem, we must have $sk(M,C,C) \simeq M$. To quickly
convince the reader that the main theorem is true before it is proven, we
provide another direct proof for this equivalence:

\begin{proposition} \label{proposition/degenerated-main-theorem}

  \noindent Let $C$ be a tensor category. Let $M_{C}$ be a right $C$-module
  category. \quad Then we have an equivalence of categories $sk(M,C,C) \simeq
  M$.
\end{proposition}

\begin{proof}
  We provide two proofs. The first proof is immediate upon using the main theorem
  \ref{main-theorem-ver-2},
  %% \ref{theorem/main-theorem},
  and the known fact that $M \boxtimes_{C} C
  \simeq M$. The second proof is direct, without using the main theorem:

  Construct the functor $M \to M \otimes C
  \to sk(M,C,C)$, where the first functor is $m\mapsto (m,1)$ and the second one is the
  canonical map (\ref{canonical_map}).
  %TODO: Jin checks this. 
  We contend that this is an
  equivalence of categories. It is straightforward to see that it is indeed
  fully faithful, so it suffices to show that it is essentially surjective. A
  typical object in the codomain $sk(M,C,C)$ is some idempotent skein
  $\II{}{}{m}{c}{\phi}{\psi}{d}$ (without loss of generality, assume $m$ to be
  simple). We contend that this object is isomorphic to $\II{}{}{m \lhd
    c}{1}{\mu_{\phi, \psi}}{1}{1}$, where
  \[
    \mu_{\phi,\psi} :=
    m \lhd c
    \xrightarrow{\phi \lhd c}
    (m \lhd d) \lhd c
    \xrightarrow[\sim]{\alpha}
    m \lhd (d \otimes c)
    \xrightarrow{m \lhd \psi}
    m \lhd c.
  \]
  Indeed, the isomorphism is provided by the following two morphisms
  \begin{align*}
    \II{}{}{m \lhd c}{1}{\mu_{\phi, \psi}}{1}{1}
    \xleftarrow{
    \II{}{}{m \lhd c}{1}{\mu_{\phi, \psi}}{1}{1}
    \, \circ \,
    \II{mc}{1}{m}{c}{u}{n}{c^{\star}}
    \, \circ \,
    \II{}{}{m}{c}{\phi}{\psi}{\overline{c}}}
    \II{}{}{m}{c}{\phi}{\psi}{\overline{c}}
    \\
    \II{}{}{m}{c}{\phi}{\psi}{\overline{c}}
    \xleftarrow{
    \II{}{}{m}{c}{\phi}{\psi}{\overline{c}}
    \, \circ \,
    \II{m}{c}{mc}{1}{1}{1}{c}
    \, \circ \,
    \II{}{}{m \lhd c}{1}{\mu_{\phi, \psi}}{1}{1}
    }
    \II{}{}{m \lhd c}{1}{\mu_{\phi, \psi}}{1}{1},
  \end{align*}
  where $c^{\star}$ denotes the right dual of $c$, $n$ denotes the counit, and
  $u$ denotes the unit for $c$. The two given morphisms may seem unnecessarily
  long, but they have to be so by the definition of of Karoubi completion (in
  which objects must be ``absorbed'' into morphism). The following pictures
  show that they compose to the identities.

  \begin{center}
    \includesvg[width=18cm]{drawing-4}
  \end{center}
\end{proof}

\noindent For computation purpose (cf. one of the applications in the
introduction), one may find the following basis theorem useful.

%% The following proposition is commented out due to
%% https://github.com/jcguu95/skein-construction-of-balanced-tensor-product/commit/99a94ba49171d8a9dc4304806743b0ef12af00ae#r150602988

%% \begin{proposition} (Basis Theorem) \label{proposition/basis-theorem}

%%   \noindent Let $C$ be a finite, semisimple tensor category, and $M_{C},
%%   _{C}N$ be finite, semisimple module categories over $C$. \quad Thus the
%%   vector spaces $Hom_{M}(m, m' \lhd c), Hom_{N}(c \rhd n, n')$ have finite
%%   bases $\beta(m, m' \lhd c), \beta(c \rhd n, n')$ respectively. \quad Then
%%   the hom space $Hom_{p.sk(M,C,N)}(m \boxtimes n, m' \boxtimes n')$ has a
%%   linear basis
%%   \[
%%     \bigsqcup_{c \in Irr(C)} \beta(m, m' \lhd c) \times \beta(c \rhd n, n').
%%   \]
%% \end{proposition}

%% \begin{proof}
%%   This follows immediately from the reductions given in
%%   \ref{remark/hom-space-reduction}.
%%   % TODO (Manuel) This reference to remark-hom-space-reduction is currently
%%   % broken. It is now resided in tmp.3.tex. Jin tried pulling it back, but
%%   % Manuel has introduced different notations from Jin's macros \II and \III.
%%   % Manuel will fix this reference.
%% \end{proof}

%%%%%%%%%%%%%%%%%%%%%%%%%%%%%%%%%%%%%%%%%%%%%%%%%%%%%%%%%%%%%%%%%%%%%%%%%%%%%%%%

% TODO: (Jin) Asks PIMS what we should do before pushing to arxiv.
%   FOLLOW UP (Joanne Jiang from PIMS) https://mail.google.com/mail/u/0/#search/PIMS/FMfcgzQVzFTShwbtcjzZTXcZcdkCTBbX
