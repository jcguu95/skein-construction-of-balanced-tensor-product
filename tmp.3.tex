%% \subsection{Other}

%% \begin{notation} (${}^{m'}_{n'}I^{m}_{n}$)

%%   \noindent Denote the equivalence class of the vector
%%   $(\phi \otimes \pi \otimes \psi) \in V((m,m'),(n,n'))$ by
%%   \[\III{m'}{n'}{m}{n}{\phi}{\psi}{c}{\pi}{c'}.\]
%%   When $m' = m$ and $n' = n$, we omit the primed symbols and write
%%   \[
%%     \III{}{}{m}{n}{\phi}{\psi}{c}{\pi}{c'} :=
%%     \III{m}{n}{m}{n}{\phi}{\psi}{c}{\pi}{c'}.
%%   \]
%%   When $\pi = 1_{c}$ (thus $c' = c$), we abbreviate it further to
%%   \[
%%     \II{m'}{n'}{m}{n}{\phi}{\psi}{c} :=
%%     \III{m'}{n'}{m}{n}{\phi}{\psi}{c}{1_{c}}{c},
%%   \]
%%   and
%%   \[
%%     \II{}{}{m}{n}{\phi}{\psi}{c} :=
%%     \III{m}{n}{m}{n}{\phi}{\psi}{c}{1_{c}}{c}.
%%   \]
%% \end{notation}

%% \begin{remark}\label{remark/skein-nature-of-the-notation-I} (Skein Nature of the Notation ${}^{m'}_{n'}I^{m}_{n}$)

%%   \noindent Informally yet instructively, it is helpful to view
%%   $\III{m'}{n'}{m}{n}{\phi}{\psi}{c}{\pi}{c'}$ as a skein flowing from the right to
%%   the left, starting from $m \boxtimes n$ to $m' \boxtimes n'$, passing through $\phi \boxtimes \psi$;
%%   during the process, the upper strain emits a particle $c$ at $\phi$, which
%%   transforms to via $\pi$ to $\overline{c}$, and hits the lower strain at
%%   $\psi$. Hence under the defined composition rule (see the equation for
%%   $\phi'' \otimes \pi'' \otimes \psi''$ above), we have
%%   \[
%%     \III{m''}{n''}{m'}{n'}{\phi'}{\psi'}{c'}{\pi'}{\overline{c}'} \circ
%%     \III{m'}{n'}{m}{n}{\phi}{\psi}{c}{\pi}{\overline{c}} =
%%     \III{m''}{n''}{m}{n}{\phi''}{\psi'}{c' \otimes c}{\pi''}{\overline{c}' \otimes \overline{c}}.
%%   \]
%% \end{remark}

%% \begin{remark}\label{remark/hom-space-reduction} (Hom Space Reduction)
%%   \noindent By the relations in the definition, the transformation
%%   \[
%%     \pi = 1_{\overline{c}} \circ \pi = \pi \circ 1_{c}
%%   \]
%%   can be absorbed into each of the strands, so
%%   \[
%%     \II{m'}{n'}{m}{n}{\phi}{{}_{\pi}\psi}{c} =
%%     \III{m'}{n'}{m}{n}{\phi}{{}_{\pi}\psi}{c}{1_{c}}{c} =
%%     \III{m'}{n'}{m}{n}{\phi}{\psi}{c}{\pi}{c'} =
%%     \III{m'}{n'}{m}{n}{\phi_{\pi}}{\psi}{c'}{1_{c'}}{c'} =
%%     \II{m'}{n'}{m}{n}{\phi_{\pi}}{\psi}{c'}.
%%   \]

%%   \noindent Furthermore, if $c \simeq \oplus_{i=1}^{l} c_{i}$, then
%%   \[
%%     \II{m'}{n'}{m}{n}{\phi}{\psi}{c} = \II{m'}{n'}{m}{n}{\phi}{\psi}{\oplus_{i=1}^{l} c_{i}} = \sum_{j=1}^{l} \III{m'}{n'}{m}{n}{\phi_{j}}{\psi}{c_{j}}{\iota_{j}}{\oplus_{i=1}^{l} c_{i}} =
%%     \sum_{j=1}^{l}\II{m'}{n'}{m}{n}{\phi_{j}}{\psi_{j}}{c_{j}},
%%   \]
%%   where $\iota_{j}$ denotes the $j$-th embedding map from $c_{j}$ to
%%   $\oplus_{i=1}^{l}c_{i}$, and the $\phi_{j}, \psi_{j}$'s denote the $j$-th
%%   projection of $\phi, \psi$ respectively. In particular, when
%%   $c \simeq x^{\oplus l}$, the this gives a reduction from
%%   \[
%%     Hom_{M}(m, m' \rhd x^{\oplus l}) \otimes Hom_{C}(x^{\oplus l}, x^{\oplus l}) \otimes Hom_{N} (x^{\oplus l} \lhd n, n')
%%   \]
%%   to
%%   \[
%%     Hom_{M}(m, m' \rhd x) \otimes Hom_{C}(x, x) \otimes Hom_{N} (x \lhd n, n').
%%   \]
%%   It is helpful to regard the result as the ``inner product'' of $\phi$ and $\psi$.
%% \end{remark}

%% \noindent The hom vector spaces are finite dimensional. An explicit basis is
%% constructed in \ref{proposition/basis-theorem}.

%% \begin{definition} \label{definition/karoubi-completion} (Karoubi Completion)

%%   \noindent Let $C$ be a category. \quad The Karoubi completion $Kar(C)$ of
%%   $C$ is defined to be the category with
%%   \[
%%     Obj(Kar(C)) = \{(c, f) \,|\, c \in Obj(C), f \in End_{C}(c), f = f^{2}\}
%%   \] and
%%   \[
%%     Mor_{Kar(C)}((c,f), (c', f')) = \{\overline{f} \in Hom_{C}(c,c') \,|\, \overline{f}f = \overline{f} = f'\overline{f}\},
%%   \]
%%   with the obvious composition rule.
%% \end{definition}

%% \begin{remark} \label{remark/karoubi-retract} (Karoubi Retracts)

%%   \noindent It is straightforward to check that every object $(c,f)$ in
%%   $Kar(C)$ is a retract of the original object $c = (c, 1_{c})$. So we have
%%   \[
%%     (c, f) \xrightarrow{\iota} c = (c, 1_{c}) \xrightarrow{\pi} (c, f).
%%   \]
%% \end{remark}

%% \begin{definition}\label{definition/skein-category} (Skein Category)

%%   \noindent Let $C$ be a tensor category. Let $M_{C}$ and $_{C}N$ be module
%%   categories. \quad Define the skein category $sk(M,C,N)$ to the Karoubi
%%   completion of its pre-skein category
%%   \[
%%     sk(M,C,N) := Kar(p.sk(M,C,N)).
%%   \]
%%   \noindent Thus, a typical object of the skein category $sk(M,C,N)$ is an
%%   idempotent matrix of skeins (the typical object is an idempotent
%%   endomorphism of a direct sum $\oplus_{i} (m_{i} \boxtimes n_{i})$, and an
%%   endomorphism of such a direct sum is just a matrix whose entries are maps
%%   $(m_{i} \boxtimes n_{i}) \to (m_{j} \boxtimes n_{j})$, i.e. skeins like
%%   $\II{m_{j}}{n_{j}}{m_{i}}{n_{i}}{\phi}{\psi}{c}$). The skein category is
%%   obviously enriched over $\Vect_{\mathbb{C}}$.

%%   More generally, for any $n \in \mathbb{N}$, let
%%   $C_{0}, C_{1}, \ldots, C_{n}$ be tensor categories. Let
%%   ${}_{C_{0}}M^{1}_{C_{1}}, \, {}_{C_{1}}M^{2}_{C_{2}}, \ldots, {}_{C_{n{\text -}1}}M^{n}_{C_{n}}$
%%   be bimodule categories. One can in a similar way define the
%%   $C_{0}{\text -}C_{n}$ bimodule category
%%   \[
%%     sk(M^{1}, C_{1}, M^{2}, C_{2}, M^{3}, \ldots, C_{n{\text -}1}, M^{n}).
%%   \]
%%   \begin{center}
%%     \includesvg[width=12cm]{drawing-3}
%%   \end{center}

%% \end{definition}

%% \begin{definition} \label{definition/induced-functor-on-skein-category} (Induced Functor on Skein Category)

%%   \noindent Assume the notation in \ref{definition/skein-category}. Let
%%   $1 \leq i \leq n$, and let $F: M^{i} \to M'^{i}$ be a
%%   $C_{i{\text -}1}{\text -}C_{i}$ bimodule category functor. \quad Then $F$
%%   naturally induces a linear functor between the skein categories
%%   \[
%%     sk(M^{1}, C_{1}, M^{2}, C_{2}, M^{3}, \ldots M^{i} \ldots, C_{n{\text -}1}, M^{n})
%%     \xrightarrow{F}
%%     sk(M^{1}, C_{1}, M^{2}, C_{2}, M^{3}, \ldots M'^{i} \ldots, C_{n{\text -}1}, M^{n}).
%%   \]
%%   For example, if $n=2$, $M^{1} = M$, $M^{2} = N$, $C^{1} = C$, $N \xrightarrow{F} N'$
%%   (with the left $C$-module structure given by $\alpha$), then we have an
%%   induced linear functor $sk(M,C,N) \to sk(M,C,N')$ sending the objects and morphisms via the map:
%%   \[
%%     \II{m'}{n'}{m}{n}{\phi}{\psi}{c}
%%     \quad \mapsto \quad
%%     \II{m'}{F(n')}{m}{F(n)}{\phi}{F(\psi) \circ \alpha}{c}.
%%   \]
%% \end{definition}

%% \noindent The main result of this paper is to show that the skein category
%% $sk(M,C,N)$ is equivalent to $M \boxtimes_{C} N$, and that the induced functor $M \boxtimes_{C} N \to M \boxtimes_{C} N'$
%% coincides with the one in \ref{definition/induced-functor-on-skein-category}
%% (proven in \ref{lemma/main-lemma}, \ref{theorem/main-theorem}). A necessary
%% ingredient is the canonical map $\boxtimes_{C}$ given in the defining universal
%% property.

%% \begin{definition}\label{definition/canonical-map} (Canonical Map $\boxtimes_{C}$)

%%   \noindent Let $C$ be a tensor category, and $M_{C}, _{C}N$ be module
%%   categories. \quad Define the functor
%%   \[
%%     M \times N \xrightarrow{\boxtimes_{C}} sk(M,C,N)
%%   \]
%%   to send the object $(m,n)$ to the object $\II{m}{n}{m}{n}{1_{m}}{1_{n}}{1_{1}}$, and the morphism
%%   \[
%%     (m,n) \xrightarrow{(\phi, \psi)} (m', n')
%%   \]
%%   to the morphism $\II{m'}{n'}{m}{n}{\phi}{\psi}{1_{1}}$.
%% \end{definition}

%% \noindent From the main theorem, we must have $sk(M,C,C) \simeq M$. To quickly
%% convince the reader that the main theorem is true before it is proven, we
%% provide another direct proof for this equivalence in the appendix (cf
%% \ref{proposition/degenerated-main-theorem}).

%% \hfill\break
%% \noindent We prove another lemma that will be useful later.

%% \begin{lemma}\label{lemma/I-provides-subobject} (Objects are Retracts)

%%   \noindent Let $C$ be a tensor category, and $M_{C}, _{C}N$ be module
%%   categories. \quad Then any typical object $\II{}{}{m}{n}{\phi}{\psi}{c}$ in
%%   $sk(M,C,N)$ is a retract (in particular, a subobject) of the canonical
%%   object $\boxtimes_{C}((m,n)) = \II{}{}{m}{n}{1_{m}}{1_{n}}{1}$.
%% \end{lemma}
%% \begin{proof}
%%   This directly follows from \ref{remark/karoubi-retract}.
%% \end{proof}



%% \subsection{Proof of the Main Equivalence Theorem}\label{section/proof-of-equivalence}

%% Unless specified otherwise, throughout this section, let $C, D, E$ be tensor
%% categories, let $M_{C}$ and $_{C}N$ be module categories, and let $L$ be a
%% linear category. We prove our main theorem (\ref{theorem/main-theorem}) in this section, justifying
%% that the skein construction $sk(M,C,N)$ is isomorphic to $M \boxtimes_{C} N$, and the
%% obvious generalization to the case of more bimodule categories.

%% \begin{lemma}\label{lemma/construction-of-theta} (Construction of $\Theta$)

%%   \noindent There exists a linear functor
%%   \[
%%     \Theta: Fun(sk(M,C,N), L) \to Fun^{C{\text -}bal}(M \times N, L).
%%   \]
%% \end{lemma}

%% \noindent We construct $\Theta$ explicitly in the proof.

%% \begin{proof}
%%   \noindent (Object) Let $G$ be an object of the domain. Define $\Theta(G)$ to
%%   be $F := G \circ \boxtimes_{C} \in Fun(M \times N, L)$. We shall provide the
%%   balanced structure $\alpha$ for $F$, so that $(F, \alpha)$ is $C$-balanced.
%%   We need to provide the $C$-balanced data for $F$:
%%   \[
%%     \alpha_{m,c,n}: F(m \lhd c, n) =
%%     G(\II{}{}{mc}{n}{1}{1}{1})
%%     \xrightarrow{\sim} G(\II{}{}{m}{cn}{1}{1}{1})
%%     = F(m, c \rhd n),
%%   \]
%%   which is clearly satisfied by $G(\II{m}{cn}{mc}{n}{1}{1}{c}).$ So defined
%%   $\alpha$ is clearly natural, and it is invertible by using the right dual of $c$.

%%   \noindent (Morphism) Let $G \xrightarrow{\eta} G'$ be a morphism in
%%   $Fun(sk(M,C,N), L)$. Its image under $\Theta$ is simply the horizontal
%%   composition $\eta \star (1_{\boxtimes_{C}})$. The remaining commutativity to be checked
%%   % cf p11 of Jin's note 20240906-110000
%%   is a direct consequence of $\eta$'s naturality.
%% \end{proof}

%% \begin{lemma}\label{lemma/theta-is-faithful} ($\Theta$ is faithful)

%%   \noindent The linear functor $\Theta$ (cf. \ref{lemma/construction-of-theta}) is faithful.
%% \end{lemma}

%% \begin{proof}
%%   (We use the same notation found in this section.) This amounts to showing
%%   that the following map is an injective linear map:
%%   \[
%%     (G \xrightarrow{\eta} G') \mapsto (F \xrightarrow{\Theta(\eta) = \eta \star (1_{\boxtimes_{C}})} F').
%%   \]
%%   It is clearly linear. For injectivity, we notice that whenever we have
%%   linear functors
%%   \[
%%     X \xrightarrow{f} Y,\quad Y \xrightarrow{g, g'} Z,
%%   \]
%%   and a linear natural transformation $\eta: g \to g'$, then the map $(\eta \mapsto \eta \star 1_{f})$ is injective is equivalent to
%%   \[
%%     (\forall x \in Obj(X), \eta_{f(x)} = 0) \Rightarrow (\forall y \in Obj(Y), \eta_{y} = 0).
%%   \]
%%   This holds if $f$ is surjective on objects. However, in our case
%%   $f = \boxtimes_{C}$ is not as strong. Fortunately, clearly it also holds if
%%   $f$ is almost-surjective, in the sense that each $y \in Obj(Y)$ has an
%%   $x \in Obj(X)$ such that $y$ is a retract of $f(X)$. Indeed,
%%   \[
%%     1_{g(y)} = g(1_{y}) = g(\pi_{y} \circ \iota_{y}),
%%   \]
%%   so
%%   \[
%%     (g(y) \xrightarrow{\eta_{y}} g'(y)) = \eta_{y} \circ 1_{g(y)} = \eta_{y} \circ g(\pi \circ \iota) = g(\pi) \circ \eta_{f(x)} \circ g(\iota) = g(\pi) \circ 0 \circ g(\iota) = 0.
%%   \]
%%   This applies to our case by putting $f = \boxtimes_{C}$ and $g = G$, because
%%   each object $\II{}{}{m}{n}{\phi}{\psi}{c}$ is clearly a retract of
%%   $\boxtimes_{C}(m,n) = \II{}{}{m}{n}{1}{1}{1}$. Therefore, $\Theta$ is faithful.
%% \end{proof}

%% \begin{lemma}\label{lemma/theta-is-full} ($\Theta$ is full)

%%   \noindent The linear functor $\Theta$ (cf. \ref{lemma/construction-of-theta}) is full.
%% \end{lemma}

%% \begin{proof}
%%   We need to show that for any $C$-balanced natural transformation
%%   \[
%%     \nu: G \circ \boxtimes_{C} = \Theta(G) \to \Theta(G') = G' \circ \boxtimes_{C}
%%   \]
%%   there is $\mu: G \to G'$ such that $\nu = \mu \star 1_{\boxtimes_{C}}$. The data $\nu$ are the maps
%%   \[
%%     \nu_{(m,n)}: G(\II{}{}{m}{n}{1}{1}{1}) \to G'(\II{}{}{m}{n}{1}{1}{1}).
%%   \]
%%   We only need to extend these data to all objects in $sk(M,C,N)$, i.e. define compatible maps
%%   \[
%%     \nu_{\II{}{}{m}{n}{\phi}{\psi}{c}}: G(\II{}{}{m}{n}{\phi}{\psi}{c}) \to G'(\II{}{}{m}{n}{\phi}{\psi}{c}).
%%   \]
%%   It is straightforward to check that the following works:
%%   \[
%%     \nu_{\II{}{}{m}{n}{\phi}{\psi}{c}}:= G(\II{}{}{m}{n}{\phi}{\psi}{c})
%%     \xrightarrow{G(\iota)}
%%     G(\II{}{}{m}{n}{1}{1}{1})
%%     \xrightarrow{\nu_{(m,n)}}
%%     G'(\II{}{}{m}{n}{1}{1}{1})
%%     \xrightarrow{G'(\pi)}
%%     G'(\II{}{}{m}{n}{\phi}{\psi}{c}),
%%   \]
%%   where $\iota$ and $\pi$ are the inclusion and projection (see
%%   Lemma \ref{lemma/I-provides-subobject}).
%% \end{proof}

%% \noindent To prove that $\Theta$ is essentially surjective, we need the following
%% lemma.

%% \begin{lemma} (Images of skeins) \label{lemma/image-of-skein}
%%   % Discussion: Can we relax conditions? We need to use the basis theorem,
%%   % so it seems that we must require semisimplicity here.
%%   \noindent
%%   Let $(F,\alpha) \in Fun^{C{\text -}bal}(M \times N, L)$. For each morphism
%%   in $sk(M,C,N)$ of the form $\III{m'}{n'}{m}{n}{\phi}{\psi}{c}{\pi}{c'}$,
%%   define the image of it under $\tilde{F}$ to be the composed morphism
%%   \[
%%     F(m,n)
%%     \xrightarrow{F(\phi \times 1)}
%%     F(m' \lhd c, n)
%%     \xrightarrow{F((1 \lhd \pi) \times 1)}
%%     F(m' \lhd c', n)
%%     \xrightarrow[\sim]{\alpha}
%%     F(m', c' \rhd n)
%%     \xrightarrow{F(1 \times \psi)}
%%     F(m',n').
%%   \]
%%   Suppose we have two skeins $\II{m'}{n'}{m}{n}{\phi}{\psi}{c}$ and
%%   $ \II{m'}{n'}{m}{n}{\phi'}{\psi'}{c'}$ as identical morphisms in $sk(M,C,N)$
%%   (recall the definitional relations (\ref{relation/a}) (\ref{relation/b})).

%%   \noindent Then
%%   \begin{equation} \label{eqn/a}
%%     \tilde{F}(\II{m'}{n'}{m}{n}{\phi}{\psi}{c}) = \tilde{F}(\II{m'}{n'}{m}{n}{\phi'}{\psi'}{c'}).
%%   \end{equation}
%%   Moreover, $\tilde{F}$ preserves compositions, i.e.
%%   \begin{equation} \label{eqn/b}
%%     \tilde{F}(\II{m''}{n''}{m'}{n'}{\overline{\phi}}{\overline{\psi}}{\overline{c}} \circ \II{m'}{n'}{m}{n}{\phi}{\psi}{c})
%%     = \tilde{F}(\II{m''}{n''}{m'}{n'}{\overline{\phi}}{\overline{\psi}}{\overline{c}})
%%     \circ
%%     \tilde{F}(\II{m'}{n'}{m}{n}{\phi}{\psi}{c}).
%%   \end{equation}
%% \end{lemma}

%% \begin{proof}
%%   To prove the first statement (\ref{eqn/a}), check the equality against the
%%   definitional relations (\ref{relation/a}) (\ref{relation/b}).

%%   \noindent To prove the second statement (\ref{eqn/b}), note that the
%%   left-hand-side is
%%   $\tilde{F}(\II{m''}{n'}{m}{n}{\overline{\phi} \otimes \phi}{\overline{\psi} \otimes \psi}{\overline{c} \otimes c}),$
%%   which is
%%   \begin{multline*}
%%     F(m,n)
%%     \xrightarrow{F(\phi \times 1)}
%%     F(m' \lhd c, n)
%%     \xrightarrow{F((\overline{\phi} \lhd 1) \times 1)} \\
%%     F((m'' \lhd \overline{c}) \lhd c, n)
%%     \xrightarrow[\sim]{}
%%     F((m'' \lhd (\overline{c} \otimes c)), n)
%%     \xrightarrow[\sim]{\alpha}
%%     F(m'', (\overline{c} \otimes c) \rhd n)
%%     \xrightarrow[\sim]{}
%%     F(m'', \overline{c} \rhd (c \rhd n)) \\
%%     \xrightarrow{F(1 \times (1 \rhd \psi))}
%%     F(m'', \overline{c} \rhd n')
%%     \xrightarrow{F(1 \times \overline{\psi})}
%%     F(m'',n'').
%%   \end{multline*}
%%   On the other hand, the right-hand-side is
%%   \begin{multline*}
%%     F(m,n)
%%     \xrightarrow{F(\phi \times 1)}
%%     F(m' \lhd c, n)
%%     \xrightarrow[\sim]{\alpha} \\
%%     F(m', c \rhd n)
%%     \xrightarrow{F(1 \times \psi)}
%%     F(m', n')
%%     \xrightarrow{F(\overline{\phi} \times 1)}
%%     F(m'' \rhd \overline{c}, n') \\
%%     \xrightarrow[\sim]{\alpha}
%%     F(m'', \overline{c} \lhd n')
%%     \xrightarrow{F(1 \times \overline{\psi})}
%%     F(m'',n'').
%%   \end{multline*}
%%   To prove that they are equal, we can omit their first and their last arrows. Note that the composed arrow in left-hand-side $F(m' \lhd c, n) \to F(m', \overline{c} \rhd (c \rhd n))$ is, by the naturality of $\alpha$, equal to
%%   \[
%%     F(m' \lhd c, n)
%%     \xrightarrow[\sim]{\alpha}
%%     F(m', c \rhd n)
%%     \xrightarrow{F(\overline{\phi} \times 1)}
%%     F(m'' \lhd \overline{c}, c \rhd n)
%%     \xrightarrow[\sim]{\alpha}
%%     F(m'', \overline{c} \rhd (c \rhd n)).
%%   \]
%%   Compose this with
%%   \[
%%     F(m'', \overline{c} \rhd (c \rhd n))
%%     \xrightarrow{F(1 \times (1 \rhd \psi))}
%%     F(m', \overline{c} \rhd n'),
%%   \]
%%   then we get
%%   \[
%%     F(m' \lhd c, n)
%%     \xrightarrow[\sim]{\alpha}
%%     F(m', c \rhd n)
%%     \xrightarrow{F(\overline{\phi} \times 1)}
%%     F(m'' \lhd \overline{c}, c \rhd n)
%%     \xrightarrow{F(1 \times \psi)}
%%     F(m'' \lhd \overline{c}, n'),
%%   \]
%%   which is equal to
%%   \[
%%     F(m' \lhd c, n)
%%     \xrightarrow[\sim]{\alpha}
%%     F(m', c \rhd n)
%%     \xrightarrow{F(\overline{\phi} \times \psi)}
%%     F(m'' \lhd \overline{c}, n').
%%   \]
%%   So both sides are equal.
%% \end{proof}

%% \begin{lemma}\label{lemma/theta-is-essentially-surjective} ($\Theta$ is essentially surjective)

%%   \noindent The linear functor $\Theta$ (cf. \ref{lemma/construction-of-theta}) is essentially surjective.
%% \end{lemma}

%% \begin{proof}
%%   Let $(F, \alpha) \in Fun^{C{\text -}bal}(M \times N, L)$. It suffices to construct
%%   $G \in Fun(sk(M,C,N), L)$ such $\Theta(G) \simeq (F,\alpha)$. Recall that $L$ is an abelian
%%   category (so each $L$-morphism has an image), $F: M \times N \to L$ is a linear
%%   functor, and that
%%   \[
%%     F(m \lhd c, n) \xrightarrow[\sim]{\alpha_{m,c,n}} F(m, c \rhd n).
%%   \]

%%   \noindent ($G$ on objects) Recall the definition and properties of
%%   $\tilde{F}$ in \ref{lemma/image-of-skein}. Define
%%   $G(\II{}{}{m}{n}{\phi}{\psi}{c})$ to be the image (in $L$) of the
%%   $L$-morphism $\tilde{F}(\II{}{}{m}{n}{\phi}{\psi}{c})$. In particular, the
%%   image is a subobject and a quotient of $F(m,n)$.

%%   \noindent ($G$ on morphisms) We use $\tilde{F}$ again. Let
%%   $\II{m'}{n'}{m}{n}{\overline{\phi}}{\overline{\psi}}{\overline{c}}$ be a morphism
%%   from $\II{}{}{m}{n}{\phi}{\psi}{c}$ to $\II{}{}{m'}{n'}{\phi'}{\psi'}{c'}$. Define
%%   $G(\II{m'}{n'}{m}{n}{\overline{\phi}}{\overline{\psi}}{\overline{c}})$ to be the
%%   map induced by
%%   \[
%%     \tilde{F}(\II{m'}{n'}{m}{n}{\overline{\phi}}{\overline{\psi}}{\overline{c}}): F(m,n) \to F(m',n').
%%   \]
%%   To justify this definition, we must show that
%%   \[
%%     ker(
%%     \tilde{F}(\II{}{}{m'}{n'}{\phi'}{\psi'}{c'})
%%     \circ
%%     \tilde{F}(\II{m'}{n'}{m}{n}{\overline{\phi}}{\overline{\psi}}{\overline{c}})
%%     )
%%     \supseteq
%%     ker(\tilde{F}(\II{}{}{m}{n}{\phi}{\psi}{c})).
%%   \]
%%   By lemma \ref{lemma/image-of-skein}, $\tilde{F}$ respects compositions, so
%%   \[
%%     ker(
%%     \tilde{F}(\II{}{}{m'}{n'}{\phi'}{\psi'}{c'})
%%     \circ
%%     \tilde{F}(\II{m'}{n'}{m}{n}{\overline{\phi}}{\overline{\psi}}{\overline{c}})
%%     )
%%     =
%%     ker(
%%     \tilde{F}(\II{}{}{m'}{n'}{\phi'}{\psi'}{c'}
%%     \circ
%%     \II{m'}{n'}{m}{n}{\overline{\phi}}{\overline{\psi}}{\overline{c}})
%%     ).
%%   \]
%%   Then by the definition of $sk(M,N,C)$ and Karoubi completion,
%%   \[
%%     ker(
%%     \tilde{F}(\II{}{}{m'}{n'}{\phi'}{\psi'}{c'}
%%     \circ
%%     \II{m'}{n'}{m}{n}{\overline{\phi}}{\overline{\psi}}{\overline{c}})
%%     )
%%     =
%%     ker(
%%     \tilde{F}(
%%     \II{m'}{n'}{m}{n}{\overline{\phi}}{\overline{\psi}}{\overline{c}}
%%     \circ
%%     \II{}{}{m}{n}{\phi}{\psi}{c}
%%     )
%%     ).
%%   \]
%%   The final step is completed by using the composing property of $\tilde{F}$ again and the fact that
%%   $ker(a \circ b) \supseteq ker(b)$.
%% \end{proof}

%% \begin{lemma} (Main Lemma) \label{lemma/main-lemma}

%%   \noindent The canonical map (\ref{definition/canonical-map})
%%   $\boxtimes_{C}$: $M \times N \to sk(M,C,N)$ satisfies the universal property
%%   in the definition of $M \boxtimes_{C} N$. In particular, we have an
%%   equivalence of categories
%%   \[
%%     sk(M,C,N) \simeq M \boxtimes_{C} N.
%%   \]
%% \end{lemma}

%% \begin{proof}
%%   We only need to show that $sk(M,C,N)$ is a $k$-linear category (in
%%   particular, an abelian category, following the definition in
%%   \cite{douglas/balanced-product}), and that $\boxtimes_{C}$ induces an equivalence of
%%   categories
%%   \[
%%     Fun(sk(M,C,N), L) \xrightarrow[\sim]{\Theta} Fun^{C{\text -}bal}(M \times N, L).
%%   \]
%%   For $k$-linearity, the crux is to show that the skein category is abelian.
%%   We postpone the proof to the appendix \ref{semisimple}. For the second
%%   statement, we constructed $\Theta$ in (\ref{lemma/construction-of-theta}), proved that $\Theta$ is faithfulness in
%%   (\ref{lemma/theta-is-faithful}), is full in (\ref{lemma/theta-is-full}), and is essentially surjective in (\ref{lemma/theta-is-essentially-surjective}).
%% \end{proof}


%% \begin{theorem} (Main Theorem: Skein Construction of Balanced Tensor Product) \label{theorem/main-theorem}

%%   \noindent (1) Let ${}_{C}M_{D}, \, {}_{D}N_{E}$ be, bimodule categories.
%%   \quad Then the canonical map (\ref{definition/canonical-map})
%%   $\boxtimes_{D}$: $M \times N \to sk(M,D,N)$ satisfies the universal property
%%   in the definition of ${}_{C}M_{D} \boxtimes_{D} {}_{D}N_{E}$. In particular,
%%   we have an equivalence of $C{\text -}E$ bimodule categories.
%%   \[
%%     {}_{C}sk(M,D,N)_{E} \simeq {}_{C}M_{D} \boxtimes_{D} {}_{D}N_{E}.
%%   \]

%%   \noindent (2) More generally, for any $n \in \mathbb{N}$, let
%%   $C_{0}, C_{1}, \ldots, C_{n}$ be tensor categories. Let
%%   ${}_{C_{0}}M^{1}_{C_{1}}, \, {}_{C_{1}}M^{2}_{C_{2}}, \ldots, {}_{C_{n{\text -}1}}M^{n}_{C_{n}}, \, $
%%   be bimodule categories. \quad Then we have an equivalence of
%%   $C_{0}{\text -}C_{n}$ bimodule categories.
%%   \[
%%     {}_{C_{0}}sk(M^{1},C_{1},M^{2},C_{2}, \ldots, C_{n{\text -}1}, M^{n})_{C_{n}}
%%     \simeq
%%     {}_{C_{0}}(M^{1}
%%     \boxtimes_{C_{1}}
%%     M^{2}
%%     \boxtimes_{C_{2}}
%%     M^{3}
%%     \ldots
%%     \boxtimes_{C_{n{\text -}1}}
%%     M^{n})_{C_{n}}.
%%   \]
%%   \noindent (3) Moreover, in addition to the previous part, if
%%   $F^{i}: M^{i} \to M'^{i}$ is a $C_{i{\text -}1}{\text -}C_{i}$ bimodule category
%%   functor, then the naturally induced linear functor (cf.
%%   \ref{definition/induced-functor-on-skein-category})
%%   \[
%%     F^{i}:
%%     sk(M^{1}, C_{1}, M^{2}, C_{2}, M^{3}, \ldots M^{i} \ldots, C_{n{\text -}1}, M^{n})
%%     \to
%%     sk(M^{1}, C_{1}, M^{2}, C_{2}, M^{3}, \ldots M'^{i} \ldots, C_{n{\text -}1}, M^{n}),
%%   \]
%%   corresponds to the functor
%%   \[
%%     F^{i}: M^{1} \boxtimes_{C_{1}} M^{2} \boxtimes_{C_{2}} M^{3} \boxtimes_{C_{3}} \ldots M^{i} \ldots \boxtimes_{C_{n{\text -}1}} M^{n} \to M^{1} \boxtimes_{C_{1}} M^{2} \boxtimes_{C_{2}} M^{3} \boxtimes_{C_{3}} \ldots M'^{i} \ldots \boxtimes_{C_{n{\text -}1}} M^{n}.
%%   \]
%%   under the equivalence.
%% \end{theorem}

%% \begin{proof}
%%   The first part is proved by restricting the proof of lemma \ref{lemma/main-lemma} to $C{\text -}E$ bimodule maps.
%%   The second part follows directly from induction and the first part. The third part is obvious.
%% \end{proof}

%% \begin{remark} (Application on the Turaev-Viro model)

%%   \noindent That the induced functor on the skein category coincides with the
%%   algebraic one is the key for computing values of the Turaev-Viro model in
%%   dimensions $(1+1)$ \cite{guu/tv-as-3-functor}.
%% \end{remark}

%% % The following corollary is commented out. The statement itself is not
%% % incorrect, but the proof is wrong (circular). We must have proven that they
%% % are abelian before showing that skein categories are actually balanced
%% % tensor products.
%% % \begin{corollary}\label{corollary/skein-category-is-abelian} (Skein Categories are Abelian)
%% %   \noindent For any $n \in \mathbb{N}$, let $C_{0}, C_{1}, \ldots, C_{n}$ be
%% %   tensor categories. Let
%% %   \[
%% %     {}_{C_{0}}M^{1}_{C_{1}}, \, {}_{C_{1}}M^{2}_{C_{2}}, \ldots, {}_{C_{n{\text -}1}}M^{n}_{C_{n}}, \,
%% %   \]
%% %   be bimodule categories. Then the skein category (cf
%% %   \ref{definition/skein-category})
%% %   \[
%% %     sk(M^{1}, C_{1}, M^{2}, C_{2}, M^{3}, \ldots M^{i} \ldots, C_{n{\text -}1}, M^{n})
%% %   \]
%% %   is an abelian category.
%% % \end{corollary}
%% % \begin{proof}
%% %   The proof for the case $n=2$ follows from the fact that
%% %   $M^{1} \boxtimes_{C_{1}} M^{2} \simeq Z_{C_{1}}(M^{1} \boxtimes M^{2})$ is
%% %   abelian (cf \cite{kirillov/fact-homo-4d-tqft}). The rest follows from
%% %   induction.
%% % \end{proof}
